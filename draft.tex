% Suha's Proposal
\documentclass{article}


\usepackage{amssymb}
\usepackage{graphicx}
\usepackage{listings}
\usepackage{arydshln}
\usepackage{amsmath}
\usepackage{amsthm}
\usepackage{color}

\usepackage{amsfonts}
\usepackage{xspace}
\usepackage{latexsym}
\usepackage{wasysym} % causing problems with llncs's bold vectors
\usepackage{stmaryrd}
\usepackage{alltt}
\usepackage{mathpartir}
\usepackage[ligature,reserved,inference]{semantic}
\usepackage[lined]{algorithm2e}
\def\ttat{\mtt{@}} % the at package clobbers this
\usepackage{at}
\usepackage{alltt}


%\usepackage[numbers]{natbib}

%\input macros

\newtheorem{dfn}{Definition}
\newtheorem{lem}{Lemma}

\title{Existence of Forward Simulations for Particular Data Structures}
\begin{document}
\maketitle
\section{Preliminaries}
Systems we consider are labeled transition systems (LTS):
\begin{dfn}
An LTS is defined over four-tuples $A=(Q,\Sigma, q_0, \delta)$ where $Q$ is the set of states, $\Sigma$ is the set of transition labels, $q_0 \in Q$ is the initial state and $\delta \subseteq Q \times \Sigma \times Q$ is the transition relation.
\end{dfn}
Executions generated by this system are alternating sequence of states and transition labels $\rho = s_0, e_0, s_1,... s_k, e_k,...$ where each $s_i \in Q$, each $e_i \in \Sigma$, $s_0 = q_0$ and each $(s_i, e_i s_{i+1}) \in \delta$. The projection of the sequence $\rho$ over the set $\Pi$ is denoted by $\rho | \Pi$, and it is the maximum subsequence of $\rho$ consisting of elements of $\Pi$. Traces of the LTS are obtained from executions by projecting them over $\Sigma$. For the rest of the paper and in all of the proofs, we consider only finite executions (denoted as $E(A)$) and/or traces (denoted as $Tr(A)$ of the LTSs in focus.

Libraries are LTSs that provide methods. Let $\mathcal{M}$ be the set of method names and $\mathcal{D}$ be the domain of values as input/output parameters for the methods. Then, this library contains transition labels of the form $inv(m,d,i)$ representing the invocation of method $m \in \mathcal{M}$ with input value $d \in \mathcal{D}$. The third field is the operation identifier for differentiating the different calls of the same method from the set $\mathcal{O}$. For simplicity, we take $\mathcal{O} = \mathbb{N}$ for the rest of the paper. We also assume that methods could have at most one input parameter. If they do not have any input arguments (like pop method of a stack), we can omit the second field from the action. They also provide actions of the form $ret(m,d,i)$ representing the return of method $m \in M$ with value $d \in D$ which has been invoked previously with action $inv(m,d',i)$. Again, we assume that the methods can return at most one parameter and we may omit the second field from the action if they have none (like enqueue method of a queue). Before starting to reason about any set of libraries, we first fix the sets $\mathcal{M}$ and $\mathcal{D}$ and libraries in our focus agree on this sets. For any transition label $e = inv(m,d,i)$ or $e=ret(m,d,i)$, we have the function $oid(e) = i$.

Since libraries are LTSs, they produce traces. A trace $e = e_1, e_2, ..., e_n$ of library $L$ is \emph{well-formed} iff (i) every return matches an earlier invocation: $e_j = ret(m,d,k)$ implies that there exists $i<j$ such that $e_i = inv(m,d',k)$ and (ii) every operation identifier is used at most one invocation/return pair: $oid(e_i) = oid(e_j) = k$ and $i<j$ implies $e_i = inv(m,d,k)$ and $e_j = ret(m,d',k)$. From now on, we assume that libraries produce well-formed traces. Let $f: \mathbb{N} \rightarrow \mathbb(N)$ be a bijection. Then, traces $e$ and $e'$ are equivalent if $e'$ is obtained from $e$ by replacing every action $inv(m,d,k)$ with $inv(m,d,f(k))$ and every action $ret(m,d,k)$ with $ret(m,d,f(k))$. 

Based on these definitions on the traces of the libraries, we can define refinement between libraries:
\begin{dfn}
Let $L_1$ and $L_2$ be two libraries agreeing on $\mathcal{M}$ and $\mathcal{D}$ sets. We define the set $A\Sigma = \{inv(m,d,i)| m \in \mathcal{M} \wedge d \in \mathcal{D} \wedge i \in \mathbb{N}\} \cup \{ret(m,d,i)| m \in \mathcal{M}\wedge d \in \mathcal{D} \wedge i \in \mathbb{N}\}$ as abstract transition labels. Note that $A\Sigma \subseteq \Sigma_{L_1}$ and $A\Sigma \subseteq \Sigma_{L_2}$. Then, we say $L_1$ refines $L_2$ iff for every trace $e \in Tr(L_1)$, there exists a trace $e' \in Tr(L_2)$ such that $e|A\Sigma$ is equivalent to $e'|A\Sigma$.
\end{dfn}
Linearizability is also a relation between two libraries and it is stricter than refinement. It requires $e'$ in Definition 2 to be a sequential one. A trace $e$ is sequential iff following two conditions hold for its projection to abstract transition labels $e|A\Sigma = e_1, ...e_n$: $(i)$ $e_1 = inv(m,d,k)$ for some $m \in \mathcal{M}, d \in \mathcal{D} and k \in \mathbb{N}$ and $(ii)$ for all $i \in [1,n)$, either $e_i = inv(m,d,k)$ and $e_{i+1} = ret(m,d',k)$ or $e_i = ret(m,d,k)$ and $e_{i+1} = inv(m',d',k')$ for some $m,m' \in \mathcal{M}$, $d,d' \in \mathcal{D}$ and $k,k' \in \mathbb{N}$.

We can extend the relations between libraries by introducing simulation relations. We will later show that simulation relations imply refinement. 

\begin{dfn}
Let $L_1$ and $L_2$ be two libraries agreeing on $\mathcal{M}$ and $\mathcal{D}$ sets. Then, the relation $fs \subseteq Q_{L_1} \times Q_{L_2}$ is called a forward simulation iff the following holds:
\begin{itemize}
\item[(i)] $fs[q_{0_{L_1}}] = \{q_{0_{L_2}} \}$
\item[(ii-a)] If $(s,inv(m,d,k),s') \in \delta_{L_1}$ and $u \in fs[s]$, then there exists $u' \in fs[s']$ such that $u \xrightarrow{a} u'$ where $a = a_1, a_2, ..., a_n$ such that $a_1 = inv(m,d,k)$ and for all $i \in [2,n]$, $a_i \in \Sigma_{L_2} \backslash A\Sigma $. The expression $u \xrightarrow{a} u'$ means that there exists a sequence of states $u_1, u_2,...,u_{n+1}$ such that $u_1 = u$, $u_{n+1} = u'$ and for all $i \in [1,n]$, $(u_i, a_i, u_{i+1}) \in \delta_{L_2}$.
\item[(ii-b)] If $(s,ret(m,d,k),s') \in \delta_{L_1}$ and $u \in fs[s]$, then there exists $u' \in fs[s']$ such that $u \xrightarrow{a} u'$ where $a = a_1, a_2, ..., a_n$ such that $a_n = ret(m,d,k)$ and for all $i \in [1,n-1]$, $a_i \in \Sigma_{L_2} \backslash A\Sigma $. 
\item[(ii-c)] If $(s,t,s') \in \delta_{L_1}$ for some $t \in \Sigma_{L_1}\backslash A\Sigma$ and $u \in fs[s]$, then there exists $u' \in fs[s']$ such that $u \xrightarrow{a} u'$ where $a = a_1, a_2, ..., a_n$ such that for all $i \in [1,n]$, $a_i \in \Sigma_{L_2} \backslash A\Sigma $. Moreover, $a$ could be the empty sequence.
\end{itemize}
\end{dfn}
If $fs[s]$ is a unique state for all $s \in Q_{L_1}$ then it is called a refinement mapping/function. A dual notion of forward simulation is the backward simulation:
\begin{dfn}
Let $L_1$ and $L_2$ be two libraries agreeing on $\mathcal{M}$ and $\mathcal{D}$ sets. Then, the relation $bs \subseteq Q_{L_1} \times Q_{L_2}$ is called a backward simulation iff the following holds:
\begin{itemize}
\item[(i)] $bs[q_{0_{L_1}}] = \{q_{0_{L_2}} \}$
\item[(ii-a)] If $(s,inv(m,d,k),s') \in \delta_{L_1}$ and $u' \in bs[s']$, then there exists $u \in bs[s]$ such that $u \xrightarrow{a} u'$ where $a = a_1, a_2, ..., a_n$ such that $a_1 = inv(m,d,k)$ and for all $i \in [2,n]$, $a_i \in \Sigma_{L_2} \backslash A\Sigma $. 
\item[(ii-b)] If $(s,ret(m,d,k),s') \in \delta_{L_1}$ and $u' \in bs[s']$, then there exists $u \in bs[s]$ such that $u \xrightarrow{a} u'$ where $a = a_1, a_2, ..., a_n$ such that $a_n = ret(m,d,k)$ and for all $i \in [1,n-1]$, $a_i \in \Sigma_{L_2} \backslash A\Sigma $. 
\item[(ii-c)] If $(s,t,s') \in \delta_{L_1}$ for some $t \in \Sigma_{L_1}\backslash A\Sigma$ and $u' \in bs[s']$, then there exists $u \in bs[s]$ such that $u \xrightarrow{a} u'$ where $a = a_1, a_2, ..., a_n$ such that for all $i \in [1,n]$, $a_i \in \Sigma_{L_2} \backslash A\Sigma $. Moreover, $a$ could be the empty sequence.
\end{itemize}
\end{dfn}
Simulation relations are used to prove refinement relations among libraries. Following lemmas show their soundness:
\begin{lem}
Let $L_1$ and $L_2$ be two libraries agreeing on $\mathcal{M}$ and $\mathcal{D}$ sets. If $fs$ ($bs$) is a forward (backward) simulation relating $L_1$ to $L_2$, then $L_1$ refines $L_2$.
\end{lem}
\begin{proof}
Looks trivial and follows Lynch paper. Can be completed later.
\end{proof}
%!TEX root = draft.tex
\vspace{-3.5mm}
\section{Queues With Fixed Dequeue Linearization Points}\label{sec:queues}
\vspace{-1.5mm}

The classical abstract queue implementation, denoted $AbsQ_0$, maintains a
sequence of enqueued values; dequeues return the oldest non-dequeued value, at
the time of their linearization points, or {\tt EMPTY}. Some implementations,
like the queue of \citet{journals/toplas/HerlihyW90}, denoted {\sc hwq}, are not
forward-simulated by $AbsQ_0$, even though they refine $AbsQ_0$, since the order
in which their enqueues are linearized to form $AbsQ_0$’s sequence is not
determined until later, when their values are dequeued.

In this section we develop an abstract queue implementation, denoted $AbsQ$,
which maintains a partial order of enqueues, rather than a linear sequence.
Since $AbsQ$ does not force refining implementations to eagerly pick among
linearizations of their enqueues, it forward-simulates many more queue
implementations. In fact, $AbsQ$ forward-simulates all queue implementations of
which we are aware that are not forward-simulated by $AbsQ_0$, including {\sc
hwq}, The Baskets Queue~\cite{DBLP:conf/opodis/HoffmanSS07}, The Linked
Concurrent Ring Queue ({\sc lcrq})~\cite{DBLP:conf/ppopp/MorrisonA13}, and The
Time-Stamped Queue~\cite{DBLP:conf/popl/DoddsHK15}.


%TODO HOW CAN WE CONVINCE THAT OTHER IMPLEMENTATIONS HAVE THE SAME PROPERTY ?
\vspace{-3.5mm}
\subsection{Enqueue Methods With Non-Fixed Linearization Points}
\vspace{-1mm}
% describe a queue implementation listed in Figure~\ref{fig:HerlihyWing} and known as the Herlihy \& Wing Queue~\cite{journals/toplas/HerlihyW90} ($\mathit{HWQ}$ for short), where only the dequeue methods have fixed linearization points.
We describe $\mathit{HWQ}$ where the linearization points of the enqueue methods are not fixed.
The shared state consists of an array {\tt items} storing the values in the queue and a counter {\tt back} storing the index of the first unused position in {\tt items}. Initially, all the positions in the array are {\tt null} and {\tt back} is 0.
An enqueue method starts by reserving a position in {\tt items} ({\tt i} stores the index of this position and {\tt back} is incremented so the same position can't be used by other enqueues) and then, stores the argument {\tt x} at this position. The dequeue method traverses the array {\tt items} starting from the beginning and atomically swaps {\tt null} with the encountered value. If the value is not {\tt null}, then the dequeue returns that value. If it reaches the end of the array, then it restarts.

\begin{wrapfigure}{l}{5.3cm}
\vspace{-9mm}
\begin{lstlisting}
void enq(int x){
  i = back++;
  items[i] = x;
}
int deq() {
  while (1) {
    range = back - 1;
    for (int i = 0; i <= range; i++){
      x = swap(items[i],null);
      if ( x != null ) return x;
}}}
  \end{lstlisting}
\vspace{-5.5mm}
\caption{Herlihy \& Wing Queue. We assume that every statement is atomic.}
\label{fig:HerlihyWing}
\vspace{-6mm}
\end{wrapfigure}
The linearization points of the enqueues are not fixed, they depend on dequeues executing in the future. Consider the following trace with two concurrent enqueues (${\tt i}(k)$ represents the value of {\tt i} in operation $k$): $inv(enq,x,1)$, $inv(enq,y,2)$, ${\tt i}(1) = \mbox{{\tt bck++}}$, ${\tt i}(2) = \mbox{{\tt bck++}}$, ${\tt items[i(}2{\tt )]} = y$.
%\vspace{-1.5mm}
%\begin{align*}
%inv(enq,x,1)\ \ \ inv(enq,y,2)\ \ \ {\tt i}(1) = \mbox{{\tt bck++}}\ \ \ {\tt i}(2) = \mbox{{\tt bck++}}\ \ \ {\tt items[i(}2{\tt )]} = y
%\vspace{-1.5mm}
%\end{align*}
%\begin{align*}
%inv(enq,x,1)\ inv(enq,y,2)\ ({\tt i}_1 = 0,{\tt bck} = 1)\ ({\tt i}_2 = 1,{\tt bck} = 2)\ ({\tt items[1]} = y)
%\end{align*}
Assuming that the linearization point corresponds to the assignment of {\tt i}, the history of this trace should be linearized to $inv(enq,x,1)$, $ret(enq,1)$, $inv(enq,y,2)$, $ret(enq,2)$. However, a dequeue executing until completion after this trace will return $y$ (only position $1$ is filled in the array {\tt items}) which is not consistent with this linearization. On the other hand, assuming that enqueues should be linearized at the assignment of {\tt items[i]} and extending the trace with ${\tt items[i(}1{\tt )]} = x$ and a completed dequeue that in this case returns $x$, leads to the incorrect linearization: $inv(enq,y,2)$, $ret(enq,2)$, $inv(enq,x,1)$, $ret(enq,1)$, $inv(deq,3)$, $ret(deq,x,3)$.
%\vspace{-1.5mm}
%\begin{align*}
%inv(enq,y,2)\ ret(enq,2) inv(enq,x,1)\ ret(enq,1)\ inv(deq,3)\ ret(deq,x,3).
%\vspace{-1.5mm}
%\end{align*}

The dequeue method has a fixed linearization point which corresponds to an execution of {\tt swap} returning a non-null value. This action alone contributes to the effect of that value being removed from the queue. Every concurrent history can be linearized to a sequential history where dequeues occur in the order of their linearization points in the concurrent history.
This claim is formally proved in Section~\ref{ssec:HerlihyWing}.

Since the linearization points of the enqueues are determined by future dequeue invocations, there exists no forward simulation from $\mathit{HWQ}$ to $AbsQ_0$.
In the following, we describe the abstract implementation $AbsQ$ for which such a forward simulation does exist.

\vspace{-3.5mm}
\subsection{Abstract Queue Implementation}
\vspace{-1.5mm}

%\textcolor{red}{Important: I think EMPTY return is problematic and I defined it wrong in my machine too. We should be able to return EMPTY if there are only pending nodes. Consider the following history of $AbsQ_0$: $inv(enq, d_1,k_1), inv(enq, d_2,k_2), inv(deq, k_3), lin(deq, \texttt{EMPTY}, k_3)$. This history should be reflected in $AbsQ$ by enabling lin \texttt{EMPTY} of dequeue when there are pending nodes. We also need to update the rules in figure.}
Informally, $AbsQ$ records the set of enqueue operations, whose argument has not yet been removed by a matching dequeue operation. In addition, it records the happens-before order between those enqueue operations: this is a partial order ordering an enqueue $k_1$ before another enqueue $k_2$ iff $k_1$ returned before $k_2$ was invoked.
%Informally, $AbsQ$ records the happens-before order between enqueue operations for which the added value has not been removed by a dequeue operation.
The linearization point of a dequeue can either remove a minimal enqueue $k$ (w.r.t. the happens-before stored in the state) and fix the return value to the value $d$ added by $k$, or fix the return value to ${\tt EMPTY}$ provided that the current state stores only pending enqueues (intuitively, the dequeue overlaps with all the enqueue operations stored in the current state and it can be linearized before all of them).
%with return value $d\neq{\tt EMPTY}$ is enabled only if the happens-before stored in the current state contains a minimal enqueue that adds the value $d$. The effect of the linearization point is that the minimal enqueue is removed from the current state and the return value is recorded in the library state.
%When
%%\noindent
%the return value is {\tt EMPTY}, the linearization point of a dequeue is enabled only if the current state stores only pending enqueues (because, the dequeue overlaps with all the enqueue operations stored in the current state and it can be linearized before all of them).
%The return of a dequeue is enabled only if the returned value matches the one fixed at the linearization point.

\begin{wrapfigure}{l}{7.2cm}
\vspace{-6mm}
\includegraphics[width=7.3cm]{fig-queue12.pdf}
%
%\vspace{2mm}
%\includegraphics[width=7cm]{fig-queue2.pdf}
\vspace{-8mm}
\caption{Simulating queue histories with $AbsQ$. An operation is pictured by a line delimited by two circles denoting the call and respectively, the return action. The representation of a dequeue operation includes a red circle that stands for a {\tt swap} returning a non-null value, which is their linearization point.}
\label{fig:queueSim}
\vspace{-6mm}
\end{wrapfigure}
Fig.~\ref{fig:queueSim} pictures two executions of $AbsQ$ for two extended histories (that include dequeue linearization points). The state of $AbsQ$ after each action is pictured as a graph below the action. The nodes of this graph represent enqueue operations and the edges happens-before constraints. Each node is labeled by a value (the argument of the enqueue) and a flag {\tt PEND} or {\tt COMP} showing whether the operation is pending or completed. For instance, in the case of the first history, the dequeue linearization point $lin(deq,y,3)$ is enabled because the current happens-before contains a \emph{minimal} enqueue operation with argument $y$. Note that a linearization point $lin(deq,x,3)$ is also enabled at this state.

For readability, we define $AbsQ$ as an abstract state machine, which is given in Fig.~\ref{fig:transitions:AbsQ}. The shared state of $AbsQ$ consists of several boolean predicates indicating whether the value added by an enqueue has not been removed yet (the predicate $\mathsf{present})$, whether an enqueue is pending (the predicate $\mathsf{pending}$), the happens-before order (the predicate $\mathsf{before}$), and a function giving the value added by an enqueue (the function $\mathsf{arg}$). In the initial state, the domain of every function (predicate) is empty.
%the states of $AbsQ$ are tuples $\tup{O,<,\ell,rv,cp}$ where $O\subseteq \<Ops>$ is a set of operation identifiers, $<\subseteq O\times O$ is a strict partial order, $\ell: O -> \<Vals>\times\{\tt{PEND,\tt{COMP}}\}$ labels every identifier with a value and a pending/completed flag (the flag is used to track the happens-before order), $rv:\<Ops> ~> \<Vals>$ records the return value of a dequeue fixed at its linearization point ($~>$ denotes a partial function), and $cp:\<Ops> ~> \{A_1,A_2,R_1,R_2,R_3\}$ records the control point of every enqueue ($A_1, A_2$) or dequeue operation ($R_1,R_2,R_3$).
%All the components are $\emptyset$ in the initial state, and the transition relation $->$ is defined in Fig.~\ref{fig:transitions:AbsQ}. The alphabet of $AbsQ$ contains call/return actions and dequeue linearization points, denoted by $lin(deq,d,k)$. $Lin(deq)$ is the set of all actions $lin(deq,d,k)$.
%
The enqueue operations consist of two macro rules: the rule {\tt inv(enq,x,k)} orders the invoked operation after all the completed enqueues present in the current state ({\tt k} is a fresh operation identifier associated to the current operation), and the rule {\tt ret(enq,k)} sets $\mathsf{pending}${\tt (k)} to {\tt false}. % provided that the operation is still present in the current state.
The dequeue operations consist of two rules {\tt inv(deq,k)} and {\tt ret(deq,y,k)} associated to the invocation and respectively, the return action, whose definition is obvious, and the rule {\tt lin(deq,y,k)} corresponding to its linearization point. When a non-empty set of pending enqueues are present, the rule {\tt lin(deq,y,k)} is non-deterministic. It can set the return value {\tt y} either to {\tt EMPTY}, or to a value {\tt d} $\neq$ {\tt EMPTY}, provided that {\tt d} has been added by an enqueue {\tt k1} which is minimal in the current happens-before. In the latter case, $\mathsf{present}${\tt (k1)} is set to {\tt false} to indicate that {\tt k1}'s argument has been removed. When there exists at least one completed enqueue whose value has not been removed, the rule {\tt lin(deq,y,k)} sets {\tt y} to the value added by a minimal enqueue (w.r.t. the current happens-before).

%{\sc call-enq} orders the invoked operation after all the completed enqueues in the current state, and the rules {\sc ret-enq1}/{\sc ret-enq2} flip the corresponding flag from {\tt PEND} to {\tt COMP} provided that the operation is still present in the current state. For dequeue operations, {\sc call-deq} only increments the control point and {\sc ret-deq} checks whether the return value is the same as the one fixed at the linearization point. The linearization point rule {\sc lin-deq1} corresponds to the case of a non-empty queue, showing that $lin(deq,d,k)$ is enabled only if $d$ has been added by an enqueue which is minimal in the current happens-before. When enabled, it removes the enqueue adding $d$ from the state. The linearization point rule {\sc lin-deq2} corresponds to the case of dequeue operations linearized with an {\tt EMPTY} return value.

%\begin{figure} [t]
\begin{figure}[t]
\hspace{-10mm}
\begin{minipage}[t]{7.5cm}
\begin{lstlisting}
void enq($\<Vals>$ v):
  atomic rule inv(enq,v)
  atomic rule lin(enq,v)
  atomic rule ret(enq,v)

\end{lstlisting}
\end{minipage}
\begin{minipage}[t]{7.5cm}
\begin{lstlisting}
$\<Vals>$ deq():
  let v with ...
  atomic rule inv(deq,v)
  atomic rule lin(deq,v)
  atomic rule ret(deq,v)
  return v

\end{lstlisting}
\end{minipage}

\begin{minipage}[t]{7.5cm}
\vspace{-3mm}
\begin{lstlisting}
function $\mathsf{present}$: $\mathbb{O} \to \mathbb{B}$
function $\mathsf{pending}$: $\mathbb{O} \to \mathbb{B}$
function $\mathsf{arg}$: $\mathbb{O} \to \mathbb{V}$
function $\mathsf{before}$: $\mathbb{O} \times \mathbb{O} \to \mathbb{B}$

rule inv(enq,v):
  let k = new $\mathbb{O}$
  $\mathsf{present}$(k) := true
  $\mathsf{pending}$(k) := true
  $\mathsf{arg}$(k) := v
  forall k' with $\mathsf{present}$(k')$\land \neg \mathsf{pending}$(k'):
    $\mathsf{before}$(k',k) := true

rule lin(enq,v):
  pass

rule ret(enq,v):
  let k with $\mathsf{arg}$(k) = v
  $\mathsf{pending}$(k) := false

\end{lstlisting}
\end{minipage}
\begin{minipage}[t]{5cm}
\vspace{-3mm}
\begin{lstlisting}
rule inv(deq,v):
  pass

rule lin(deq,v):
  if v = EMPTY:
    forall k with $\mathsf{present}$(k):
      assert $\mathsf{pending}$(k)
  else:
    assert $\mathsf{present}$(k)
    let k with $\mathsf{arg}$(k) = v
    forall k' with $\mathsf{present}$(k'):
      assert $\lnot\mathsf{before}$(k', k)
    $\mathsf{present}$(k) := false

rule ret(deq,v):
  pass

\end{lstlisting}
\end{minipage}
%	\caption{The abstract queue implementation $AbsQ$.}
%	\label{fig:signatures}
%\end{figure}


%{\scriptsize
%  \centering
%  \begin{mathpar}
%    \inferrule[call-enq]{
%      k\not\in dom(cp) \\
%      d\neq {\tt EMPTY}
%    }{
%      O,<,\ell,rv,cp
%      \xrightarrow{inv(enq,d,k)}
%      %O\cup\{k\},<\cup \{(k',k): \ell_2(k')={\tt COMP}\},\ell[k\mapsto (d,{\tt PEND})],rv,cp[k\mapsto 1]
%      O\cup\{k\},<\cup\ {\tt COMP}(O)\times\{k\},\ell[k\mapsto (d,{\tt PEND})],rv,cp[k\mapsto A_1]
%    }\hspace{5mm}
%
%    \inferrule[call-deq]{
%      k\not\in dom(cp) \\
%    }{
%      O,<,\ell,rv,cp
%      \xrightarrow{inv(deq,k)}
%      O,<,\ell,rv,cp[k\mapsto R_1]
%    }\hspace{5mm}
%    \inferrule[ret-deq]{
%       cp(k) = R_2 \\
%       rv(k)=d
%    }{
%      O,<,\ell,rv,cp
%      \xrightarrow{ret(deq,d,k)}
%      O,<,\ell,rv,cp[k\mapsto R_3]
%    }\hspace{5mm}
%
%    \inferrule[ret-enq1]{
%      cp(k) = A_1 \\
%      k \in O \\
%      \ell(k) = (d,{\tt PEND})
%    }{
%      O,<,\ell,rv,cp
%      \xrightarrow{ret(enq,k)}
%      O,<,\ell[k\mapsto (d,{\tt COMP})],rv,cp[k\mapsto A_2]
%    }\hspace{5mm}
%    \inferrule[ret-enq2]{
%      cp(k) = A_1 \\
%      k \not\in O
%    }{
%      O,<,\ell,rv,cp
%      \xrightarrow{ret(enq,k)}
%      O,<,\ell,rv,cp[k\mapsto A_2]
%    }\hspace{5mm}
%
%    \inferrule[lin-deq1]{
%       cp(k) = R_1 \\
%       d\neq{\tt EMPTY} \\
%       k'\in min(O) \\
%       \ell_1(k')=d
%    }{
%      O,<,\ell,rv,cp
%      \xrightarrow{lin(deq,d,k)}
%      O\setminus \{k'\},<\uparrow k',\ell,rv[k\mapsto d],cp[k\mapsto R_2]
%    }\hspace{5mm}
%    \inferrule[lin-deq2]{
%       cp(k) = R_1 \\
%       \forall o\in O.\ \ell_2(o)={\tt PEND}
%    }{
%      O,<,\ell,rv,cp
%      \xrightarrow{lin(deq,{\tt EMPTY},k)}
%      O,<,\ell,rv[k\mapsto {\tt EMPTY}],cp[k\mapsto R_2]
%    }\hspace{5mm}
%      \end{mathpar}
%  }
 \vspace{-4mm}
  \caption{The abstract queue implementations $AbsQ$. Each macro rule is executed atomically in one step (indicated by the keyword {\ttfamily \bfseries atomic}).
%  transition relation of $AbsQ$. We use the following notations: $\ell_i(k)$ denotes the projection of $\ell(k)$ over the $i$-th component, for each $i\in\{1,2\}$, ${\tt COMP}(O)=\{k\in O: \ell_2(k)={\tt COMP}\}$, $\mathit{f}[x\mapsto y]$ is the function $g$ such that $g(z)=f(z)$ for all $z\neq x$ in the domain of $f$, and $g(x)=y$, $min(O)$ is the set of elements of $O$ which are minimal in the order relation $<$, and $<\uparrow k$ denotes the relation $<$ where all the pairs containing $k$ have been removed.
  %\textcolor{red}{Call-Enq must have $d!= \texttt{EMPTY}$ as a premise. Also lin deq returning empty must be changed as before.}
  }
  \label{fig:transitions:AbsQ}
\vspace{-6mm}
\end{figure}

% Let $AbsQ_0$ denote this implementation (formally defined in Appendix~\ref{app:absImplQueue}).
The following result states that the library $AbsQ$ has exactly the same set of histories as the standard abstract library $AbsQ_0$. % (see Appendix~\ref{app:absImplQueue} for a proof).

\vspace{-1.5mm}
\begin{theorem}\label{th:absImplQueue}
$AbsQ$ is a refinement of $AbsQ_0$ and vice-versa.
\vspace{-2mm}
\end{theorem}

The abstract state machine in Fig.~\ref{fig:transitions:AbsQ} defines an LTS over the alphabet $C\cup R\cup Lin(deq)$. The transitions corresponding to the rule {\tt lin(deq,y,k)} are labeled by  actions $lin(deq,y,k)$. Also, we assume that the transition corresponding to the invocation of a method is done atomically with the first step of the same invocation (the macro rule {\tt inv(...)}), and similarly for transitions corresponding to returning from a method, they are done atomically with the last step (the macro rule {\tt ret(...)}). They are labeled as expected with call/return actions (borrowing the operation identifier which occurs as argument of the macro rules). It can be easily proved that this assumption doesn't modify the set of traces projected on $C\cup R\cup Lin(deq)$ (compared to an LTS where these steps are not atomic).

%~\footnote{Without this assumption, while proving forward simulations, a call/return action in the concrete implementation is simulated by two steps of the abstract implementation, the call/return action together with the first/last macro rule.}.


A trace of a queue implementation is called \emph{$Lin(deq)$-complete} when every completed dequeue has a linearization point, i.e., each return action $ret(deq,d,k)$ is preceded by an action $lin(deq,d,k)$. A queue implementation $L$ over alphabet $\Sigma$, such that $C\cup R\cup Lin(deq)\subseteq \Sigma$, is called \emph{with fixed dequeue linearization points} when every trace $@t\in Tr(L)$ is $Lin(deq)$-complete.

%TODO NEEDS DATA INDEPENDENCE FOR THE LINEARIZATION POINT TRANSITIONS TO BE DETERMINISTIC

The following result shows that $C\cup R\cup Lin(deq)$-forward simulations are a sound and complete proof method for showing the correctness of a queue implementation with fixed dequeue linearization points (up to the correctness of the linearization points). It is obtained from Theorem~\ref{th:absImplQueue} and Theorem~\ref{th:forSim} using the fact that the alphabet of $AbsQ$ is exactly $C\cup R\cup Lin(deq)$ and $AbsQ$ is deterministic. The determinism of $AbsQ$ relies on the assumption that every value is added at most once. Without this assumption, $AbsQ$ may reach a state with two enqueues adding the same value being both minimal in the happens-before. A transition corresponding to the linearization point of a dequeue from this state can remove any of these two enqueues leading to two different states. Therefore, $AbsQ$ becomes non-deterministic. Note that this is independent of the fact that $AbsQ$ manipulates  operation identifiers.

\vspace{-1.5mm}
\begin{corollary}
A queue implementation $L$ with fixed dequeue linearization points is a $C\cup R\cup Lin(deq)$-refinement of $AbsQ_0$ if{f} there exists a $C\cup R\cup Lin(deq)$-forward simulation from $L$ to $AbsQ$.
\vspace{-1.5mm}
\end{corollary}

\vspace{-4.5mm}
\subsection{A Correctness Proof For Herlihy\&Wing Queue}\label{ssec:HerlihyWing}
\vspace{-1mm}
We describe a forward simulation $F_1$ from $\mathit{HWQ}$ to $AbsQ$. The description of $\mathit{HWQ}$ in Fig.~\ref{fig:HerlihyWing} defines an LTS whose states contain the shared array ${\tt items}$ and the shared counter ${\tt back}$ together with a valuation for the local variables ${\tt i}$, ${\tt x}$, and ${\tt range}$, and the control location of each operation. A transition is either a call or a return action, or a statement in one of the two methods ${\tt enq}$ or ${\tt deq}$.

An enqueue operation in a $\mathit{HWQ}$ state is pending and present if and only if its argument is stored in the array ${\tt items}$ or it has not yet written to the array  ${\tt items}$. Those operations should be $\mathsf{pending}$ and $\mathsf{present}$ in the related $AbsQ$ states. In addition to this, the forward simulation $F_1$ imposes following restrictions:
%A $\mathit{HWQ}$ state is related by $F_1$ to an $AbsQ$ state where $\mathsf{present}$ is true for all the enqueues whose argument is  stored in the array ${\tt items}$, and all the pending enqueues that have not yet written to the array ${\tt items}$ (and only for these enqueues). The order relation $\mathsf{before}$ between these enqueues is defined as follows:
\vspace{-2mm}
\begin{itemize}
	\item[(a)] pending and present enqueues are maximal, i.e., for every two $\mathsf{present}$ enqueues $k$ and $k'$ such that $k'$ is $\mathsf{pending}$, we have that $\neg \mathsf{before}(k',k)$, % $k'\not< k$,
	\item[(b)] $\mathsf{before}$ is consistent with the order in which positions of ${\tt items}$ have been reserved, i.e., for every two present enqueues $k$ and $k'$ such that ${\tt i}(k) < {\tt i}(k')$, we have that $\neg \mathsf{before}(k',k)$, %$k' \not< k$,
	\item[(c)] an enqueue which has reserved a position $i$ %and executed only the first statement
	can't be ordered before another enqueue that has reserved a position $j \geq i$ when the position $i$ has been ``observed'' by a non-linearized dequeue that may ``observe'' $j$ in the current array traversal, i.e., for every two present enqueues $k$ and $k'$, and a dequeue $k_d$, such that

	\vspace{-2mm}
	\noindent
	{\small
	\begin{align}
	\hspace{-8mm}
	{\tt canRemove}(k_d,k) \land (i(k') < i(k_d) \lor (i(k')=i(k_d) \land {\tt afterSwapNull}(k_d)))
\label{eq:inst}
	\end{align}}

	\vspace{-6mm}
	\noindent
	we have that $\neg \mathsf{before}(k,k')$. The predicate ${\tt canRemove}(k_d,k)$ holds when $k_d$ has currently visited a {\tt null} item in {\tt items} and the present enqueue index $i(k)$ is in the range of $(k_d)$ i.e., $\mathsf{present}(k) \land (x(k_d) = {\tt null} \land i(k) < {\tt range}(k_d) \land i(k_d) < i(k)) \lor (i(k_d) = i(k) \land {\tt beforeSwap}(k_d) \land {\tt items}[i(k)] != {\tt null})$. The predicates ${\tt afterSwapNull}(k_d)$ (resp., ${\tt beforeSwap}(k_d)$) holds when the dequeue $k_d$ is at the control point after a ${\tt swap}$ returning ${\tt null}$ (resp., before a ${\tt swap}$).
\vspace{-2mm}
\end{itemize}

\noindent
We have that $\mathsf{pending}(k)$ is ${\tt true}$ whenever $k$ is a pending enqueue, and $\mathsf{arg}(k)=d$ whenever the argument of the enqueue $k$ is $d$.
%An enqueue is labeled by $(d,{\tt PEND})$ where $d$ is the input value if it's pending and by  $(d,{\tt COMP})$, otherwise.
Also, for every dequeue operation $k$ such that ${\tt x}(k)=d\neq {\tt null}$, we have that ${\tt y}(k)=d$ (recall that ${\tt y}$ is a local variable of the dequeue method in $AbsQ$).

The restrictions of $F_1$ aim to ensure two important points:  (i) identifier of a pending and present enqueue method in the $\mathit{HWQ}$ state should be maximal according to $\mathsf{before}$ predicate in the related $AbsQ$ state and (ii) identifiera of a present enqueue method in $\mathit{HWQ}$ state of which data value is about to be removed by a dequeue operation should be minimal in the related $AbsQ$ state.

The first point is ensured by the restriction (a). Present and pending enqueue identifiers in the $\mathit{HWQ}$ state are $\mathsf{pending}$ in the related $AbsQ$ state and $\mathsf{pending}$ enqueues are maximal in a valid $AbsQ$ state.

Restrictions (b) and (c) ensure the point (ii). Consider a present enqueue $k$ that inserted its argument to ${\tt items}$ and  there exists a pending dequeue $k_d$ such that ${\tt canRemove}(k_d, k)$ and $k_d$ is just before its swap action at the reserved position of $k$ i.e., $i(k_d) = i(k)$. A pending enqueue operation cannot be ordered $\mathsf{before}$ $k$ since pending enqueues are maximal by (a). Regarding the completed and present enqueues $k'$, we consider two cases: $i(k') > i(k)$ and $i(k') < i(k)$. For the former case, the restriction (b) ensures $\neg \mathsf{before}(k',k)$ and for the latter case the restriction (c) ensures $\neg \mathsf{before}(k',k)$. Consequently, $k$ is a minimal element in $\mathsf{before}$ relation just before $k_d$ removes its data value.

Next, we show that $F_1$ is indeed a $C\cup R\cup Lin(deq)$-forward simulation. Let $s$ and $t$ be states of $\mathit{HWQ}$ and $AbsQ$, respectively, such that $(s,t)\in F_1$.
We omit discussing the trivial case of transitions labeled by call and return actions which are simulated by similar transitions of $AbsQ$ (for the return a dequeue operation $k$, we use the equality between the local variable ${\tt x}(k)$ in $s$ and the component $rv(k)$ in $t$).
%\textcolor{red}{ I think it is good to mention again that call/return actions in HWQ correspond to the same call/return actions in AbsQ (without any other internal action). I also think that invoke enqueue operation is non-trivial. Preservation of the strict partial order and all of the above items a, b and c needs to be rechecked.}

%\setlength{\parskip}{0pt}
We show that each internal step of an enqueue or dequeue, except the execution of {\tt swap} returning a non-null value in dequeue (which represents its linearization point), is simulated by an \emph{empty} sequence of $AbsQ$ transitions, i.e., for every state $s'$ obtained through one of these steps, if $(s,t)\in F_1$, then $(s',t)\in F_1$ for each $AbsQ$ state $t$.
Essentially, this consists in proving the following property, called \emph{monotonicity}: the set of possible $\mathsf{before}$ relations associated by $F_1$ to $s'$ doesn't exclude any order $\mathsf{before}$ associated to $s$.
%Essentially, this boils down to showing that the constraints over $<$ in the definition of $\mathit{fs}$ are an invariant for these steps.

Concerning enqueue rules, let $s'$ be the state obtained from $s$ when a pending enqueue $k$ reserves an array position. This enqueue must be maximal in both $t$ and any state $t'$ related to $s'$ (since it's pending). Moreover, there is no dequeue that can ``observe'' this position before restarting the array traversal. Therefore, item (c) in the definition of $F_1$ doesn't constrain the order between $k$ and some other enqueue neither in $s$ nor in $s'$. Since this transition doesn't affect the constraints on the order between enqueues different from $k$ (their local variables remain unchanged), monotonicity holds. This property is trivially satisfied by the second step of enqueue which doesn't affect {\tt i}.

To prove monotonicity in the case of dequeue internal steps different from its linearization point, it is important to track the non-trivial instantiations of item (c) in the definition of $\mathsf{before}$ over the two states $s$ and $s'$, i.e., the triples $(k,k',k_d)$ for which (\ref{eq:inst}) holds. Instantiations that are enabled only in $s'$ may in principle lead to a violation of monotonicity (since they restrict the orders $\mathsf{before}$ associated to $s'$). For the two steps that begin an array traversal, i.e., reading the index of the last used position and setting {\tt i} to $0$, there exist no  such new instantiations in $s'$ because the value of {\tt i} is either not set or $0$. % (it is trivial to notice that applying these steps doesn't disable such instantiations that were possible in $s$).
%The same holds for the step incrementing the iterator {\tt i}.
%
%The execution of {\tt swap} returning {\tt null} may introduce one new non-trivial instantiation $(k,k',k_d)$ of item (c).
%We write ${\tt i}_s(k)$ to refer to the value of the variable {\tt i} of operation $k$ in state $s$. Assume that indeed, there exist two enqueue operations $k$ and $k'$ such that ${\tt i}_{s'}(k) < {\tt i}_{s'}(k_d) \leq {\tt i}_{s'}(k')$, ${\tt x}_{s'}(k_d)={\tt null}$, ${\tt i}_{s'}(k') \leq {\tt range}_{s'}(k_d)$ TODO SWAP. Since {\tt swap} returnes {\tt null}, the position ${\tt i}_{s'}(k_d)$
%
%
% and ${\tt i}_{s}(k) = {\tt i}_{s'}(k_d)$. The latter constraint guarantees that this instantiation is not enabled in state $s$. The increment of {\tt i} being enabled, implies that
%
The same is true for the increment of {\tt i} in a dequeue $k_d$ since the predicate ${\tt afterSwapNull}(k_d)$ holds in state $s$.
The execution of {\tt swap} returning {\tt null} in a dequeue $k_d$ enables new instantiations $(k,k',k_d)$ in $s'$, thus adding potentially new constraints $\neg \mathsf{before}(k,k')$. We show that these instantiations are however vacuous because $k$ must be pending in $s$ and thus maximal in every order $\mathsf{before}$ associated by $F_1$ to $s$.
Let $k$ and $k'$ be two enqueues such that together with the dequeue $k_d$ they satisfy the property (\ref{eq:inst}) in $s'$ but not in $s$.
We write ${\tt i}_s(k)$ for the value of the variable {\tt i} of operation $k$ in state $s$.
We have that ${\tt i}_{s'}(k) = {\tt i}_{s'}(k_d) \leq {\tt i}_{s'}(k')$ and ${\tt items}[{\tt i}_{s'}(k_d)]={\tt null}$. The latter implies that the enqueue $k$ didn't execute
the second statement (since the position it reserved is still {\tt null}) and it is pending in $s'$. The step that swaps the null item does not modify anything except the control point of $k_d$ that makes ${\tt afterSwapNull}(k_d)$ true in $s'$ . Hence, $i_s(k) = i_s(k_d)\leq i_s(k')$ and ${\tt items}[i_s(k_d)] = {\tt null}$ is also true. Therefore, $k$ is pending in $s$ and maximal. Hence, $\mathsf{before}(k,k')$ is not true in both $s$ and $s'$.

Finally, we show that the linearization point of a dequeue $k$ of $\mathit{HWQ}$, i.e., an execution of {\tt swap} returning a non-null value $d$, from state $s$ and leading to a state $s'$ is simulated by a transition labeled by $lin(deq,d,k)$ of $AbsQ$ from state $t$. By the definition of $\mathit{HWQ}$, there is a unique enqueue $k_e$ which filled the position updated by $k$, i.e., ${\tt i}_s(k_e)=i_s(k)$ and ${\tt x}_{s'}(k)={\tt x}_s(k_e)$.

We show that $k_e$ is minimal in the order $\mathsf{before}$ of $t$ which implies that $k_e$ could be chosen by $lin(deq,d,k)$ step applied on $t$. As explained previously, instantiating item (c) in the definition of $\mathsf{before}$ with $k'=k_e$ and $k_d=k$; and instantiating item (b) with $k=k_e$, we ensures the minimality of $k_e$. Moreover, the state $t'$ obtained from $t$ through a $lin(deq,d,k)$ transition is related to $s'$ because the value added by $k_e$ is not anymore present in {\tt items} and $\mathsf{present}(k_e)$ doesn't hold in $t'$.
%Thus, instantiating item (c) in the definition of $\mathsf{before}$ with $k'=k_e$ and $k_d=k$ we get that every enqueue that reserved a position smaller than the one of $k_e$ can't be ordered before $k_e$ in the order $\mathsf{before}$. Also, applying item (b) with $k=k_e$ we get the same for every enqueue that reserved a bigger position. An enqueue that didn't reserve a position is by definition maximal in $<$ and therefore, not a predecessor of $k_e$. Then, the state $t'$ obtained from $t$ through a $lin(deq,d,k)$ transition is related to $s'$ because (1) the value added by $k_e$ is not anymore present in {\tt items} which implies that $k_e$ doesn't occur in any $AbsQ$ state related to $s'$, and (2) the value of ${\tt x}(k)$ is set to $d\neq {\tt null}$ which implies that $rv(k)$ is set to $d$ in every $AbsQ$ state related to $s'$.
%
%\vfill
%


\section{Existence of Forward Simulations for Stack Implementations that have Fixed Pop Linearization Points}
\section{Relaxation for the Data Structures Without Fixed Remove Linearization Points}
We have observed that some implementations (like time-stamped stack) do not have fixed remove (pop) linearization points that will correspond to $lin(pop,e,k)$ where $e$ could be a data value or $\texttt{EMPTY}$. However, we observe that, these implementations contain some points in their pop methods that logically removes the element from the pool. We call them commit points. If a method of a library has a fixed linearization  point, it is also a commit point. In this sense, commit points are weaker versions of fixed linearization points. Fixed linearization point of a pop preserves the following properties:
\begin{itemize}
\item If a $ret$ comes before a linearization point in the concrete execution, this order is preserved in the linearization of this execution.
\item If a $lin$ pop comes before another $lin$ pop in the concrete execution, this order is preserved in the linearization of this execution.
\item If a $lin$ pop comes before an $inv$ in the concrete execution, this order is preserved in the linearization of this execution. 
\end{itemize} 
A commit point is weaker than a fixed linearization point in the sense that it does not need to satisfy the first and the second conditions. 

To our intuition, if a pop (remove) method has multiple finite linearization points, commit point is the latest element in this set. We have never observed an implementation in which a pop is linearized after it logically removes the element from the data structure. 

From now on, we will restrict ourselves to stacks, since our example implementation that we will show linearizability of is the time-stamped (TS) stack. However, the notions and the machines we will introduce can be extended to queues easily. 

We fix $\mathcal{M} = \{push, pop\}$ and $\mathcal{D} = \mathbb{N} \cup \{\texttt{EMPTY}\}$. We extend the alphabet $A\Sigma$ for stacks with commit points as $ACS\Sigma = A\Sigma \cup \{com(pop,d,k)|d \in \mathcal{D}, k \in \mathcal{M}\}$. We define cs-refinement and cs-linearizability as we defined q-refinement, q-linearizability in the previous sections. We also change definitions of backward and forward simulation relations for stacks with commit points as we do in the previous sections by replacing linearization points with commit points in this extensions. Lemma 1 of the first section still holds with new simulation relation definitions and cs-linearizability and cs-linearizability implies the original linearizability definition. 

Our road map for this section is as follows: We will first introduce an intermediate stack machine $L_I$ that will be deterministic with respect to the alphabet $ACS\Sigma$. We will show that $L_I$ is equivalent to the standard abstract stack $L_A$ defined in the previous section with respect to the language $A\Sigma$. We show this by first showing that $L_I$ is a refinement of $L_A$ wrt alphabet $A\Sigma$ by finding a backward simulation relation between them and then $L_A$ is a refinement of $L_I$ with respect to alphabet $A\Sigma$ by finding a forward simulation relation between them. Since $L_I$ is deterministic wrt $ACS\Sigma$, if we have an implementation $L_C$ that is a cs-refinement of $L_I$, we can find a forward simulation between them. As an example, we will pick $L_C$ as time-stamped stack and establish a forward simulation relation between it and our $L_I$ machine. 

Let us continue with defining $L_I$ first:
\begin{itemize}
\item A state of $L_I$ again consists of a partial strict order and a program counter: $Q_I \subseteq ND \times ED \times (\mathbb{N} \rightarrow Lbl_I)$ where $Lbl_I = \{N, A_0, A_1, R_0, R_1, R_2\}$ is the set of transition labels for the operations. Different from the fixed linearization point case, this time nodes are not triples but 5-tuples of the form $(k,d,st,mc,con)$. First three fields are the same as previous intermediate machines: $k \in \mathbb{N}$ is operation identifier of a push operation, $d \in \mathcal{D}$ is the data value of that push and $st \in \{\texttt{PENDING}, \texttt{CLOSED}\}$ is the current status of the push operation. The fourth field $mc \subset \mathbb{N}$ keeps the operation identifiers of pop operations such that this push was maximally closed when the pop began. A node $n$ is maximally closed in a state $s$ iff $n.st  = \texttt{CLOSED}$ and if there is an edge $n \rightarrow n' \in ed_s$, then $n'.st = \texttt{OPEN}$. The fifth field $con \subset \mathbb{N}$ is the set of operation identifiers of the pop methods that are concurrent with this push i.e. the either this node was open when the pop started or this node is created when the pop was pending.
\item The transition labels consist of invocation and return actions of both methods and commit action for only pop method. Hence $\Sigma_I = ACS\Sigma$. Number of parameters for all actions common in previous intermediate stack machine and this intermediate stack machine are the same. Commit actions contain the second data field. They are of the form: $com(pop,d,k)$ where $d \in \mathcal{D}$ and $k \in \mathbb{N}$. 
\item Initial state consists of an empty strict partial order and a function mapping ever operation to $N$: ${q_0}_I = (\emptyset, \emptyset, f_{{q_0}_I})$ where  $f_{{q_0}_I}(i) =N$ for all $i \in \mathbb{N}$.
\item We define $\delta_I$ less formally, by giving verbal explanations to the transitions, omitting the obvious updates on the $f$ part and not mentioning about the parts of the nodes that does not change:
\begin{itemize}
\item $(q, inv(push,d,k),q') \in \delta_I$ iff a new node $n=(k,d,\texttt{PENDING}, \emptyset, con_n)$ is added to $nd_{q'}$, where $con_n = \{ i \in \mathbb{N}| f_q(i) = R_0\}$; $n' \rightarrow n$ will be added to $ed_{q'}$ if $n'$ is a closed node at the state $q$.
\item $(q, ret(push,k),q') \in \delta_I$ iff either there is a \texttt{PENDING} node $n$ in state $q$ and this node becomes \texttt{CLOSED} in state $q'$ or there is no node with identifier $k$ in $q$ and nothing else than $f$ field changes in $q'$.
\item $(q, inv(pop,k),q') \in \delta_I$ iff for every open node $n \in nd_q$, $k$ is added to $n.con$ in $q'$ and for every maximally closed node $m \in nd_q$, $k$ is added to $m.mc$ in state $q'$. 
\item $(q, com(pop,k,d), q') \in \delta_I$ iff there exists a node $n$ in state $q$ such that $n.d = d$ and either $k \in n.con$ or $k \in n.mc$, this node $n$ and all the nodes adjacent to it are removed in the state $q'$, $k$ is removed from all $con$ and $mc$ fields of all nodes in state $q'$ and for all other pop operations $k'$ in $n.mc$ or in $n.con$ and for all states $n' \in nd_q$ such that $n' \rightarrow n \in ed_q$ and for all $n'' \in nd_q$, $n' \rightarrow n'' \in ed_q$ implies $k' \in n''.con$ (implies $k \notin n''.mc$ but it is stronger than this condition: if there are three closed states $p,q,r$ s.t. $p \rightarrow q$, $q \rightarrow r$, $p \rightarrow r$, $k'$ is only in $r.mc$ and we delete $r$, former condition only allows $k' \in q.mc$ whereas the latter one allows $k' \in p.mc$ in addition) we have $k' \in n'.mc$ in the state $q'$. Note that we need to assume data independence to make this action deterministic.
\item $(q, ret(pop,k,d), q') \in \delta_I$ iff $q=q'$ ignoring the $f$ fields.
\end{itemize}
\end{itemize}
$L_A$ is the same machine defined in the previous section. The common alphabet between $L_A$ and $L_I$ is $A\Sigma$. We will show that they are equivalent in terms of this alphabet.

\begin{lem}
$L_I$ is a refinement of $L_A$.
\end{lem}
\begin{proof}
We will provide a backward simulation relation $bs$ between states of $L_I$ and states of $L_A$. Our relation $bs \subseteq Q_I \times Q_A$ relates state $q=(so_q, f_q)$ to $q' =(s_{q'}, f_{q'}$ iff (i) for all operation identifiers $k \in \mathbb{N}$, if $f_q(k) \in \{N,R_1,R_2\}$, then $f_q(k) = f_{q'}(k)$; if $f_q(k) = A_0$, then $f_{q'}(k) = A_0$ and the data value $d$ associated with this add operation is not inserted to $s_{q'}$ or $f_{q'}(k) = A_1$;  if $f_q(k) = A_1$, then $f_{q'}(k) = A_2$; if $f_q(k) = R_0$, then $f_{q'}(k)=R_0$ or $f_{q'}(k)=R_1$; (ii) Let us call a pop operation pending if $f_{q}(k) = R_0$ and $PP_q$ be set of pending pops. There there exists a function $g: PP_q \rightarrow ND_q \cup \{\texttt{NONE}\}$ such that $k \in g(k).mc$ or $k \in g(k).con$ for all pop operation identifiers $k$ such that $g(k) \neq \texttt{NONE}$ and $g$ is one-to-one if we neglect \texttt{NONE}; $s_{q'}$ is obtained by extending $so_q$ to a total order in which pending nodes may not take place and nodes (open or closed) $n$ such that there exists a pop operation with identifier $k$ so that $g(k)=n$ surely do not take place, (iii) $g(k) = \texttt{NONE}$ implies $f_{q'}(k) = R_0$ and $g(k) \neq \texttt{NONE}$ implies $f_{q'}(k) =R_1$, (iv) if $g_{q'}(k) =n$ and $n$ is a pending node, then $f_{q'}(k') = A_1$ where $k'$ is the operation identifier part of $n$, (v) if there is a pending node with identifier $k$ and it takes part in the linearization, then $f_{q'}(k) = A_1$. 

Now, we will show that $bs$ is a backward simulation relation:
\begin{itemize}
\item[$\langle i \rangle$] $bs[{q_0}_I] =\{{q_0}_A\}$
\item[$\langle ii-a-push \rangle$] Let $(q,inv(push,d,k),q') \in \delta_I$ and $t' \in bs[q']$. We consider two cases: Either the newly added node in $q'$ takes place in the linearization and there exists an index $i$ such that $s_{t'}(i) = d$ or this new node does not exist in the linearization. For the former case construct $s_t = \langle s_{t'}(1), s_{t'}(2),..., s_{t'}(i-1) \rangle$ and $f_t = f_{t'}$ for all the operations except the ones of which data values are linearized as $s_{t'}(j)$,  for $j\geq i$. For those nodes, $f_{t'}(k') = A_1$ whereas we assign $f_t(k') = A_0$ for $j>i$ and $f_t(k) = N$. Let operation identifiers of these nodes be $k_j$ for $j>i$. Then, $t \xrightarrow{\alpha} t'$ holds where $\alpha = inv(push,d,k), lin(push,d,k), lin(push,d_{i+1},k_{i+1}),....$. Moreover, $t \in bs[q]$ since $s_t$ is a valid linearization of $so_q$ using the same $g$ function and omitting more open nodes and $f_t$ obeys the conditions. For the latter case, we pick $s_t = s_{t'}$. We have two subcases:  There exists a pending pop with identifier $k'$ such that $g(k')$ is the new node with identifier $k$ or not. For the first subcase, we pick $f_t = f_{t'}$ except $f_t(k) = N$ and $f_t(k') = R_0$. Then $t xrightarrow{inv(push,d,k),lin(pop,d,k')} t'$ is a path in $L_A$ and $t \in bs[q]$. For the second subcase, we just pick $f_t = f_{t'}$ except $f_t(k) = N$. Then, it is easy to see that $t \xrightarrow{inv(push,d,k)} t'$ holds and $t \in bs[q]$. 
\item[$\langle ii-a-pop \rangle$] Let $(q, inv(pop,k),q') \in \delta_A$ and $t' \in bs[q']$. We will again consider two cases: When relating $t'$ to $q'$, either $g(k) = \texttt{NONE}$ or $g(k)$ is a node in $q'$. In other words, either the newly invoked pop operation $k$ did not linearize yet or it linearizes and removes an element inserted by a linearized push. The second case also splits into two cases: The element removed by pop $k$ is inserted by a push $k'$ that is still pending or the push has returned. We will look at all three cases separately. The easiest one is the first case. Construct $s_t=s_{t'}$ and $f_t = f_{t'}$ except that $f_t(k) = N$ whereas $f_{t'}(k)= R_0$. One can see that $t \xrightarrow{inv(pop,k)} t'$ is a step in $L_A$ and $t \in bs[q]$. For the first case of the second case, we construct $s_t = s_{t'} \circ \langle d \rangle$ where $d$ is the data of node identifier $k'$ and $f_t = f_{t'}$ except that $f_t(k) = N$ whereas $f_{t'}(k) = R_1$. One can see that $t \xrightarrow{inv(pop,k), lin(pop,d,k)}$ is a valid path in $L_A$. Moreover, $t \in bs[q]$ since $s_t$ is a valid linearization of $so_q$. This is true because the node with identifier $k'$ is a maximal node in $so_q$ and we can add it to the end of linearization of $so_{q'}$. For the second subcase, we obtain $s_t$   from $s_{t'}$ by the following procedure: Let $n$ be the node with identifier $k'$ and $k''$ be the node such that $k' \rightarrow k''$ is an edge in $so_{q'}$, $k''$ takes part in the linearization of $so_{q'}$ to $s_{t'}$ and it has the minimum index $i$ in the $s_{t'}$ among all such nodes. Then, $s_t = \langle s_{t'}(1), s_{t'}(2),..., s_{t'}(i-1), d \rangle$ where $d$ is the data value of node with identifier $k$. Let $k_j$ and $d_j$ be the identifiers and data values of nodes that constitute $s_{q'}(j)$ for $j>i$. Clearly, $t \xrightarrow{\alpha} t'$ is a path in $L_A$ where $\alpha = inv(pop,k), lin(pop,d,k), lin(push, d_i, k_i),...$. Moreover, $t \in bs[q]$ because $s_t$ is a valid linearization of $so_q$. It is true because $so_q = so_{q'}$, $k'$ is a maximally closed node in $so_q$ and all the nodes with identifier $k_j$ ($j>i$) are pending nodes.
\item[$\langle ii-c \rangle$] Let $(q, com(pop,d,k), q') \in \delta_I$ and $t' \in bs[q']$. We will consider two cases. The first case is commit action removes a maximally closed or an open node. For these cases, we can construct $t$ as in the second case of the previous item (invoke pop case). The same arguments apply for constructing a path between $t$ and $t'$ in $L_A$ and showing that $t \in bs[q]$.  The second case is that commit action removes a non-maximal closed node. This time, pick $t = t'$. Then, $t \xrightarrow{\epsilon} t'$ is the path in $L_A$ and one can show that $t \in bs[q]$ by choosing $g(k)$ as the node that is removed. 
\item[$\langle ii-b-push \rangle$] Let $(q,ret(push,k),q') \in \delta_I$ and $t' \in bs[q']$. We consider two cases. Either there is a node $n$ with identifier $k$ and data value $d$ in state $q$ or not. For the first case, we consider two subcases. Either this node takes part in linearization or not (if there exists a pop $k'$ such that $g(k')=n$). For the first subcase, we can pick $s_t = s_{t'}$ and $f_q = f_{q'}$ except that $f_q(k) = A_0$.  Then, $t \xrightarrow{ret(push,k)} t'$ is a path in $L_A$. Moreover, $t \in bs[q]$ since $n$ is a maximal node in $q$. For the second subcase, we pick $s_t = s_{t'} \circ \langle d \rangle$ and $f_t = f_{t'}$ except that $f_t(k) = A_1$ and $f_t(k') = R_0$. Then, $t \xrightarrow{lin(pop,d,k'), ret(push,k)} t'$ is a path $L_A$ and $t \in bs[q]$ since $n$ is a maximal open node in $q$. For the second case, we pick $s_t = s_{t'}$ and $f_t = f_{t'}$ except that $f_t(k) = A_1$. Then, $t \xrightarrow{ret(push,k)} t'$ is a path in $L_A$ and $t \in bs[q]$. 
\item[$\langle ii-b-pop \rangle$] Let $(q,ret(pop,d,k),q') \in \delta_I$ and $t' \in bs[q']$. We pick $s_t = s_{t'}$ and $f_t = f_{t'}$ except that $f_t(k) =R_1$. One can see that $t \xrightarrow{ret(pop,d,k)} t'$ is a valid action in $L_A$ and $t \in bs[q]$.
\end{itemize}
\end{proof}

\begin{lem}
$L_A$ is a refinement of $L_I$.
\end{lem}
\begin{proof}
We will construct a forward simulation relation $fs$ between $L_A$ and $L_I$. Our relation $fs \subseteq Q_A \times Q_I$ relates state $q = (s_q, f_q) \in Q_A$ to a state $q' =(so_{q'}, f_{q'}) \in Q_I$ iff (i) for all operation identifiers $k \in \mathbb{N}$, if $f_q(k) \in \{N, A_0, R_0, R_1, R_2\}$ then $f_{q'}(k) = f_q(k)$; if $f_q(k) = A_1$, then $f_{q'}(k) = A_0$; if $f_q(k) = A_2$, then $f_{q'}(k)= A_1$; (ii) We form nodes of $so_{q'}$ ($ND_{q'}$ as follows:  If $k$ is a push operation adding data value $d$ and either $f_q(k)= R_0$ or $f_q(k)=R_1$ and data added by this push exists in $s_q$, then there is a \texttt{PENDING} node in $ND_{q'}$ with identifier $k$ and data value $d$. If $f_q(k)= R_2$, then there is a \texttt{CLOSED} node in $nd_q$ with identifier $k$ and data value $d$. If there is an operation identifier $k$ such that $f_q(k) = R_0$, we call this a pending pop and this pop takes place $mc$ or $con$ fields of nodes of $q'$. If $n \in ND_{q'}$ is a \texttt{PENDING} node, then $k \in n.con$. If $n$ is maximally closed, then $k \in n.con$ or $n.mc$. If $k \in n.mc$ or $k \in n.con$ and there exists another node $n' \in ND_{q'}$ such that $n \rightarrow n' \in ED_{q'}$, then $k \in n'.con$. If $k \in n.mc$ and there exists another node $n' \in ND_{q'}$ such that $n' \rightarrow n \in ED_{q'}$, then neither $k \in n'.mc$ nor $k \in n'.con$. For any node $n \in ND_{q'}$, either $k \in n.mc$ or $k \in n.con$. (iii) We form edges of $so_{q'}$ ($ED_{q'}$) as follows: \texttt{PENDING} nodes are maximal. Edges obey the strict partial order conditions. (iv) We can find a linearization $so_{q'}$ that is equal to $s_q$. \texttt{PENDING} nodes may not participate in the linearization. Note that we do not need $g$ function for keeping track of linearized pops unlike the previous proof.

Next, we will show that $fs$ is a forward simulation relation.
\begin{itemize}
\item[$\langle i \rangle$] $fs[{q_0}_A] = \{{q_0}_I\}
$ 
\item[$\langle ii-a-push \rangle$] Let $(q, inv(push,d,k), q') \in \delta_A$ and $t \in fs[q]$. Pick $t'$ such that $(t, inv(push,d,k), t') \in \delta_I$. Since $s_q=s_{q'}$ and $so_{t'}$ contains a maximal open new node as the only difference from $so_t$, we can linearize $so_{t'}$ so that linearization is equal to $s_{q'}$. By checking the other conditions, one can observe that $t' \in fs[q']$.
\item[$\langle ii-a-pop \rangle$] Let $(q, inv(pop,k), q') \in \delta_A$ and $t \in fs[q]$. Pick $t'$ such that $(t, inv(pop,k), t') \in \delta_I$. Only difference between $t$ and $t'$ is that maximally closed and open nodes in $t'$ contain $k$ in their $mc$ or $con$ fields. Since this new addition obeys our forward simulation definition, $t' \in fs[q']$.
\item[$\langle ii-c-push \rangle$] Let $(q, lin(push,d,k) q') \in \delta_A$ and $t \in fs[q]$. Pick $t'=t$. $s_{q'}$ is still a linearization of $so_t$ since the node with identifier $k$ is a maximal node in $ND_t$ and we can linearize it at the end. So, $t' \in fs[q']$ holds.
\item[$\langle ii-c-pop \rangle$] Let $(q, lin(pop,d,k), q') \in \delta_A$ and $t \in fs[q]$. Pick $t'$ such that $(t, com(pop,d,k) t') \in \delta_I$. The action $com(pop,d,k)$ is a valid action in $L_I$ because $d$ is the last element in $s_q$. Hence, the node $n$ with identifier $k$ and data value $d$ is a maximal element in $so_q$ and either $k \in n.mc$ or $k \in n.con$ by the properties of $fs$. Hence, the node with identifier $k$ can be removed by a $com$ action. In addition, $s_{q'}$ is a linearization of $so_{t'}$ because removed node is a maximal node and $s_{q'}$ is obtained from $s_q$ by deleting the maximum node. Hence, $t' \in fs[q']$ holds.
\item[$\langle ii-b-push \rangle$] Let $(q, ret(push,k), q') \in \delta_A$ and $t \in fs[q]$. We will consider two cases: either the element inserted by the push with identifier $k$ is removed by a concurrent pop or the element is still in $s_q$. For the former case, there is no node with identifier $k$  in state $t$ and we can pick $t'$ such that $(t, ret(push,k), t') \in \delta_I$. We have $so_t = so_{t'}$ for this case. Since $s_q = s_{q'}$ also holds, $s_{t'}$ is a linearization of $so_{q'}$ and $t' \in fs[q']$. For the latter case, we again pick $t'$ such that $(t, ret(push,k), t') \in \delta_I$ holds. This time, only difference between $so_t$ and $so_{t'}$ is that the node with identifier $k$ is \texttt{CLOSED} in $so_{t'}$. Since the edges are the same, $s_{q'}$ is a valid linearization of $s_{t'}$ and $t' \in fs[q']$ holds. 
\item[$langle ii-b-pop \rangle$] Let $(q, ret(pop,d,k), q') \in \delta_A$ and $t \in fs[q]$. We pick $t'$ such that $(t, ret(pop,d,k) t') \in \delta_I$. Only difference between $t$ and $t'$ is that $f_{t'}(k) = R_2$ whereas $f_t(k) = R_1$. Since $s_q = s_{q'}$ and $so_t = so_{t'}$, $s_{q'}$ is a valid linearization of $so_{t'}$. By checking the other conditions, we see that $t' \in fs[q']$.  
\end{itemize}
\end{proof}

Now, we show that TS-Stack is linearizable by showing the concrete TS-Stack implementation $L_C$ is a cs-refinement of $L_I$. As $L_C$, we pick the simplest version that omits the \texttt{EMPTY} returns of the pop  methods, does not allow unlinking and elimination.
First, describe the TS-Stack algorithm:

\begin{lstlisting}
Type Node{
 elt:Vals;
 ts: int;
 next: Node;
 taken: bool;
};


pools: [Tid]Node; //Keeps top of pools, 
//Initially they point to sentinel dummy nodes s s.t.
//s.next = s and s.ts = -1 and s.taken=false
TSGen: int;    //Timestamp generator initially 0

void push(e:Vals, tid:Tid){
  Node n = new Node(e, MAX_INT, NULL, false);
  n.next = pools[tid];
  pools[tid] = n;
  i: int = TSGen++;
  n.ts = i;
}

Vals pop(){
 success: bool = false;
 maxTS: int = -1;
 youngest: Node = NULL;
 while(!success){
  maxTS: int = -1;
  youngest = NULL;
  for(current in pools){
    n: Node = current;
    while(n.taken && n.next != n)
      n = n.next;
    if(maxTS < n.ts)
      youngest = n;
  }
  success = CAS(youngest.taken,false,true);
 }
 return youngest;
}

\end{lstlisting}
\section{Existence of Forward Simulations for Set Implementations That Have Fixed Remove Linearization Points }
For all the set libraries we fix $\mathcal{M} = \{ add, rmv, cnt\}$ where $rmv$ is short for remove and $cnt$ is short for contains methods and $\mathcal{D} = \{1, \texttt{TRUE}, \texttt{FALSE} \}$. We assume that only single element can be inserted into our list. Our results for this domain extends to other domains such as when $\mathcal{D} = \mathbb{N}$. We extend the definition of $A\Sigma$ introduced in Definition 2 for the set in our focus as $AS\Sigma = A\Sigma \cup \{lin(rmv,d,k)| d \in \{1\}, k \in \mathbb{N}\}$. We define s-refinement, s-linearizability and change the definitions of forward and backward simulations as in the beginning of Section 2.

We define $L_A$ as follows:
\begin{itemize}
\item $Q_A = 2^{\{1\}} \times (\mathbb{N} \rightarrow Lbl_A)$ where $Lbl_A = \{N, A_0, A_{1T}, A_{1F} A_2, R_0, R_{1T}, R_{1F}, R_2, C_0, C_{1T}, C_{1F}, C_2 \}$. For each $q \in Q_A$, $s_q$ represents the set component (first component), and $f_q$ represents the program counter (second component).
\item Transition labels consists of invocations, returns and linearizations of methods in $\mathcal{M}$: $\Sigma_A: AS\Sigma \cup \{lin(m,d,k)| m \in \{add,cnt\}, d\in \{1\}, k \in \mathbb{N} \}$. All methods return \texttt{TRUE} or \texttt{FALSE} and all take an element in $\{1\}$ as the input value.
\item ${q_0}_A = (\emptyset, f_{{q_0}_A}$ where $f_{{q_0}_A}(k) = N$ for all $k \in \mathbb{N}$.
\item State transitions:
\begin{itemize}
\item $(q, inv(add, d,k), q') \in \delta_A$ iff $d \in \{1\} \wedge f_q(k) = N \wedge f_{q'}(k) = A_0$
\item $(q, lin(add,d,k), q') \in \delta_A$ iff $d \in \{1\} \wedge f_q(k) = A_0 \wedge (d \in s_q \wedge s_q = s_{q'} \wedge f_{q'}(k) = A_{1F} \vee d \notin s_q \wedge s_{q'} = s_q \cup \{d\} \wedge f_{q'}(k) = A_{1T})$
\item $(q, ret(add,d,k) q') \in delta_A)$ iff $(f_q(k) = A_{1T} \wedge d = \texttt{TRUE} \vee f_q(k) = A_{1F} \wedge d = \texttt{FALSE}) \wedge f_{q'}(k) = A_2$
\item $(q, inv(rmv,d,k) q') \in \delta_A$ iff $d \in \{1\} \wedge f_q(k) = N \wedge f_{q'}(k)= R_0 $
\item $(q, lin(rmv,d,k), q') \in \delta_A$ iff $d \in \{1\} \wedge f_q(k) = R_0 \wedge (d \in s_q \wedge s_q = s_{q'} \cup \{d\} \wedge f_{q'}(k) = R_{1T} \vee d \notin s_q \wedge s_q = s_{q'} \wedge f_{q'}(k) = R_{1F} )$
\item $(q, ret(rmv,d,k) q') \in delta_A)$ iff $(f_q(k) = R_{1T} \wedge d = \texttt{TRUE} \vee f_q(k) = R_{1F} \wedge d = \texttt{FALSE}) \wedge f_{q'}(k) = R_2$
\item $(q, inv(cnt,d,k) q') \in \delta_A$ iff $d \in \{1\} \wedge f_q(k) = N \wedge f_{q'}(k)= C_0 $
\item $(q, lin(cnt,d,k), q') \in \delta_A$ iff $d \in \{1\} \wedge f_q(k) = C_0 \wedge s_q = s_{q'} \wedge (d \in s_q \wedge f_{q'}(k) = C_{1T} \vee d \notin s_q \wedge f_{q'}(k) = C_{1F} )$
\item $(q, ret(cnt,d,k) q') \in delta_A)$ iff $(f_q(k) = C_{1T} \wedge d = \texttt{TRUE} \vee f_q(k) = C_{1F} \wedge d = \texttt{FALSE}) \wedge f_{q'}(k) = C_2$
\end{itemize}
\end{itemize}

We define $L_I$ as follows:
\begin{itemize}
\item A state $q \in Q_I$ is a tuple of the form $(sac_q, src_q, UBSA_q, LBSA_q, CC_q, ubi,lbi, f_q)$ where $sac_q \in \mathbb{N}$ keeps the number of adds that return true so far, $src_q \in \mathbb{N}$ keeps the number of successful linearizations of remove (ones that move program counter from $R_0$ to $R_{1T}$), $f:\mathbb{N} \rightarrow Lbl_I$ is program counter that maps every operation to a label in $Lbl_I = \{N, A_0, A_1, R_0, R_{1T}, R_{1F}, R_2, C_0, C_1\}$. Let $PA_q = \{k \in \mathbb{N}| f_q(k) = A_0\}$ be the set of pending adds at state $q$. Then, $UBSA_q, LBSA_q: \mathbb{N} \rightarrow 2^{PE_q}$ are functions such that $UBSA_q(i)$ is a set of pending adds at most $i$ of which may return true and $LBSA_q(i)$ is a set of pending adds at least $i$ of which may return true. $ubi, lbi \in \mathbb{N}$ keeps the upper bound (lower bound) set indices that a new pending enqueue will be inserted to. Let $IC_q = \{k \in \mathbb{N}| f_q(k) = C_0 \vee f_q(k) = C_1$  be the set of contains operation identifiers. Then, $CC_q: IC_q \rightarrow 2^{PA}$ is a map that keeps a set of pending adds of which at least one should return true due to a true return of a contains operation.
\item Transition labels are exactly the abstract transition labels we have defined earlier: $\Sigma_I = AS\Sigma$
\item ${q_0}_I =(sac_{{q_0}_I}, src_{{q_0}_I}, UBSA_{{q_0}_I}, LBSA_{{q_0}_I}, CC_{{q_0}_I}, ubi_{{q_0}_I}, lbi_{{q_0}_I}, f_{{q_0}_I})$ where $sac_{{q_0}_I} = src_{{q_0}_I} =  0$, $lbi_{{q_0}_I} = ubi_{{q_0}_I} = 1$ and $f_{{q_0}_I}(k) = N$ for all $k \in \mathbb{N}$. Hence $PA_{{q_0}_I} = IC_{{q_0}_I} = \emptyset$ and $UBSA_{{q_0}_I}$, $LBSA_{{q_0}_I}$ and $CC_{{q_0}_I}$ are empty mappings.
\item Instead of giving state transition relation formally, I would like to explain how $UBSA$, $LBSA$ and $CC$ works by considering the possible state transformations (consider $q$ as pre-state and $q'$ as the post state as convention for the following):
\begin{itemize}
\item[$UBSA$] Initially, $ubi_{{q_0}_I} = 1$. Hence, a newly invoked add operation $a$ will be inserted into $UBSA_{{q_0}_I}(1)$. All the newly invoked operations are inserted to $UBSA_q(1)$ until a linearization of remove comes or one of the adds return successful. If the first case happens, we increment the index of every set by 1 i.e. $UBSA_{q'}(i) = UBSA_q(i-1)$ for all $i>0$ and $UBSA_{q'}(0) = \emptyset$. We also set $ubi_{q'} = 1$, if it was set to $0$ somehow before (Consider the trace $inv(add,a,1), ret(add,true,1), inv(add,a,2), lin(rmv,a,3)$. $ubi =0$ after second event and it needs to be incremented to $1$ after the fourth event). Next, consider the case of successful add. Let the operation identifier of the add that will return successful be $k$ and $i$ be the index such that $k \in UBSA_q(i)$. Then, for all $j > i$, $UBSA_{q'} (j-1) = UBSA_q(j)$, $UBSA_{q'}(i-1) = UBSA_q(i-1) \cup UBSA_q(i) \backslash \{ k\}$ and for all $j < i-1$, $UBSA_{q'}(j) = UBSA_q(j)$. We do not allow elements in $UBSA_q(0)$ to return true. So, $i=0$ cannot be true. If $i = 1$, we change $ubi_{q'} = 0$ to denote that no new coming add can return true from now on (consider the case: $inv(add,a,1), lin(rmv,a,2):true, inv(add,a,3), inv(add,a,4), lin(rmv,a,5):true, ret(add,true,3), ret(add,true,4), inv(add,a,6)$. This last invocation should be added to set with index $0$. So, we need to change $ubi$ to $0$ after last return. Lastly, a failing return of add simply removes this add from its set, without changing its index. Let $i$ be the index of the failing add operation $k$. Then, $UBSA_{q'}(i) = UBSA_q(i) \backslash \{k\}$. Observing a failing contains operation may change the upper bound. Consider the following trace: $inv(add,a,1), inv(add,a,2), inv(cnt,a,3), ret(cnt,false,3)$. Although $UBSA_{q'}(1) = \{1,2\}$ after the last event, none of $1$ or $2$ can return true due to the restrictions we impose on contains operations that we will explain later on.
\item[$LBSA$] Initially $lbi_{{q_0}_I} = 1$. When a new event of type invoke add comes, we modify $LBSA_{q'}(lbi_{q'}) = PE_{q'}$ where $lbi_{q'} = lbi_q$. Although it looks like that we build $LBSA_{q'}$ from scratch with every new add invocation, in practice, this is merely adding new pending add to the old set of the same index unless the $lbi$ field are the same. If a new successful linearization of remove comes, first it updates $LBSA_{q'}(lbi_q) = PE_{q'}$. Then, it increments the $lbi$ value by one if it was not $0$ before. If it was $0$ before, new $lbi$ value becomes $src_{q'} - sac_{q'}+1$. Note that a successful remove does not modify $LBSA$ field if a new add is invoked after the previous linearization of a successful remove. The first new add invocation that comes after a successful remove, copies the set $LBSA_{q}(lbi_q)$ and adds the new add's operation identifier as $LBSA_{q'}(lbi_{q'})$. However, if no new add comes between two successful remove linearizations, then $LBSA$ is modified by a sucsessful remove linearization. Next, consider the return events of add operations. First, consider the successful return. If an add operation with identifier $k$ returns true, we remove this identifier from the $LBSA$ sets of which $k$ is element of and decrement the index values of these sets by $1$. We can observe that if $k$ is the element of the set with index $i$ then it is element of every set with index $j>i$ \textcolor{red}{(We need to prove this)}. Hence we decrement the index of every set bigger than some $i$ value. For this case, we may end up with a situation that two sets fall into same index (at index $i-1$). In this case, one of the sets must be strictly subset of the other one \textcolor{red}{(We need to prove this)}. In this case, we keep the subset as the set of this index and delete the larger set. If $k$ is deleted from just no sets, we change $lbi_{q'} = 0$. To rationalize this behavior, consider the trace $inv(add,a,1), lin(rmv,a,2):true, inv(add,a,3), inv(add,a,4), lin(rmv,a,5):true, ret(add,true,3), ret(add,true,4), inv(add,a,6)$. The last add invocation should be put to the set with index $0$ and after the add with identifier $4$ returns true, our procedure makes $lbi_{q'} = 0$. If the removed element is in $LBSA_q(lbi_q) \backslash LBSA_q(lbi_q -1)$, then we also set $lbi_{q'} = 0$. The rationale behind this is the trace: $inv(add,a,1), lin(rmv,a,2):true, inv(add,a,3), ret(add,true,3), inv(add,a,4)$. The last operation with ID $4$ should be able to return false. If an add operation with identifier $k$ returns false, then we remove $k$ from all $LBSA$ sets that $k$ is element of. The constraint we check on $LBSA$ sets is that a return false event of an add operation keeps $|LBSA_{q'}(i)| \geq i$ for the indices $i$ it modified.
\item [$CC$] We need to keep a cc flag to show that now this operation may return true. It should be a map from identifiers to Boolean.
\end{itemize}
\end{itemize}

\end{document}
