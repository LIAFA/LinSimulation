% Suha's Proposal
\documentclass[orivec]{llncs}


\usepackage{amssymb}
\usepackage{graphicx}
\usepackage{listings}
\usepackage{arydshln}
\usepackage{amsmath}
%\usepackage{amsthm}
\usepackage{color}


\usepackage{amsfonts}
\usepackage{xspace}
\usepackage{latexsym}
\usepackage{wasysym} % causing problems with llncs's bold vectors
\usepackage{stmaryrd}
\usepackage{alltt}
\usepackage{mathpartir}
\usepackage[ligature,reserved,inference]{semantic}
\usepackage[lined]{algorithm2e}
\def\ttat{\mtt{@}} % the at package clobbers this
\usepackage{at}
\usepackage{alltt}
\usepackage{listings}
\usepackage[lined]{algorithm2e}
\usepackage{wrapfig}
%\newcommand{\xRightarrow}[2][]{\ext@arrow 0359\Rightarrowfill@{#1}{#2}}

\newcommand{\atCP}{\makeatletter @\makeatother}

\usepackage{pslatex}

\usepackage[numbers]{natbib}

\input macros
\input macros-programs

\lstset{
  language={Java},
  basicstyle=\ttfamily\scriptsize
}


\newtheorem{dfn}{Definition}
\newtheorem{lem}{Lemma}

\title{Proving linearizability using forward simulations}
\author{ }
\institute{ } 
\begin{document}

%\raggedbottom

\maketitle

\vspace{-10mm}
\begin{abstract}
Efficient concurrent implementations of abstract data structures such as stacks and queues are especially susceptible to programming errors, and necessitate automatic verification.
Linearizability is the correctness criterion in this context. It allows to establish observational refinement between a concurrent implementation and an atomic reference implementation.
Proving linearizability, or equivalently observational refinement, requires identifying linearization points for each method invocation along all possible computations, leading to valid sequential executions, or alternatively, establishing forward and backward simulations. In both cases, carrying out proofs is hard and complex in general. In particular, backward reasoning is difficult in the context of programs with data structures, and strategies for identifying statically linearization points cannot be defined for all existing implementations.  In this paper, we show that, contrary to common belief, many such complex implementations, including, e.g., the Herlihy\&Wing queue and the Time-Stamped Stack, can be proved correct using only forward simulation arguments. This leads to simple and natural correctness proofs for these implementations that are amenable to automation. 

%Efficient implementations of atomic objects such as concurrent stacks and queues are especially susceptible to programming errors, and necessitate automatic verification. Unfortunately their correctness criteria -- conformance to atomic reference implementations -- are hard to verify. Conformance can be formalized as a trace inclusion problem, also known as observational refinement, which in general, can be proved using compositions of forward and backward simulations. Contradicting common belief, we show that many complex implementations, e.g., the Herlihy and Wing queue and the Time-Stamped Stack, can be proved correct using only forward simulation arguments. We define classes of queue and stack implementations for which forward simulations are sound and complete for proving their correctness.
\vspace{-2mm}
\end{abstract}

%!TEX root = draft.tex
\section{Introduction}

Efficient concurrent implementations of atomic collections including stacks and queues are vital to modern computer systems. Programming them is however error prone. To minimize synchronization overhead between concurrent method invocations, implementors avoid blocking operations like lock acquisition, allowing methods to execute concurrently. However, concurrency risks unintended inter-operation interference, and risks conformance to atomic reference implementations. Conformance is formally captured by \emph{observational refinement}, which assures that all behaviors of programs
using these efficient implementations would also be possible were the atomic reference implementations used instead.

Observational refinement can be formalized as a trace inclusion problem, and the latter can itself be reduced to an assertion (invariant) checking problem, but this requires in general the consideration of additional history and prophecy variables. Alternatively, verifying observational refinement requires in general establishing both a forward simulation {\em and} a backward simulation. While simulations are natural concepts, reasoning about backward simulations, corresponding to the use of prophecy variables, is however less intuitive, and from the practical point of view, backward reasoning is in general hard and complex for programs manipulating data structures. Therefore, a crucial issue is to understand the limits of forward reasoning in establishing observational refinement. More precisely, we consider that an important question is to determine for which concurrent abstract data structures, and for which types of their implementation, is it possible to carry out an observational refinement proof using only forward reasoning.

In order to get rid of backward simulations (and equivalently of prophecy variables), it is necessary to have reference implementations of the considered abstract data structures that are {\em deterministic}. Interestingly, determinism allows also to simplify the forward simulation checking problem. Indeed, in this case, this problem can be reduced to an invariant checking problem. Basically, the simulation relation can be seen as an invariant of the system composed by the two compared programs. Therefore, existing methods and tools for invariant checking can be leveraged in the context of proving observational refinement. 

But, in order to determine precisely what is meant by determinism, an important point is to fix the alphabet of observable events along computations. 
Typically, to reason about observational refinement between two library implementations, the only observable events are the calls and returns corresponding to the method invocations along computations. This means that only the external interface of the library is considered to compare behaviors, and nothing else from the implementations is exposed. Unfortunately, it can be shown that in this case, it is impossible to have deterministic atomic reference implementations for common data structures such as stacks and queues (see, e.g., \cite{DBLP:conf/cav/SchellhornWD12}). Then, an important question is what is the necessary amount of information that should be exposed by the implementations to overcome this problem ?

One approach addressing this question is based on the notion of linearizability \cite{journals/toplas/HerlihyW90} and its correspondence with observational refinement (w.r.t. atomic reference implementations) \cite{journals/tcs/FilipovicORY10,DBLP:conf/popl/BouajjaniEEH15}. Linearizability of a computation (of some concurrent implementation) means that each of the method invocations can be seen as happening at some point, called {\em linearization point}, occurring somewhere between the call and return events of that invocation, such that the obtained sequence of linearization points along the computation defines a sequence of operations that is possible in the atomic reference implementation. Proving the existence of such sequences of linearization points, for all the computations of a concurrent library, is a complex problem \cite{journals/iandc/AlurMP00,conf/esop/BouajjaniEEH13,DBLP:conf/netys/Hamza15}. 
%
However, proving linearizability becomes less complex when linearization points are fixed for each method, i.e., associated (once and for all) with the execution of a designated statement in its source code \cite{conf/esop/BouajjaniEEH13}. In this case, we can consider that libraries expose not only the call and returns events, but also events signaling linearization points. By extending this way the alphabet of observable events, it becomes straightforward to define {\em deterministic} atomic reference implementations. Therefore, proving linearizability can be carried out as an invariant checking problem when linearization pointst are fixed, e.g.,~\cite{conf/ppopp/VafeiadisHHS06,conf/cav/AmitRRSY07,conf/vmcai/Vafeiadis09,conf/tacas/AbdullaHHJR13}.
%
While this approach can be useful in some cases, it is unfortunately not applicable to many efficient implementations such as the LCRQ queue~\cite{DBLP:conf/ppopp/MorrisonA13} (based on the principle of the Herlihy\&Wing queue \cite{journals/toplas/HerlihyW90}), and the Time-Stamped Stack \cite{DBLP:conf/popl/DoddsHK15}. The proofs of linearizability of these implementations are highly nontrivial, very involved, and hard to read, understand and automatize. Therefore, the crucial question we address is what is precisely the kind of information that is necessary to expose in order to obtain deterministic atomic reference implementations for such data structures, allowing to derive simple and natural linearizability proofs for such complex implementations, which can be carried out as systematic invariant checking proofs that are amenable to automation?

We observe that the main difficulty in reasoning about these implementations is that, linearization points of enqueue/push operations occurring along some given computation, depend in general on the linearization points of dequeue/pop operations that are occurring later in that computation, at arbitrary far positions. Therefore, since linearization points for enqueue/push operations cannot be determined in advance, the information that could be fixed and exposed can concern only the dequeue/pop operations. 

One first idea is to consider that linearization points are fixed for dequeue/pop methods and only for these methods. We show that under the assumption that implementations expose linearizations points for these methods, it is possible to define deterministic atomic reference implementations for both queues and stacks. We show that this is indeed useful by providing a simple proof of the Herlihy\&Wing queue (based on establishing a forward simulation) that can be carried out as an invariant checking proof.

However, in the case of the Time-Stamped Stack, fixing linearization points of pop operations is actually too restrictive. Nevertheless, we show that our approach can be generalized to handle this case. The key idea is to reason about what we call {\em commit points}, and  that correspond roughly speaking to the last point a method accesses to the shared data structure during its execution. We prove that by exposing commit points (instead of linearization points) for pop methods, we can still provide deterministic reference implementations. We show that using this approach leads to a quite simple proof of the Time-Stamped Stack.


%Efficient implementations of concurrent objects such as semaphores, locks, and atomic collections including stacks and queues are vital to modern computer systems. Programming them is however error prone. To minimize synchronization overhead between concurrent object-method invocations, implementors avoid blocking operations like lock acquisition, allowing methods to execute concurrently. However, concurrency risks unintended inter-operation interference, and risks conformance to atomic reference implementations. Conformance is formally captured by \emph{observational refinement}, which assures that all behaviors of programs
%using these efficient implementations would also be possible were the atomic
%reference implementations used instead.
%An equivalent property is \emph{linearizability}~\cite{journals/toplas/HerlihyW90} which requires that each concurrent execution
%has a linearization which is a valid sequential execution according to a
%specification, given by an abstract data type or reference implementation.
%
%Existing proof methods for refinement/linearizability combine reductions to assertion checking with compositional reasoning techniques for proving the assertions.
%
%The reductions to assertion checking are based on simulation relations: the assertions describe a simulation relation to standard atomic reference implementations
%
%No surprise, simulations are the standard approach for proving refinement
%
%This works smoothly for particular classes of implementations, e.g., with fixed linearization points. The explanation is that in this case, forward simulations are enough and the reference implementation is deterministic when the linearization points are observable
%
%Why they are enough ? Forward simulations are not complete for proving refinement: the reference implementation is not deterministic when projected over call and returns. 
%
%However, in this case, the set of observable actions includes the linearization points. And the reference implementations becomes deterministic for this larger set of observable actions.
%
%Determinism also simplifies the assertions
%
%When linearization points are not observable, reference implementations are not deterministic and one may need backward simulations. There is a paper arguing for this fact ~\cite{DBLP:conf/cav/SchellhornWD12}.
%
%However, using backward simulations is hard, they require prophecy variables. They are not good for automation.
%
%Question: We want to avoid backward simulations, so what can we do about implementations without fixed linearization points, e.g, Herlihy\&Wing, Time-Stamped Stack ? Can they be proved using forward simulations ?
%
%We answer positively for queues and stacks. The idea is to define new abstract implementations over a set of observable actions bigger than the set of call/returns, but smaller than all linearization points.
%
%Queues: only dequeues have fixed linearization points.
%
%Stacks: pops have fixed commit points
%
%\newpage%~\newpage

\section{Preliminaries}
Systems we consider are labeled transition systems (LTS):
\begin{dfn}
An LTS is defined over four-tuples $A=(Q,\Sigma, q_0, \delta)$ where $Q$ is the set of states, $\Sigma$ is the set of transition labels, $q_0 \in Q$ is the initial state and $\delta \subseteq Q \times \Sigma \times Q$ is the transition relation.
\end{dfn}
Executions generated by this system are alternating sequence of states and transition labels $\rho = s_0, e_0, s_1,... s_k, e_k,...$ where each $s_i \in Q$, each $e_i \in \Sigma$, $s_0 = q_0$ and each $(s_i, e_i s_{i+1}) \in \delta$. The projection of the sequence $\rho$ over the set $\Pi$ is denoted by $\rho | \Pi$, and it is the maximum subsequence of $\rho$ consisting of elements of $\Pi$. Traces of the LTS are obtained from executions by projecting them over $\Sigma$. For the rest of the paper and in all of the proofs, we consider only finite executions (denoted as $E(A)$) and/or traces (denoted as $Tr(A)$ of the LTSs in focus.

Libraries are LTSs that provide methods. Let $\mathcal{M}$ be the set of method names and $\mathcal{D}$ be the domain of values as input/output parameters for the methods. Then, this library contains transition labels of the form $inv(m,d,i)$ representing the invocation of method $m \in \mathcal{M}$ with input value $d \in \mathcal{D}$. The third field is the operation identifier for differentiating the different calls of the same method from the set $\mathcal{O}$. For simplicity, we take $\mathcal{O} = \mathbb{N}$ for the rest of the paper. We also assume that methods could have at most one input parameter. If they do not have any input arguments (like pop method of a stack), we can omit the second field from the action. They also provide actions of the form $ret(m,d,i)$ representing the return of method $m \in M$ with value $d \in D$ which has been invoked previously with action $inv(m,d',i)$. Again, we assume that the methods can return at most one parameter and we may omit the second field from the action if they have none (like enqueue method of a queue). Before starting to reason about any set of libraries, we first fix the sets $\mathcal{M}$ and $\mathcal{D}$ and libraries in our focus agree on this sets. For any transition label $e = inv(m,d,i)$ or $e=ret(m,d,i)$, we have the function $oid(e) = i$.

Since libraries are LTSs, they produce traces. A trace $e = e_1, e_2, ..., e_n$ of library $L$ is \emph{well-formed} iff (i) every return matches an earlier invocation: $e_j = ret(m,d,k)$ implies that there exists $i<j$ such that $e_i = inv(m,d',k)$ and (ii) every operation identifier is used at most one invocation/return pair: $oid(e_i) = oid(e_j) = k$ and $i<j$ implies $e_i = inv(m,d,k)$ and $e_j = ret(m,d',k)$. From now on, we assume that libraries produce well-formed traces. Let $f: \mathbb{N} \rightarrow \mathbb(N)$ be a bijection. Then, traces $e$ and $e'$ are equivalent if $e'$ is obtained from $e$ by replacing every action $inv(m,d,k)$ with $inv(m,d,f(k))$ and every action $ret(m,d,k)$ with $ret(m,d,f(k))$. 

Based on these definitions on the traces of the libraries, we can define refinement between libraries:
\begin{dfn}
Let $L_1$ and $L_2$ be two libraries agreeing on $\mathcal{M}$ and $\mathcal{D}$ sets. We define the set $A\Sigma = \{inv(m,d,i)| m \in \mathcal{M} \wedge d \in \mathcal{D} \wedge i \in \mathbb{N}\} \cup \{ret(m,d,i)| m \in \mathcal{M}\wedge d \in \mathcal{D} \wedge i \in \mathbb{N}\}$ as abstract transition labels. Note that $A\Sigma \subseteq \Sigma_{L_1}$ and $A\Sigma \subseteq \Sigma_{L_2}$. Then, we say $L_1$ refines $L_2$ iff for every trace $e \in Tr(L_1)$, there exists a trace $e' \in Tr(L_2)$ such that $e|A\Sigma$ is equivalent to $e'|A\Sigma$.
\end{dfn}
Linearizability is also a relation between two libraries and it is stricter than refinement. It requires $e'$ in Definition 2 to be a sequential one. A trace $e$ is sequential iff following two conditions hold for its projection to abstract transition labels $e|A\Sigma = e_1, ...e_n$: $(i)$ $e_1 = inv(m,d,k)$ for some $m \in \mathcal{M}, d \in \mathcal{D} and k \in \mathbb{N}$ and $(ii)$ for all $i \in [1,n)$, either $e_i = inv(m,d,k)$ and $e_{i+1} = ret(m,d',k)$ or $e_i = ret(m,d,k)$ and $e_{i+1} = inv(m',d',k')$ for some $m,m' \in \mathcal{M}$, $d,d' \in \mathcal{D}$ and $k,k' \in \mathbb{N}$.

We can extend the relations between libraries by introducing simulation relations. We will later show that simulation relations imply refinement. 

\begin{dfn}
Let $L_1$ and $L_2$ be two libraries agreeing on $\mathcal{M}$ and $\mathcal{D}$ sets. Then, the relation $fs \subseteq Q_{L_1} \times Q_{L_2}$ is called a forward simulation iff the following holds:
\begin{itemize}
\item[(i)] $fs[q_{0_{L_1}}] = \{q_{0_{L_2}} \}$
\item[(ii-a)] If $(s,inv(m,d,k),s') \in \delta_{L_1}$ and $u \in fs[s]$, then there exists $u' \in fs[s']$ such that $u \xrightarrow{a} u'$ where $a = a_1, a_2, ..., a_n$ such that $a_1 = inv(m,d,k)$ and for all $i \in [2,n]$, $a_i \in \Sigma_{L_2} \backslash A\Sigma $. The expression $u \xrightarrow{a} u'$ means that there exists a sequence of states $u_1, u_2,...,u_{n+1}$ such that $u_1 = u$, $u_{n+1} = u'$ and for all $i \in [1,n]$, $(u_i, a_i, u_{i+1}) \in \delta_{L_2}$.
\item[(ii-b)] If $(s,ret(m,d,k),s') \in \delta_{L_1}$ and $u \in fs[s]$, then there exists $u' \in fs[s']$ such that $u \xrightarrow{a} u'$ where $a = a_1, a_2, ..., a_n$ such that $a_n = ret(m,d,k)$ and for all $i \in [1,n-1]$, $a_i \in \Sigma_{L_2} \backslash A\Sigma $. 
\item[(ii-c)] If $(s,t,s') \in \delta_{L_1}$ for some $t \in \Sigma_{L_1}\backslash A\Sigma$ and $u \in fs[s]$, then there exists $u' \in fs[s']$ such that $u \xrightarrow{a} u'$ where $a = a_1, a_2, ..., a_n$ such that for all $i \in [1,n]$, $a_i \in \Sigma_{L_2} \backslash A\Sigma $. Moreover, $a$ could be the empty sequence.
\end{itemize}
\end{dfn}
If $fs[s]$ is a unique state for all $s \in Q_{L_1}$ then it is called a refinement mapping/function. A dual notion of forward simulation is the backward simulation:
\begin{dfn}
Let $L_1$ and $L_2$ be two libraries agreeing on $\mathcal{M}$ and $\mathcal{D}$ sets. Then, the relation $bs \subseteq Q_{L_1} \times Q_{L_2}$ is called a backward simulation iff the following holds:
\begin{itemize}
\item[(i)] $bs[q_{0_{L_1}}] = \{q_{0_{L_2}} \}$
\item[(ii-a)] If $(s,inv(m,d,k),s') \in \delta_{L_1}$ and $u' \in bs[s']$, then there exists $u \in bs[s]$ such that $u \xrightarrow{a} u'$ where $a = a_1, a_2, ..., a_n$ such that $a_1 = inv(m,d,k)$ and for all $i \in [2,n]$, $a_i \in \Sigma_{L_2} \backslash A\Sigma $. 
\item[(ii-b)] If $(s,ret(m,d,k),s') \in \delta_{L_1}$ and $u' \in bs[s']$, then there exists $u \in bs[s]$ such that $u \xrightarrow{a} u'$ where $a = a_1, a_2, ..., a_n$ such that $a_n = ret(m,d,k)$ and for all $i \in [1,n-1]$, $a_i \in \Sigma_{L_2} \backslash A\Sigma $. 
\item[(ii-c)] If $(s,t,s') \in \delta_{L_1}$ for some $t \in \Sigma_{L_1}\backslash A\Sigma$ and $u' \in bs[s']$, then there exists $u \in bs[s]$ such that $u \xrightarrow{a} u'$ where $a = a_1, a_2, ..., a_n$ such that for all $i \in [1,n]$, $a_i \in \Sigma_{L_2} \backslash A\Sigma $. Moreover, $a$ could be the empty sequence.
\end{itemize}
\end{dfn}
Simulation relations are used to prove refinement relations among libraries. Following lemmas show their soundness:
\begin{lem}
Let $L_1$ and $L_2$ be two libraries agreeing on $\mathcal{M}$ and $\mathcal{D}$ sets. If $fs$ ($bs$) is a forward (backward) simulation relating $L_1$ to $L_2$, then $L_1$ refines $L_2$.
\end{lem}
\begin{proof}
Looks trivial and follows Lynch paper. Can be completed later.
\end{proof}
%!TEX root = draft.tex

\section{Refinement Proofs}

Library refinement is the instance of a more general notion of refinement between LTSs
which for some alphabet $\Gamma$ of \emph{observable actions} is defined as the inclusion of sets of 
traces projected on $\Gamma$. Library refinement corresponds to the case $\Gamma=C\cup R$. 
Typically, $\Gamma$-refinement between two LTSs $A$ and $B$ is proved using \emph{simulation relations} which roughly, require that 
$B$ can mimic every step of $A$ using a (possibly empty) sequence of steps. Mainly, there are two kinds of simulation
relations, forward or backward, depending on whether the preservation of steps is proved starting from a similar state
forward or backward. It has been shown
that $\Gamma$-refinement is equivalent to the existence of \emph{backward simulations}, modulo the addition of history variables
that record events in the implementation, and to the existence of \emph{forward simulations} provided that the right-hand side
LTS $B$ is $\Gamma$-deterministic~\cite{DBLP:journals/tcs/AbadiL91,DBLP:journals/iandc/LynchV95}. 
We focus on proofs based on forward simulations because they are easier to automatize.

TODO SAY THAT THE EXTENSION TO $\Gamma$-determinism is new (the previous results were about determinism).

In general, forward simulations are \emph{not} a complete proof method for library refinement because libraries are not 
$C\cup R$-deterministic (the same sequence of call and 
return actions can lead to different states depending on the interleaving of the internal actions).
However, there are classes of atomic libraries, e.g., libraries with ``fixed linearization points'' (defined later in this section), 
for which it is possible to identify a larger alphabet $\Gamma$ of observable actions (including call/return actions), 
and implementations that are $\Gamma$-deterministic. When focusing on queues and stacks, 
Section~\ref{sec:queues} and Section~\ref{sec:stacks} define other such classes of implementations that cover
all the implementations that we are aware of.

%Refinement between two libraries $L_1$ and $L_2$  can be proved using  In general, forward simulations are easier to establish, but they are incomplete, i.e., refinement may hold while
%there exists no forward simulation. On the other hand, proofs based on backward simulations are more difficult, but modulo the
%addition of history variables (auxiliary variables recording events in the implementation) backward simulations are complete for 
%proving refinement~\footnote{Adding history variables has no impact on the incompleteness of forward simulations.}. 

%This section introduces notions used to enlarge the scope of proofs based on forward simulations, which is the main goal of this paper.
%We first introduce a stronger notion of refinement that compares the set of traces of two libraries projected
%on a larger alphabet of actions than the set of call/return actions. These actions are called \emph{observable actions},
%and they may represent for instance, linearization points which informally, are points in time where an operation 
%is conceptually effectuated. We then define a notion of forward simulation which is complete for proving
%this notion of refinement provided that the alphabet of the ``abstract'' library $L_2$ equals the set of 
%observable actions and $L_2$ is deterministic. We discuss an application of this result, namely,
%proving the correctness of libraries with ``fixed linearization points'', where the so-called ``linearization points''
%are fixed to particular implementation actions.

\begin{dfn}\label{def:gref}
Let $L_1$ and $L_2$ be two libraries over alphabets $\Sigma_1$ and $\Sigma_2$, respectively, such that $C\cup R \subseteq \Sigma_1\cap\Sigma_2$. Also, let $\Gamma$ be a set of actions such that $C\cup R\subseteq \Gamma\subseteq \Sigma_1\cap\Sigma_2$. 
The library $L_1$ \emph{$\Gamma$-refines} $L_2$ if{f} $Tr(L_1) | \Gamma \subseteq Tr(L_2) | \Gamma$.
%We say that $L_1$ \emph{refines} $L_2$ when $H(L_1) \subseteq H(L_2)$.
%for every trace $e \in Tr(L_1)$, there exists a trace $e' \in Tr(L_2)$ such that $e|A\Sigma$ is equivalent to $e'|A\Sigma$.
\end{dfn}

Notice that $\Gamma$-refinement implies refinement for any $\Gamma$ as in Definition~\ref{def:gref}.

We define a notion of \emph{forward} simulation that can be used to prove $\Gamma$-refinement. 
A dual notion of \emph{backward} simulation is defined in Appendix~\ref{app:backSim}. 
%We define a notion of forward simulation that is a sound approximation of refinement and a notion of backward simulation that precisely characterizes refinement. 
For a relation $R\subseteq A\times B$, $R[X]$ denotes the set of elements of $B$ related to elements of $X$ by $R$, i.e., $R[X]=\set{y:\exists x\in X.\ R(x,y)}$.

%We can extend the relations between libraries by introducing simulation relations. We will later show that simulation relations imply refinement. 

\begin{dfn}
Let $L_1=(Q_1,\Sigma, s_0^1, \delta_1)$ and $L_2=(Q_2,\Sigma, s_0^2, \delta_2)$ be two libraries over alphabets $\Sigma_1$ and $\Sigma_2$, respectively, such that $C\cup R \subseteq \Sigma_1\cap\Sigma_2$, and $\Gamma$ a set of actions such that $C\cup R\subseteq \Gamma\subseteq \Sigma_1\cap\Sigma_2$. A relation $\mathit{fs} \subseteq Q_{1} \times Q_{2}$ is called a \emph{$\Gamma$-forward simulation} from $L_1$ to $L_2$ iff the following holds:
\begin{itemize}
\item[(i)] $\mathit{fs}[s_0^1] = \{s_0^2 \}$ 
\item[(ii-a)] If $(s,c,s') \in \delta_1$, for some $c\in C$, and $u \in \mathit{fs}[s]$, then there exists $u' \in \mathit{fs}[s']$ such that $u \xrightarrow{@s} u'$, $@s_0=c$, and $@s_i\in \Sigma_2\setminus\Gamma$, for each $0<i<|@s|$.
%$@s = a_1, a_2, ..., a_n$ such that $a_1 = inv(m,d,k)$ and for all $i \in [2,n]$, $a_i \in \Sigma_{L_2} \backslash A\Sigma $. The expression $u \xrightarrow{a} u'$ means that there exists a sequence of states $u_1, u_2,...,u_{n+1}$ such that $u_1 = u$, $u_{n+1} = u'$ and for all $i \in [1,n]$, $(u_i, a_i, u_{i+1}) \in \delta_{L_2}$.
\item[(ii-b)] If $(s,r,s') \in \delta_{1}$, for some $r\in R$, and $u \in \mathit{fs}[s]$, then there exists $u' \in \mathit{fs}[s']$ such that $u \xrightarrow{@s} u'$, $@s_{|@s| -1}=r$, and $@s_i\in \Sigma_2\setminus\Gamma$, for each $0\leq i<|@s| -1$.
% where $a = a_1, a_2, ..., a_n$ such that $a_n = ret(m,d,k)$ and for all $i \in [1,n-1]$, $a_i \in \Sigma_{L_2} \backslash A\Sigma $. 
\item[(ii-c)] If $(s, \gamma , s') \in \delta_1$, for some $\gamma\in \Gamma\setminus (C\cup R)$, and $u \in fs[s]$, then there exists $u' \in fs[s']$ such that $\delta_2(u,\gamma, u')$. 
\item[(ii-d)] If $(s,e,s') \in \delta_1$, for some $e \in \Sigma_1\setminus \Gamma$ and $u \in \mathit{fs}[s]$, then there exists $u' \in \mathit{fs}[s']$ such that $u \xrightarrow{\sigma} u'$ and $@s_i\in \Sigma_2\setminus\Gamma$, for each $0\leq i<|@s|$.  
%where $a = a_1, a_2, ..., a_n$ such that for all $i \in [1,n]$, $a_i \in \Sigma_{L_2} \backslash A\Sigma $. Moreover, $a$ could be the empty sequence.
\end{itemize}
\end{dfn}
%If $\mathit{fs}[s]$ is a unique state for all $s \in Q_1$ then $\mathit{fs}$ is called a refinement mapping/function. 
A $\Gamma$-forward simulation shows that every step of $L_1$ corresponds to a (possibly empty) sequence of steps of $L_2$. In order to imply $\Gamma$-refinement, steps of $L_1$ labeled by observable actions are treated in a particular way: a step of $L_1$ labeled by a call, resp., return, action is simulated by a sequence of steps of $L_2$ that start, resp., end, with the same action, and a step of $L_1$ labeled by another observable action should be matched by a step of $L_2$ labeled by the same action. The rest of the transitions in $L_1$ are matched to a possibly empty sequence of transitions of $L_2$ with arbitrary labels.

The following theorem shows the soundness and the completeness of $\Gamma$-forward simulations.
\begin{theorem}\label{th:forSim}
Let $L_1$ and $L_2$ be two libraries over alphabets $\Sigma_1$ and $\Sigma_2$, respectively, such that $C\cup R \subseteq \Sigma_1\cap\Sigma_2$. Also, let $\Gamma$ be a set of actions such that $C\cup R\subseteq \Gamma\subseteq \Sigma_1\cap\Sigma_2$. If there exists a $\Gamma$-forward simulation from $L_1$ to $L_2$, then $L_1$ $\Gamma$-refines $L_2$. Moreover, if $L_1$ $\Gamma$-refines $L_2$ and $L_2$ is $\Gamma$-deterministic, then there exists a $\Gamma$-forward simulation from $L_1$ to $L_2$.
\end{theorem}
\begin{proof}
TODO
\end{proof}

The linearization of a concurrent history can be also defined in terms of \emph{linearization points}. Informally, a linearization point of 
an operation in an execution is a point in time where the operation is conceptually effectuated; given the linearization points of 
each operation, the linearization of a concurrent history is the sequential history which takes operations in order of their linearization points.
For some libraries, the linearization points correspond to a fixed set of actions. For instance, in the case of atomic libraries  
where method bodies are guarded with a global-lock acquisition, the linearization point of every method invocation corresponds to the execution 
of the body. When the linearization points are fixed, we assume that the library is an LTS over an alphabet that includes actions 
$lin(m,d,k)$ with $m\in\<Methods>$, $d\in\<Vals>$ and $k\in \<Ops>$. The action $lin(m,d,k)$ represents the linearization point of the operation $k$ 
which is an invocation of method $m$ returning value $d$.
Let $Lin$ denote the set of such actions. We extend the notion of
well-formedness to sequences $@t$ over alphabet $\Sigma$,
such that $C\cup R\cup Lin\subseteq \Sigma$, by adding the following constraints:
\begin{itemize}
	\item every linearization point corresponds to an operation that has been called, i.e., TODO
	\item every linearization point occurs between the call and the return of the corresponding operation, i.e., TODO
	\item the return value of an operation is fixed by the linearization point, i.e., TODO
\end{itemize}
%The fact that a history $h_1$ is linearizable w.r.t. another sequential history $h_2$ can be stated as: there exists a way of choosing a linearization
%point 
%Formally, a \emph{linearization point annotation} (LP-annotation for short) of a history $h$ is a sequence $\hat{h}$ obtained from $h$ by 
%inserting actions $lin(m,d,k)$ with $m\in\<Methods>$, $d\in\<Vals>$ and $k\in \<Ops>$ such that:
%\begin{itemize}
%	\item every linearization point corresponds to an operation that has been called, i.e., TODO
%	\item every linearization point occurs between the call and the return of the corresponding operation, i.e., TODO
%	\item the return value of an operation is fixed by the linearization point, i.e., TODO
%\end{itemize}
We assume that libraries produce only well-formed traces. The projection of a library trace over $C\cup R\cup Lin$ is called an 
\emph{extended history}. A trace or extended history is called \emph{$Lin$-complete} when every completed operation has a linearization 
point, i.e., TODO. 
A library $L$ over alphabet $\Sigma$ is called \emph{with fixed linearization points} if{f} $C\cup R\cup Lin\subseteq \Sigma$ 
and every trace $@t\in Tr(L)$ is $Lin$-complete. 

Proving the correctness of an implementation $L_1$ of a concurrent object such as a queue or a stack with fixed linearization points
reduces to proving that $L_1$ is a $(C\cup R\cup Lin)$-refinement of an abstract implementation $L_2$ of the same object where method
bodies are guarded with a global-lock acquisition. Assuming that the abstract implementation is $(C\cup R\cup Lin)$-deterministic, which is typically the case,
by Theorem~\ref{th:forSim}, proving $(C\cup R\cup Lin)$-refinement is equivalent to finding a $(C\cup R\cup Lin)$-forward simulation 
from $L_1$ to $L_2$.

Section~\ref{sec:queues} and Section~\ref{sec:stacks} extend this result to implementations of queues and stacks where the linearization point of the methods 
that \emph{add} values to the collection is \emph{not} fixed.


%!TEX root = draft.tex
\vspace{-3.5mm}
\section{Queues With Fixed Dequeue Linearization Points}\label{sec:queues}
\vspace{-1.5mm}

The classical abstract queue implementation, denoted $AbsQ_0$, maintains a
sequence of enqueued values; dequeues return the oldest non-dequeued value, at
the time of their linearization points, or {\tt EMPTY}. Some implementations,
like the queue of \citet{journals/toplas/HerlihyW90}, denoted {\sc hwq}, are not
forward-simulated by $AbsQ_0$, even though they refine $AbsQ_0$, since the order
in which their enqueues are linearized to form $AbsQ_0$’s sequence is not
determined until later, when their values are dequeued.

In this section we develop an abstract queue implementation, denoted $AbsQ$,
which maintains a partial order of enqueues, rather than a linear sequence.
Since $AbsQ$ does not force refining implementations to eagerly pick among
linearizations of their enqueues, it forward-simulates many more queue
implementations. In fact, $AbsQ$ forward-simulates all queue implementations of
which we are aware that are not forward-simulated by $AbsQ_0$, including {\sc
hwq}, The Baskets Queue~\cite{DBLP:conf/opodis/HoffmanSS07}, The Linked
Concurrent Ring Queue ({\sc lcrq})~\cite{DBLP:conf/ppopp/MorrisonA13}, and The
Time-Stamped Queue~\cite{DBLP:conf/popl/DoddsHK15}.


%TODO HOW CAN WE CONVINCE THAT OTHER IMPLEMENTATIONS HAVE THE SAME PROPERTY ?
\vspace{-3.5mm}
\subsection{Enqueue Methods With Non-Fixed Linearization Points}
\vspace{-1mm}
% describe a queue implementation listed in Figure~\ref{fig:HerlihyWing} and known as the Herlihy \& Wing Queue~\cite{journals/toplas/HerlihyW90} ($\mathit{HWQ}$ for short), where only the dequeue methods have fixed linearization points.
We describe $\mathit{HWQ}$ where the linearization points of the enqueue methods are not fixed.
The shared state consists of an array {\tt items} storing the values in the queue and a counter {\tt back} storing the index of the first unused position in {\tt items}. Initially, all the positions in the array are {\tt null} and {\tt back} is 0.
An enqueue method starts by reserving a position in {\tt items} ({\tt i} stores the index of this position and {\tt back} is incremented so the same position can't be used by other enqueues) and then, stores the argument {\tt x} at this position. The dequeue method traverses the array {\tt items} starting from the beginning and atomically swaps {\tt null} with the encountered value. If the value is not {\tt null}, then the dequeue returns that value. If it reaches the end of the array, then it restarts.

\begin{wrapfigure}{l}{5.3cm}
\vspace{-9mm}
\begin{lstlisting}
void enq(int x){
  i = back++;
  items[i] = x;
}
int deq() {
  while (1) {
    range = back - 1;
    for (int i = 0; i <= range; i++){
      x = swap(items[i],null);
      if ( x != null ) return x;
}}}
  \end{lstlisting}
\vspace{-5.5mm}
\caption{Herlihy \& Wing Queue. We assume that every statement is atomic.}
\label{fig:HerlihyWing}
\vspace{-6mm}
\end{wrapfigure}
The linearization points of the enqueues are not fixed, they depend on dequeues executing in the future. Consider the following trace with two concurrent enqueues (${\tt i}(k)$ represents the value of {\tt i} in operation $k$): $inv(enq,x,1)$, $inv(enq,y,2)$, ${\tt i}(1) = \mbox{{\tt bck++}}$, ${\tt i}(2) = \mbox{{\tt bck++}}$, ${\tt items[i(}2{\tt )]} = y$.
%\vspace{-1.5mm}
%\begin{align*}
%inv(enq,x,1)\ \ \ inv(enq,y,2)\ \ \ {\tt i}(1) = \mbox{{\tt bck++}}\ \ \ {\tt i}(2) = \mbox{{\tt bck++}}\ \ \ {\tt items[i(}2{\tt )]} = y
%\vspace{-1.5mm}
%\end{align*}
%\begin{align*}
%inv(enq,x,1)\ inv(enq,y,2)\ ({\tt i}_1 = 0,{\tt bck} = 1)\ ({\tt i}_2 = 1,{\tt bck} = 2)\ ({\tt items[1]} = y)
%\end{align*}
Assuming that the linearization point corresponds to the assignment of {\tt i}, the history of this trace should be linearized to $inv(enq,x,1)$, $ret(enq,1)$, $inv(enq,y,2)$, $ret(enq,2)$. However, a dequeue executing until completion after this trace will return $y$ (only position $1$ is filled in the array {\tt items}) which is not consistent with this linearization. On the other hand, assuming that enqueues should be linearized at the assignment of {\tt items[i]} and extending the trace with ${\tt items[i(}1{\tt )]} = x$ and a completed dequeue that in this case returns $x$, leads to the incorrect linearization: $inv(enq,y,2)$, $ret(enq,2)$, $inv(enq,x,1)$, $ret(enq,1)$, $inv(deq,3)$, $ret(deq,x,3)$.
%\vspace{-1.5mm}
%\begin{align*}
%inv(enq,y,2)\ ret(enq,2) inv(enq,x,1)\ ret(enq,1)\ inv(deq,3)\ ret(deq,x,3).
%\vspace{-1.5mm}
%\end{align*}

The dequeue method has a fixed linearization point which corresponds to an execution of {\tt swap} returning a non-null value. This action alone contributes to the effect of that value being removed from the queue. Every concurrent history can be linearized to a sequential history where dequeues occur in the order of their linearization points in the concurrent history.
This claim is formally proved in Section~\ref{ssec:HerlihyWing}.

Since the linearization points of the enqueues are determined by future dequeue invocations, there exists no forward simulation from $\mathit{HWQ}$ to $AbsQ_0$.
In the following, we describe the abstract implementation $AbsQ$ for which such a forward simulation does exist.

\vspace{-3.5mm}
\subsection{Abstract Queue Implementation}
\vspace{-1.5mm}

%\textcolor{red}{Important: I think EMPTY return is problematic and I defined it wrong in my machine too. We should be able to return EMPTY if there are only pending nodes. Consider the following history of $AbsQ_0$: $inv(enq, d_1,k_1), inv(enq, d_2,k_2), inv(deq, k_3), lin(deq, \texttt{EMPTY}, k_3)$. This history should be reflected in $AbsQ$ by enabling lin \texttt{EMPTY} of dequeue when there are pending nodes. We also need to update the rules in figure.}
Informally, $AbsQ$ records the set of enqueue operations, whose argument has not yet been removed by a matching dequeue operation. In addition, it records the happens-before order between those enqueue operations: this is a partial order ordering an enqueue $k_1$ before another enqueue $k_2$ iff $k_1$ returned before $k_2$ was invoked.
%Informally, $AbsQ$ records the happens-before order between enqueue operations for which the added value has not been removed by a dequeue operation.
The linearization point of a dequeue can either remove a minimal enqueue $k$ (w.r.t. the happens-before stored in the state) and fix the return value to the value $d$ added by $k$, or fix the return value to ${\tt EMPTY}$ provided that the current state stores only pending enqueues (intuitively, the dequeue overlaps with all the enqueue operations stored in the current state and it can be linearized before all of them).
%with return value $d\neq{\tt EMPTY}$ is enabled only if the happens-before stored in the current state contains a minimal enqueue that adds the value $d$. The effect of the linearization point is that the minimal enqueue is removed from the current state and the return value is recorded in the library state.
%When
%%\noindent
%the return value is {\tt EMPTY}, the linearization point of a dequeue is enabled only if the current state stores only pending enqueues (because, the dequeue overlaps with all the enqueue operations stored in the current state and it can be linearized before all of them).
%The return of a dequeue is enabled only if the returned value matches the one fixed at the linearization point.

\begin{wrapfigure}{l}{7.2cm}
\vspace{-6mm}
\includegraphics[width=7.3cm]{fig-queue12.pdf}
%
%\vspace{2mm}
%\includegraphics[width=7cm]{fig-queue2.pdf}
\vspace{-8mm}
\caption{Simulating queue histories with $AbsQ$. An operation is pictured by a line delimited by two circles denoting the call and respectively, the return action. The representation of a dequeue operation includes a red circle that stands for a {\tt swap} returning a non-null value, which is their linearization point.}
\label{fig:queueSim}
\vspace{-6mm}
\end{wrapfigure}
Fig.~\ref{fig:queueSim} pictures two executions of $AbsQ$ for two extended histories (that include dequeue linearization points). The state of $AbsQ$ after each action is pictured as a graph below the action. The nodes of this graph represent enqueue operations and the edges happens-before constraints. Each node is labeled by a value (the argument of the enqueue) and a flag {\tt PEND} or {\tt COMP} showing whether the operation is pending or completed. For instance, in the case of the first history, the dequeue linearization point $lin(deq,y,3)$ is enabled because the current happens-before contains a \emph{minimal} enqueue operation with argument $y$. Note that a linearization point $lin(deq,x,3)$ is also enabled at this state.

For readability, we define $AbsQ$ as an abstract state machine, which is given in Fig.~\ref{fig:transitions:AbsQ}. The shared state of $AbsQ$ consists of several boolean predicates indicating whether the value added by an enqueue has not been removed yet (the predicate $\mathsf{present})$, whether an enqueue is pending (the predicate $\mathsf{pending}$), the happens-before order (the predicate $\mathsf{before}$), and a function giving the value added by an enqueue (the function $\mathsf{arg}$). In the initial state, the domain of every function (predicate) is empty.
%the states of $AbsQ$ are tuples $\tup{O,<,\ell,rv,cp}$ where $O\subseteq \<Ops>$ is a set of operation identifiers, $<\subseteq O\times O$ is a strict partial order, $\ell: O -> \<Vals>\times\{\tt{PEND,\tt{COMP}}\}$ labels every identifier with a value and a pending/completed flag (the flag is used to track the happens-before order), $rv:\<Ops> ~> \<Vals>$ records the return value of a dequeue fixed at its linearization point ($~>$ denotes a partial function), and $cp:\<Ops> ~> \{A_1,A_2,R_1,R_2,R_3\}$ records the control point of every enqueue ($A_1, A_2$) or dequeue operation ($R_1,R_2,R_3$).
%All the components are $\emptyset$ in the initial state, and the transition relation $->$ is defined in Fig.~\ref{fig:transitions:AbsQ}. The alphabet of $AbsQ$ contains call/return actions and dequeue linearization points, denoted by $lin(deq,d,k)$. $Lin(deq)$ is the set of all actions $lin(deq,d,k)$.
%
The enqueue operations consist of two macro rules: the rule {\tt inv(enq,x,k)} orders the invoked operation after all the completed enqueues present in the current state ({\tt k} is a fresh operation identifier associated to the current operation), and the rule {\tt ret(enq,k)} sets $\mathsf{pending}${\tt (k)} to {\tt false}. % provided that the operation is still present in the current state.
The dequeue operations consist of two rules {\tt inv(deq,k)} and {\tt ret(deq,y,k)} associated to the invocation and respectively, the return action, whose definition is obvious, and the rule {\tt lin(deq,y,k)} corresponding to its linearization point. When a non-empty set of pending enqueues are present, the rule {\tt lin(deq,y,k)} is non-deterministic. It can set the return value {\tt y} either to {\tt EMPTY}, or to a value {\tt d} $\neq$ {\tt EMPTY}, provided that {\tt d} has been added by an enqueue {\tt k1} which is minimal in the current happens-before. In the latter case, $\mathsf{present}${\tt (k1)} is set to {\tt false} to indicate that {\tt k1}'s argument has been removed. When there exists at least one completed enqueue whose value has not been removed, the rule {\tt lin(deq,y,k)} sets {\tt y} to the value added by a minimal enqueue (w.r.t. the current happens-before).

%{\sc call-enq} orders the invoked operation after all the completed enqueues in the current state, and the rules {\sc ret-enq1}/{\sc ret-enq2} flip the corresponding flag from {\tt PEND} to {\tt COMP} provided that the operation is still present in the current state. For dequeue operations, {\sc call-deq} only increments the control point and {\sc ret-deq} checks whether the return value is the same as the one fixed at the linearization point. The linearization point rule {\sc lin-deq1} corresponds to the case of a non-empty queue, showing that $lin(deq,d,k)$ is enabled only if $d$ has been added by an enqueue which is minimal in the current happens-before. When enabled, it removes the enqueue adding $d$ from the state. The linearization point rule {\sc lin-deq2} corresponds to the case of dequeue operations linearized with an {\tt EMPTY} return value.

%\begin{figure} [t]
\begin{figure}[t]
\hspace{-10mm}
\begin{minipage}[t]{7.5cm}
\begin{lstlisting}
void enq($\<Vals>$ v):
  atomic rule inv(enq,v)
  atomic rule lin(enq,v)
  atomic rule ret(enq,v)

\end{lstlisting}
\end{minipage}
\begin{minipage}[t]{7.5cm}
\begin{lstlisting}
$\<Vals>$ deq():
  let v with ...
  atomic rule inv(deq,v)
  atomic rule lin(deq,v)
  atomic rule ret(deq,v)
  return v

\end{lstlisting}
\end{minipage}

\begin{minipage}[t]{7.5cm}
\vspace{-3mm}
\begin{lstlisting}
function $\mathsf{present}$: $\mathbb{O} \to \mathbb{B}$
function $\mathsf{pending}$: $\mathbb{O} \to \mathbb{B}$
function $\mathsf{arg}$: $\mathbb{O} \to \mathbb{V}$
function $\mathsf{before}$: $\mathbb{O} \times \mathbb{O} \to \mathbb{B}$

rule inv(enq,v):
  let k = new $\mathbb{O}$
  $\mathsf{present}$(k) := true
  $\mathsf{pending}$(k) := true
  $\mathsf{arg}$(k) := v
  forall k' with $\mathsf{present}$(k')$\land \neg \mathsf{pending}$(k'):
    $\mathsf{before}$(k',k) := true

rule lin(enq,v):
  pass

rule ret(enq,v):
  let k with $\mathsf{arg}$(k) = v
  $\mathsf{pending}$(k) := false

\end{lstlisting}
\end{minipage}
\begin{minipage}[t]{5cm}
\vspace{-3mm}
\begin{lstlisting}
rule inv(deq,v):
  pass

rule lin(deq,v):
  if v = EMPTY:
    forall k with $\mathsf{present}$(k):
      assert $\mathsf{pending}$(k)
  else:
    assert $\mathsf{present}$(k)
    let k with $\mathsf{arg}$(k) = v
    forall k' with $\mathsf{present}$(k'):
      assert $\lnot\mathsf{before}$(k', k)
    $\mathsf{present}$(k) := false

rule ret(deq,v):
  pass

\end{lstlisting}
\end{minipage}
%	\caption{The abstract queue implementation $AbsQ$.}
%	\label{fig:signatures}
%\end{figure}


%{\scriptsize
%  \centering
%  \begin{mathpar}
%    \inferrule[call-enq]{
%      k\not\in dom(cp) \\
%      d\neq {\tt EMPTY}
%    }{
%      O,<,\ell,rv,cp
%      \xrightarrow{inv(enq,d,k)}
%      %O\cup\{k\},<\cup \{(k',k): \ell_2(k')={\tt COMP}\},\ell[k\mapsto (d,{\tt PEND})],rv,cp[k\mapsto 1]
%      O\cup\{k\},<\cup\ {\tt COMP}(O)\times\{k\},\ell[k\mapsto (d,{\tt PEND})],rv,cp[k\mapsto A_1]
%    }\hspace{5mm}
%
%    \inferrule[call-deq]{
%      k\not\in dom(cp) \\
%    }{
%      O,<,\ell,rv,cp
%      \xrightarrow{inv(deq,k)}
%      O,<,\ell,rv,cp[k\mapsto R_1]
%    }\hspace{5mm}
%    \inferrule[ret-deq]{
%       cp(k) = R_2 \\
%       rv(k)=d
%    }{
%      O,<,\ell,rv,cp
%      \xrightarrow{ret(deq,d,k)}
%      O,<,\ell,rv,cp[k\mapsto R_3]
%    }\hspace{5mm}
%
%    \inferrule[ret-enq1]{
%      cp(k) = A_1 \\
%      k \in O \\
%      \ell(k) = (d,{\tt PEND})
%    }{
%      O,<,\ell,rv,cp
%      \xrightarrow{ret(enq,k)}
%      O,<,\ell[k\mapsto (d,{\tt COMP})],rv,cp[k\mapsto A_2]
%    }\hspace{5mm}
%    \inferrule[ret-enq2]{
%      cp(k) = A_1 \\
%      k \not\in O
%    }{
%      O,<,\ell,rv,cp
%      \xrightarrow{ret(enq,k)}
%      O,<,\ell,rv,cp[k\mapsto A_2]
%    }\hspace{5mm}
%
%    \inferrule[lin-deq1]{
%       cp(k) = R_1 \\
%       d\neq{\tt EMPTY} \\
%       k'\in min(O) \\
%       \ell_1(k')=d
%    }{
%      O,<,\ell,rv,cp
%      \xrightarrow{lin(deq,d,k)}
%      O\setminus \{k'\},<\uparrow k',\ell,rv[k\mapsto d],cp[k\mapsto R_2]
%    }\hspace{5mm}
%    \inferrule[lin-deq2]{
%       cp(k) = R_1 \\
%       \forall o\in O.\ \ell_2(o)={\tt PEND}
%    }{
%      O,<,\ell,rv,cp
%      \xrightarrow{lin(deq,{\tt EMPTY},k)}
%      O,<,\ell,rv[k\mapsto {\tt EMPTY}],cp[k\mapsto R_2]
%    }\hspace{5mm}
%      \end{mathpar}
%  }
 \vspace{-4mm}
  \caption{The abstract queue implementations $AbsQ$. Each macro rule is executed atomically in one step (indicated by the keyword {\ttfamily \bfseries atomic}).
%  transition relation of $AbsQ$. We use the following notations: $\ell_i(k)$ denotes the projection of $\ell(k)$ over the $i$-th component, for each $i\in\{1,2\}$, ${\tt COMP}(O)=\{k\in O: \ell_2(k)={\tt COMP}\}$, $\mathit{f}[x\mapsto y]$ is the function $g$ such that $g(z)=f(z)$ for all $z\neq x$ in the domain of $f$, and $g(x)=y$, $min(O)$ is the set of elements of $O$ which are minimal in the order relation $<$, and $<\uparrow k$ denotes the relation $<$ where all the pairs containing $k$ have been removed.
  %\textcolor{red}{Call-Enq must have $d!= \texttt{EMPTY}$ as a premise. Also lin deq returning empty must be changed as before.}
  }
  \label{fig:transitions:AbsQ}
\vspace{-6mm}
\end{figure}

% Let $AbsQ_0$ denote this implementation (formally defined in Appendix~\ref{app:absImplQueue}).
The following result states that the library $AbsQ$ has exactly the same set of histories as the standard abstract library $AbsQ_0$. % (see Appendix~\ref{app:absImplQueue} for a proof).

\vspace{-1.5mm}
\begin{theorem}\label{th:absImplQueue}
$AbsQ$ is a refinement of $AbsQ_0$ and vice-versa.
\vspace{-2mm}
\end{theorem}

The abstract state machine in Fig.~\ref{fig:transitions:AbsQ} defines an LTS over the alphabet $C\cup R\cup Lin(deq)$. The transitions corresponding to the rule {\tt lin(deq,y,k)} are labeled by  actions $lin(deq,y,k)$. Also, we assume that the transition corresponding to the invocation of a method is done atomically with the first step of the same invocation (the macro rule {\tt inv(...)}), and similarly for transitions corresponding to returning from a method, they are done atomically with the last step (the macro rule {\tt ret(...)}). They are labeled as expected with call/return actions (borrowing the operation identifier which occurs as argument of the macro rules). It can be easily proved that this assumption doesn't modify the set of traces projected on $C\cup R\cup Lin(deq)$ (compared to an LTS where these steps are not atomic).

%~\footnote{Without this assumption, while proving forward simulations, a call/return action in the concrete implementation is simulated by two steps of the abstract implementation, the call/return action together with the first/last macro rule.}.


A trace of a queue implementation is called \emph{$Lin(deq)$-complete} when every completed dequeue has a linearization point, i.e., each return action $ret(deq,d,k)$ is preceded by an action $lin(deq,d,k)$. A queue implementation $L$ over alphabet $\Sigma$, such that $C\cup R\cup Lin(deq)\subseteq \Sigma$, is called \emph{with fixed dequeue linearization points} when every trace $@t\in Tr(L)$ is $Lin(deq)$-complete.

%TODO NEEDS DATA INDEPENDENCE FOR THE LINEARIZATION POINT TRANSITIONS TO BE DETERMINISTIC

The following result shows that $C\cup R\cup Lin(deq)$-forward simulations are a sound and complete proof method for showing the correctness of a queue implementation with fixed dequeue linearization points (up to the correctness of the linearization points). It is obtained from Theorem~\ref{th:absImplQueue} and Theorem~\ref{th:forSim} using the fact that the alphabet of $AbsQ$ is exactly $C\cup R\cup Lin(deq)$ and $AbsQ$ is deterministic. The determinism of $AbsQ$ relies on the assumption that every value is added at most once. Without this assumption, $AbsQ$ may reach a state with two enqueues adding the same value being both minimal in the happens-before. A transition corresponding to the linearization point of a dequeue from this state can remove any of these two enqueues leading to two different states. Therefore, $AbsQ$ becomes non-deterministic. Note that this is independent of the fact that $AbsQ$ manipulates  operation identifiers.

\vspace{-1.5mm}
\begin{corollary}
A queue implementation $L$ with fixed dequeue linearization points is a $C\cup R\cup Lin(deq)$-refinement of $AbsQ_0$ if{f} there exists a $C\cup R\cup Lin(deq)$-forward simulation from $L$ to $AbsQ$.
\vspace{-1.5mm}
\end{corollary}

\vspace{-4.5mm}
\subsection{A Correctness Proof For Herlihy\&Wing Queue}\label{ssec:HerlihyWing}
\vspace{-1mm}
We describe a forward simulation $F_1$ from $\mathit{HWQ}$ to $AbsQ$. The description of $\mathit{HWQ}$ in Fig.~\ref{fig:HerlihyWing} defines an LTS whose states contain the shared array ${\tt items}$ and the shared counter ${\tt back}$ together with a valuation for the local variables ${\tt i}$, ${\tt x}$, and ${\tt range}$, and the control location of each operation. A transition is either a call or a return action, or a statement in one of the two methods ${\tt enq}$ or ${\tt deq}$.

An enqueue operation in a $\mathit{HWQ}$ state is pending and present if and only if its argument is stored in the array ${\tt items}$ or it has not yet written to the array  ${\tt items}$. Those operations should be $\mathsf{pending}$ and $\mathsf{present}$ in the related $AbsQ$ states. In addition to this, the forward simulation $F_1$ imposes following restrictions:
%A $\mathit{HWQ}$ state is related by $F_1$ to an $AbsQ$ state where $\mathsf{present}$ is true for all the enqueues whose argument is  stored in the array ${\tt items}$, and all the pending enqueues that have not yet written to the array ${\tt items}$ (and only for these enqueues). The order relation $\mathsf{before}$ between these enqueues is defined as follows:
\vspace{-2mm}
\begin{itemize}
	\item[(a)] pending and present enqueues are maximal, i.e., for every two $\mathsf{present}$ enqueues $k$ and $k'$ such that $k'$ is $\mathsf{pending}$, we have that $\neg \mathsf{before}(k',k)$, % $k'\not< k$,
	\item[(b)] $\mathsf{before}$ is consistent with the order in which positions of ${\tt items}$ have been reserved, i.e., for every two present enqueues $k$ and $k'$ such that ${\tt i}(k) < {\tt i}(k')$, we have that $\neg \mathsf{before}(k',k)$, %$k' \not< k$,
	\item[(c)] an enqueue which has reserved a position $i$ %and executed only the first statement
	can't be ordered before another enqueue that has reserved a position $j \geq i$ when the position $i$ has been ``observed'' by a non-linearized dequeue that may ``observe'' $j$ in the current array traversal, i.e., for every two present enqueues $k$ and $k'$, and a dequeue $k_d$, such that

	\vspace{-2mm}
	\noindent
	{\small
	\begin{align}
	\hspace{-8mm}
	{\tt canRemove}(k_d,k) \land (i(k') < i(k_d) \lor (i(k')=i(k_d) \land {\tt afterSwapNull}(k_d)))
\label{eq:inst}
	\end{align}}

	\vspace{-6mm}
	\noindent
	we have that $\neg \mathsf{before}(k,k')$. The predicate ${\tt canRemove}(k_d,k)$ holds when $k_d$ has currently visited a {\tt null} item in {\tt items} and the present enqueue index $i(k)$ is in the range of $(k_d)$ i.e., $\mathsf{present}(k) \land (x(k_d) = {\tt null} \land i(k) < {\tt range}(k_d) \land i(k_d) < i(k)) \lor (i(k_d) = i(k) \land {\tt beforeSwap}(k_d) \land {\tt items}[i(k)] != {\tt null})$. The predicates ${\tt afterSwapNull}(k_d)$ (resp., ${\tt beforeSwap}(k_d)$) holds when the dequeue $k_d$ is at the control point after a ${\tt swap}$ returning ${\tt null}$ (resp., before a ${\tt swap}$).
\vspace{-2mm}
\end{itemize}

\noindent
We have that $\mathsf{pending}(k)$ is ${\tt true}$ whenever $k$ is a pending enqueue, and $\mathsf{arg}(k)=d$ whenever the argument of the enqueue $k$ is $d$.
%An enqueue is labeled by $(d,{\tt PEND})$ where $d$ is the input value if it's pending and by  $(d,{\tt COMP})$, otherwise.
Also, for every dequeue operation $k$ such that ${\tt x}(k)=d\neq {\tt null}$, we have that ${\tt y}(k)=d$ (recall that ${\tt y}$ is a local variable of the dequeue method in $AbsQ$).

The restrictions of $F_1$ aim to ensure two important points:  (i) identifier of a pending and present enqueue method in the $\mathit{HWQ}$ state should be maximal according to $\mathsf{before}$ predicate in the related $AbsQ$ state and (ii) identifiera of a present enqueue method in $\mathit{HWQ}$ state of which data value is about to be removed by a dequeue operation should be minimal in the related $AbsQ$ state.

The first point is ensured by the restriction (a). Present and pending enqueue identifiers in the $\mathit{HWQ}$ state are $\mathsf{pending}$ in the related $AbsQ$ state and $\mathsf{pending}$ enqueues are maximal in a valid $AbsQ$ state.

Restrictions (b) and (c) ensure the point (ii). Consider a present enqueue $k$ that inserted its argument to ${\tt items}$ and  there exists a pending dequeue $k_d$ such that ${\tt canRemove}(k_d, k)$ and $k_d$ is just before its swap action at the reserved position of $k$ i.e., $i(k_d) = i(k)$. A pending enqueue operation cannot be ordered $\mathsf{before}$ $k$ since pending enqueues are maximal by (a). Regarding the completed and present enqueues $k'$, we consider two cases: $i(k') > i(k)$ and $i(k') < i(k)$. For the former case, the restriction (b) ensures $\neg \mathsf{before}(k',k)$ and for the latter case the restriction (c) ensures $\neg \mathsf{before}(k',k)$. Consequently, $k$ is a minimal element in $\mathsf{before}$ relation just before $k_d$ removes its data value.

Next, we show that $F_1$ is indeed a $C\cup R\cup Lin(deq)$-forward simulation. Let $s$ and $t$ be states of $\mathit{HWQ}$ and $AbsQ$, respectively, such that $(s,t)\in F_1$.
We omit discussing the trivial case of transitions labeled by call and return actions which are simulated by similar transitions of $AbsQ$ (for the return a dequeue operation $k$, we use the equality between the local variable ${\tt x}(k)$ in $s$ and the component $rv(k)$ in $t$).
%\textcolor{red}{ I think it is good to mention again that call/return actions in HWQ correspond to the same call/return actions in AbsQ (without any other internal action). I also think that invoke enqueue operation is non-trivial. Preservation of the strict partial order and all of the above items a, b and c needs to be rechecked.}

%\setlength{\parskip}{0pt}
We show that each internal step of an enqueue or dequeue, except the execution of {\tt swap} returning a non-null value in dequeue (which represents its linearization point), is simulated by an \emph{empty} sequence of $AbsQ$ transitions, i.e., for every state $s'$ obtained through one of these steps, if $(s,t)\in F_1$, then $(s',t)\in F_1$ for each $AbsQ$ state $t$.
Essentially, this consists in proving the following property, called \emph{monotonicity}: the set of possible $\mathsf{before}$ relations associated by $F_1$ to $s'$ doesn't exclude any order $\mathsf{before}$ associated to $s$.
%Essentially, this boils down to showing that the constraints over $<$ in the definition of $\mathit{fs}$ are an invariant for these steps.

Concerning enqueue rules, let $s'$ be the state obtained from $s$ when a pending enqueue $k$ reserves an array position. This enqueue must be maximal in both $t$ and any state $t'$ related to $s'$ (since it's pending). Moreover, there is no dequeue that can ``observe'' this position before restarting the array traversal. Therefore, item (c) in the definition of $F_1$ doesn't constrain the order between $k$ and some other enqueue neither in $s$ nor in $s'$. Since this transition doesn't affect the constraints on the order between enqueues different from $k$ (their local variables remain unchanged), monotonicity holds. This property is trivially satisfied by the second step of enqueue which doesn't affect {\tt i}.

To prove monotonicity in the case of dequeue internal steps different from its linearization point, it is important to track the non-trivial instantiations of item (c) in the definition of $\mathsf{before}$ over the two states $s$ and $s'$, i.e., the triples $(k,k',k_d)$ for which (\ref{eq:inst}) holds. Instantiations that are enabled only in $s'$ may in principle lead to a violation of monotonicity (since they restrict the orders $\mathsf{before}$ associated to $s'$). For the two steps that begin an array traversal, i.e., reading the index of the last used position and setting {\tt i} to $0$, there exist no  such new instantiations in $s'$ because the value of {\tt i} is either not set or $0$. % (it is trivial to notice that applying these steps doesn't disable such instantiations that were possible in $s$).
%The same holds for the step incrementing the iterator {\tt i}.
%
%The execution of {\tt swap} returning {\tt null} may introduce one new non-trivial instantiation $(k,k',k_d)$ of item (c).
%We write ${\tt i}_s(k)$ to refer to the value of the variable {\tt i} of operation $k$ in state $s$. Assume that indeed, there exist two enqueue operations $k$ and $k'$ such that ${\tt i}_{s'}(k) < {\tt i}_{s'}(k_d) \leq {\tt i}_{s'}(k')$, ${\tt x}_{s'}(k_d)={\tt null}$, ${\tt i}_{s'}(k') \leq {\tt range}_{s'}(k_d)$ TODO SWAP. Since {\tt swap} returnes {\tt null}, the position ${\tt i}_{s'}(k_d)$
%
%
% and ${\tt i}_{s}(k) = {\tt i}_{s'}(k_d)$. The latter constraint guarantees that this instantiation is not enabled in state $s$. The increment of {\tt i} being enabled, implies that
%
The same is true for the increment of {\tt i} in a dequeue $k_d$ since the predicate ${\tt afterSwapNull}(k_d)$ holds in state $s$.
The execution of {\tt swap} returning {\tt null} in a dequeue $k_d$ enables new instantiations $(k,k',k_d)$ in $s'$, thus adding potentially new constraints $\neg \mathsf{before}(k,k')$. We show that these instantiations are however vacuous because $k$ must be pending in $s$ and thus maximal in every order $\mathsf{before}$ associated by $F_1$ to $s$.
Let $k$ and $k'$ be two enqueues such that together with the dequeue $k_d$ they satisfy the property (\ref{eq:inst}) in $s'$ but not in $s$.
We write ${\tt i}_s(k)$ for the value of the variable {\tt i} of operation $k$ in state $s$.
We have that ${\tt i}_{s'}(k) = {\tt i}_{s'}(k_d) \leq {\tt i}_{s'}(k')$ and ${\tt items}[{\tt i}_{s'}(k_d)]={\tt null}$. The latter implies that the enqueue $k$ didn't execute
the second statement (since the position it reserved is still {\tt null}) and it is pending in $s'$. The step that swaps the null item does not modify anything except the control point of $k_d$ that makes ${\tt afterSwapNull}(k_d)$ true in $s'$ . Hence, $i_s(k) = i_s(k_d)\leq i_s(k')$ and ${\tt items}[i_s(k_d)] = {\tt null}$ is also true. Therefore, $k$ is pending in $s$ and maximal. Hence, $\mathsf{before}(k,k')$ is not true in both $s$ and $s'$.

Finally, we show that the linearization point of a dequeue $k$ of $\mathit{HWQ}$, i.e., an execution of {\tt swap} returning a non-null value $d$, from state $s$ and leading to a state $s'$ is simulated by a transition labeled by $lin(deq,d,k)$ of $AbsQ$ from state $t$. By the definition of $\mathit{HWQ}$, there is a unique enqueue $k_e$ which filled the position updated by $k$, i.e., ${\tt i}_s(k_e)=i_s(k)$ and ${\tt x}_{s'}(k)={\tt x}_s(k_e)$.

We show that $k_e$ is minimal in the order $\mathsf{before}$ of $t$ which implies that $k_e$ could be chosen by $lin(deq,d,k)$ step applied on $t$. As explained previously, instantiating item (c) in the definition of $\mathsf{before}$ with $k'=k_e$ and $k_d=k$; and instantiating item (b) with $k=k_e$, we ensures the minimality of $k_e$. Moreover, the state $t'$ obtained from $t$ through a $lin(deq,d,k)$ transition is related to $s'$ because the value added by $k_e$ is not anymore present in {\tt items} and $\mathsf{present}(k_e)$ doesn't hold in $t'$.
%Thus, instantiating item (c) in the definition of $\mathsf{before}$ with $k'=k_e$ and $k_d=k$ we get that every enqueue that reserved a position smaller than the one of $k_e$ can't be ordered before $k_e$ in the order $\mathsf{before}$. Also, applying item (b) with $k=k_e$ we get the same for every enqueue that reserved a bigger position. An enqueue that didn't reserve a position is by definition maximal in $<$ and therefore, not a predecessor of $k_e$. Then, the state $t'$ obtained from $t$ through a $lin(deq,d,k)$ transition is related to $s'$ because (1) the value added by $k_e$ is not anymore present in {\tt items} which implies that $k_e$ doesn't occur in any $AbsQ$ state related to $s'$, and (2) the value of ${\tt x}(k)$ is set to $d\neq {\tt null}$ which implies that $rv(k)$ is set to $d$ in every $AbsQ$ state related to $s'$.
%
%\vfill
%

%\section{Existence of Forward Simulations for Stack Implementations that have Fixed Pop Linearization Points}
%!TEX root = draft.tex
\vspace{-3.5mm}
\section{Stacks With Fixed Pop Commit Points}\label{sec:stacks}
\vspace{-1.5mm}
While the abstract queue in Section~\ref{sec:queues} can be adapted to stacks (the linearization point $lin(pop,d,k)$ with $d\neq{\tt EMPTY}$ is enabled when $k$ is added by a push which is maximal in the happens-before order stored in the state), it can't simulate (through forward simulations) existing stack implementations like the Time-Stamped Stack~\cite{DBLP:conf/popl/DoddsHK15} ($\mathit{TSS}$, for short) where the linearization points of the pop operations are not fixed. Exploiting particular properties of the stack semantics, we refine the ideas used in $AbsQ$ and define 
a new abstract implementation for stacks, denoted as $AbsQ$, which is able to simulate such implementations. Forward simulations to $AbsS$ are complete for proving the correctness of stack implementations provided that the point in time where the return value of a pop operation is determined, called \emph{commit point}, corresponds to a fixed action.
% provided that the point in time where the return value of a pop operation is determined corresponds to a fixed action. To demonstrate the use of this abstract implementation we provide a correctness proof for a simplified version of Time-Stamped Stack that preserves its most complex behaviors.

\vspace{-3.5mm}
\subsection{Pop Methods With Fixed Commit Points}
\vspace{-1mm}
\begin{wrapfigure}{l}{5.2cm}
\vspace{-10.5mm}
\begin{lstlisting}
struct Node{
  int data;
  int ts;
  Node* next;
  boolean taken;
};
Node* pools[maxThreads];
int TS = 0;   

void push(int x) {
  Node* n = new Node(x,MAX_INT,
                        null,false);
  n->next = pools[myTID];
  pools[myTID] = n;
  int i = TS++;
  n->ts = i;
}
int pop() {
 boolean success = false;
 int maxTS = -1;
 Node* youngest = null;
 while ( !success ) {
   maxTS = -1; youngest = null;
   for(int i=0; i<maxThreads; i++){
     Node* n = pools[i];
     while (n->taken && n->next != n)
       n = n->next;
     if(maxTS < n->ts) {
       maxTS = n->ts; youngest = n;
     }
   }
   if (youngest != null)
     success=CAS(youngest->taken,
                       false,true);
 }
 return youngest->data;
}
\end{lstlisting}
\vspace{-6mm}
\caption{Time-Stamped Stack.} %We assume that every statement is atomic.}
\label{fig:TimeStamped}
\vspace{-7mm}
\end{wrapfigure}
We explain the meaning of the commit points on a simplified version of the Time-Stamped Stack~\cite{DBLP:conf/popl/DoddsHK15} ($\mathit{TSS}$, for short) given in Figure~\ref{fig:TimeStamped}. This  implementation maintains an array of singly-linked lists, one for each thread, where list nodes contain a data value (field {\tt data}), a timestamp (field {\tt ts}), the next pointer (field {\tt next}), and a boolean flag indicating whether the node represents a value removed from the stack (field {\tt taken}). Initially, each list contains a sentinel dummy node pointing to itself with timestamp $-1$ and the flag {\tt taken} set to {\tt false}.

Pushing a value to the stack proceeds in several steps: adding a node with maximal timestamp in the list associated to the thread executing the push (given by the special variable {\tt myTID}), asking for a new timestamp (given by the shared variable {\tt TS}), and updating the timestamp of the added node. Popping a value from the stack consists in traversing all the lists, finding the first element which doesn't represent a removed value (i.e., {\tt taken} is {\tt false}) in each list, and selecting the element with the maximal timestamp. A compare-and-swap (CAS) is used to set the {\tt taken} flag of this element to {\tt true}. The procedure restarts if the CAS fails.

\begin{figure}[t]
\centering
\includegraphics[width=11.5cm]{fig-tss.pdf}
\vspace{-4mm}
\caption{An execution of $\mathit{TSS}$. An operation is pictured by a line delimited by two circles denoting the call and respectively, the return action. Pop operations with identifier $k$ and removing value $d$ are labeled $pop(d,k)$. Their representation includes another circle that stands for a successful CAS which is their commit point. The library state after an execution prefix delimited at the right by a dotted line is pictured in the bottom part (the picture immediately to the left of the dotted line). A pair $(d,t)$ represents a list node with ${\tt data}=d$ and ${\tt ts}=t$, and ${\tt i}(1)$ denotes the value of {\tt i} in the pop with identifier 1. We omit the nodes where the field {\tt taken} is {\tt true}.}
\label{fig:commit}
\vspace{-6mm}
\end{figure}


The push operations don't have a fixed linearization point because adding a node to a list and updating its timestamp are not executed in a single atomic step. The nodes can be added in an order which is not consistent with the order between the timestamps assigned later in the execution. Also, the value added by a push that just added an element to a list can be popped before the value added by a completed push (since it has a maximal timestamp). The same holds for pop operations: The only reasonable choice for a linearization point is a successful CAS (that results in updating the field {\tt taken}). Fig.~\ref{fig:commit} pictures an execution showing that this action doesn't correspond to a linearization point, i.e., an execution for which the pop operations in every correct linearization are not ordered according to the order between successful CASs. In every correct linearization of that execution, the pop operation removing $x$ is ordered before the one removing $z$ although they perform a successful CAS in the opposite order.

An interesting property of the successful CASs in pop operations is that they fix the return value, i.e., the return value is {\tt youngest->data} where {\tt youngest} is the node updated by the CAS. We call such actions \emph{commit points}. More generally, commit points are actions that access shared variables, from which every control-flow path leads to the return control point and contains no more accesses to the shared memory (i.e., after a commit point, the return value is computed using only local variables).

%The complete version of $\mathit{TSS}$~\cite{DBLP:conf/popl/DoddsHK15} contains also a mechanism for elimination (a pop can remove a value without traversing all the lists if it has been added by a concurrent push) and emptiness checking. In both cases, the commit points can be identified in the code and correspond to certain boolean conditions evaluated to {\tt true}.

%TODO WHAT CAN WE SAY ABOUT COMMIT POINTS IN OTHER IMPLEMENTATIONS

%Usually, a stack implementation stores the pushed values into a data structure, e.g., a singly-linked list, and a pop operation contains an action (typically, a compare-and-swap) that removes an element from this data structure (or sets a flag associated to this element to a ``deleted'' state). Such an action represents a commit point since the pop operation will return the value stored in this element. 


%TODO GIVE AN EXAMPLE TO SHOW THE ISSUE WITH POP LINEARIZATION POINTS, AND TO SHOW THAT COMMIT POINTS ARE RIGHT LIMITS FOR THE INTERVAL OF A LIN POINT, I.E., A POP WITH A LATER COMMIT POINT CAN BE LINEARIZED BEFORE. NEED ONLY 2 POPS IN AN EXECUTION 

When the commit points of pop operations are fixed to particular implementation actions (e.g., a successful CAS) we assume that the library is an LTS over an alphabet that contains actions $com(pop,d,k)$ with $d\in\<Vals>$ and $k\in\<Ops>$ (denoting the commit point of the pop with identifier $k$ and returning $d$). Let $Com(pop)$ be the set of such actions.
%We consider a set of actions $Com(pop)=\set{com(pop,d,k):d\in\<Vals>, k\in\<Ops>}$, called \emph{commit points}, representing actions of a pop operation where its return value is determined. 
%TODO DESCRIBE THE TIME-STAMPED STACK, EXPLAIN THAT NO METHOD HAS A FIXED LIN POINT, GIVE THE COMMIT POINTS, SHOW THAT COMMIT POINTS ARE NOT LIN POINTS
%
%COMMIT POINTS ARE LINEARIZATION POINTS WHEN THE LIBRARY IS INTERPRETED AS A MULTISET

\vspace{-3mm}
\subsection{Abstract stack implementation}
\vspace{-1mm}
We define an abstract stack $AbsS$ over alphabet $C\cup R\cup Com(pop)$ that essentially, similarly to $AbsQ$, maintains the happens-before order of the pushes whose value has not been yet removed. Pops are treated differently since the commit points are not necessarily linearization points, intuitively, a pop can be linearized before its commit. Each pop operation starts by taking a snapshot of the greatest completed push operations in the happens-before order, and continuously tracks the push operations which are overlapping with it. The commit point $com(pop,d,k)$ with $d\neq {\tt EMPTY}$ is enabled only if $d$ was added by one of the push operations in the initial snapshot, or by a push happening earlier when all the values from the initial snapshot have been removed, or by one of the push operations that overlaps with pop $k$. The commit point $com(pop,{\tt EMPTY},k)$ is enabled only if all the values added by push operations ending before $k$ started have been removed. The effect of the commit points is explained below through examples.
\vspace{-.4mm}

\begin{wrapfigure}{l}{6.8cm}
\vspace{-6mm}
\includegraphics[width=7cm]{fig-stack.pdf}
%
%\vspace{2mm}
%\includegraphics[width=7cm]{fig-queue2.pdf}
\vspace{-8mm}
\caption{Simulating stack histories with $AbsS$.}
\label{fig:stackSim}
\vspace{-6.5mm}
\end{wrapfigure}
Figure~\ref{fig:stackSim} pictures two executions of $AbsS$ for two extended histories (that include pop commit points). For readability, we give the state of $AbsS$ only after several execution prefixes delimited at the right by a dotted line. We focus on pop operations -- the effect of push calls and returns is similar to enqueue calls and returns in $AbsQ$. Let us first consider the history on the top part. The first state we give is reached after the call of pop with identifier $3$. This shows the effect of a pop invocation: the greatest completed pushes according to the current happens-before (here, the push with identifier $1$) are marked as $be(3)$ (from ``before'' operation 3), and the pending pushes are marked as $ov(3)$ (from ``overlapping'' with operation 3). As a side remark, any other push operation that starts after pop $3$ would be also marked as $ov(3)$.
The commit point $com(pop,x,3)$ (pictured with a red circle) is enabled because $x$ was added by a push marked as $be(3)$. The effect of the commit point is that push $1$ is removed from the state (the execution on the bottom shows a more complicated case). For the second pop, the commit point $com(pop,y,4)$ is enabled because $y$ was added by a push marked as $ov(4)$. The execution on the bottom shows an example where the marking $be(k)$ for some pop $k$ is updated at commit points. The pushes $3$ and $4$ are marked as $be(5)$ and $be(6)$
when the pops $5$ and $6$ start. Then, $com(pop,t,5)$ is enabled since $t$ was added by $push(t,4)$ which is marked as $be(5)$. Besides removing $push(t,4)$, the commit point produces a state where a pop committing later, e.g., pop $6$, can remove $y$ which was added by a predecessor of $push(t,4)$ in the happens-before ($y$ could become the top of the stack when $t$ is removed). This history is valid because $push(y,2)$ can be linearized after $push(x,1)$ and $push(z,3)$. Thus, push 2, a predecessor of the push which is removed, is marked as $be(6)$. Push $1$ which is also a predecessor of the removed push is not marked as $be(6)$ because it happens before another push, i.e., push 3, which is already marked as $be(6)$ (the value added by push 3 should be removed before the value added by push 1 could become the top of the stack).

Formally, the states of $AbsS$ are tuples $\tup{O,<,\ell,rv,cp,be,ov}$ where $<$ is a strict partial order over the set $O$ of operation identifiers, $\ell: O -> \<Vals>\times\{\tt{PEND,\tt{COMP}}\}$ labels every identifier in $O$ with a value and a pending/completed flag, $rv:\<Ops> ~> \<Vals>$ records the return value of a pending pop fixed at its commit point, $cp:\<Ops> ~> \{A_1,A_2,R_1,R_2,R_3\}$ records the control point of every push ($A_1, A_2$) or pop operation ($R_1,R_2,R_3$), $be:\<Ops> ~> 2^O$ records the greatest completed push operations before a pop started or happening earlier provided that the values of all the push happening later have been removed, and $ov: \<Ops> ~> 2^O$ records push operations overlapping with a pop.
%The initial state has all these components set to $\emptyset$ and the transition relation $->$ is defined in Figure~\ref{fig:transitions:AbsS}.
All the components are $\emptyset$ in the initial state, and the transition relation $->$ is defined in Fig.~\ref{fig:transitions:AbsS}.

The transition rules which don't correspond to commit point actions are similar to those for $AbsQ$. The rule {\sc com-pop1} for $com(pop,d,k)$ is enabled only if there exists a push $k'$ which added value $d$ and which belongs to $be(k)$ or $ov(k)$. When enabled, the push $k'$ is removed from the set $O$ (and the order $<$) and for every other pop $k_1$ such that $k'$ belongs to $be(k_1)$, $k'$ is replaced in $be(k_1)$ by its predecessors which are followed exclusively by pushes overlapping with $k_1$ (these predecessors become maximal closed pushes once $k'$ is removed). Also, $rv(k)$ is set to $d$. The rule {\sc com-pop1} for $com(pop,{\tt EMPTY},k)$ is enabled only if $be(k)$ is empty (i.e., all the values added by pushes ending before $k$, if any, have been removed). Then, $rv(k)$ is set to ${\tt EMPTY}$.

\begin{figure} [t]
\vspace{-2mm}
{\scriptsize
  \centering
  \begin{mathpar}
    \inferrule[call-push]{
      k\not\in dom(cp) \\ 
      d\neq {\tt EMPTY} \\
      \forall k'.\ ov'(k') = ov(k')\cup \{k\}
    }{
      O,<,\ell,rv,cp,be,ov 
      \xrightarrow{inv(push,d,k)} 
      O\cup\{k\},<\cup\ {\tt COMP}(O)\times\{k\},\ell[k\mapsto (d,{\tt PEND})],rv,cp[k\mapsto A_1],be,ov'
    }\hspace{5mm}

    \vspace{-1mm}
    \inferrule[call-pop]{
      k\not\in dom(cp) %\\ k\not\in dom(cp) \\ k\not\in dom(cp)
    }{
      O,<,\ell,rv,cp,be,ov
      \xrightarrow{inv(pop,k)} 
      O,<,\ell,rv,cp[k\mapsto R_1],be[k\mapsto maxCo(O)],ov[k\mapsto {\tt PEND}(O)]
    }\hspace{5mm}
    
    \vspace{-1mm}
    \inferrule[ret-pop]{
       cp(k) = R_2 \\
       rv(k)=d  
    }{
      O,<,\ell,rv,cp,be,ov
      \xrightarrow{ret(pop,d,k)}
      O,<,\ell,rv,cp[k\mapsto R_3],be,ov
    }\hspace{5mm}

    \vspace{-1mm}
    \inferrule[ret-push1]{
      cp(k) = A_1 \\
      k \in O \\
      \ell(k) = (d,{\tt PEND}) 
    }{
      O,<,\ell,rv,cp,be,ov
      \xrightarrow{ret(push,k)}
      O,<,\ell[k\mapsto (d,{\tt COMP})],rv,cp[k\mapsto A_2],be,ov
    }\hspace{5mm}
    \inferrule[ret-push2]{
      cp(k) = A_1 \\
      k \not\in O 
    }{
      O,<,\ell,rv,cp,be,ov
      \xrightarrow{ret(push,k)}
      O,<,\ell,rv,cp[k\mapsto A_2],be,ov
    }\hspace{5mm}

    \inferrule[com-pop1]{
       cp(k) = R_1 \\
       d\neq{\tt EMPTY} \\
       k'\in be(k)\cup ov(k) \\
        \ell_1(k')=d \\
       \forall k_1.\ k'\not\in be(k_1) \Rightarrow be'(k_1)=be(k_1) \\     \\
       \forall k_1.\ k'\in be(k_1) \Rightarrow be'(k_1)=(be(k_1)\setminus\{k'\})\cup \{k_2: k_2\in pred_{<}(k') \land \forall k_3. (k_2\in pred_{<}(k_3) \land k_3\neq k') => k_3\in ov(k_1)\} \\
    }{
      O,<,\ell,rv,cp,be,ov
      \xrightarrow{com(pop,d,k)} \\
      O\setminus \{k'\},<\uparrow k',\ell,rv[k\mapsto d],cp[k\mapsto R_2],be',ov %be[\forall k''.\ k'\in be(k'') \Rightarrow k'' \mapsto (be(k'')\cup pred_{<}(k'))\setminus\{k'\}]
    }\hspace{5mm}

    %\vspace{-1mm}
    \inferrule[com-pop2]{
       cp(k) = R_1 \\
       be(k)=\emptyset
    }{
      O,<,\ell,rv,cp,be,ov
      \xrightarrow{com(pop,{\tt EMPTY},k)}
      O,<,\ell,rv[k\mapsto {\tt EMPTY}],cp[k\mapsto R_2],be,ov
    }\hspace{5mm}

    
      \end{mathpar}
  }
 \vspace{-6mm}
  \caption{The transition relation of $AbsQ$. We use the following notions: $maxCo(O)$ is the set of greatest operations in $O$ (w.r.t. $<$) which are completed, i.e., $maxCo(O)=\set{k\in O: \ell_2(k)={\tt COMP}, \forall k'\in O.\ k' < k \vee \ell_2(k')={\tt PEND}}$, ${\tt PEND}(O)=\{k\in O: \ell_2(k)={\tt PEND}\}$, and $pred_{<}(k')$ is the set of immediate predecessors of $k'$ according to $<$, i.e., $pred_{<}(k')=\set{k\in O: k < k'\land \forall k''\in O.\ k'' > k' \vee k'' < k}$.%\textcolor{red}{call-push should have $d \neq EMPTY$ in the premise. If I understand correctly, there is a problem change in $be$ in lin-pop1 rule. What I understand is we move $k''$ to every predecesoor of $k'$. But we should not move it to predecessor of $k'$ if this predecessor has a successor $m$ such that $m \in be(k'')$. Is this case reflected in your definition. Please read sub-bullet 4 of bullet 4 in the definition of $L_I$ in my document. Lin-pop2 seems ok for me. Last note: Bibliography seems not correct to me.}  $f[\forall k.\ k\mapsto expr(k)]$ is the function $g$ defined by $g(k)=expr(k)$,   $g(k)=f(k)$ for all $k$ such that $guard(k)$ is false, and 
  }
  \label{fig:transitions:AbsS}
\vspace{-5mm}
\end{figure}

Let $AbsS_0$ be the standard abstract implementation of a stack (where elements are stored in a sequence; push, resp., pop operations add, resp., remove, an element from the beginning of the sequence in one atomic step). For $\<Methods>=\{push,pop\}$, the alphabet of $AbsS_0$ is $C\cup R\cup Lin$.
The following result states that the library $AbsS$ has exactly the same set of histories as $AbsS_0$ (see Appendix~\ref{app:absImplStack} for a proof).

\vspace{-2mm}
\begin{theorem}\label{th:absImplStack}
$AbsS$ is a refinement of $AbsS_0$ and vice-versa.
\vspace{-2mm}
\end{theorem}

A trace of a stack implementation is called \emph{$Com(pop)$-complete} when every completed pop has a commit point, i.e., each return $ret(pop,d,k)$ is preceded by an action $com(pop,d,k)$. A stack implementation $L$ over $\Sigma$, such that $C\cup R\cup Com(pop)\subseteq \Sigma$, is called \emph{with fixed pop commit points} when every trace $@t\in Tr(L)$ is $Com(pop)$-complete.

%TODO NEEDS DATA INDEPENDENCE FOR THE COMMIT POINT TRANSITIONS TO BE DETERMINISTIC

As a consequence of Theorem~\ref{th:forSim}, $C\cup R\cup Com(pop)$-forward simulations are a sound and complete proof method for showing the correctness of a stack implementation with fixed pop commit points (up to the correctness of the commit points). 

%TODO EXPLAIN THAT THIS IS DIFFERENT W.R.T. QUEUES ($Abs_0$ doesn't have commit points).

\vspace{-1.5mm}
\begin{corollary}
A stack $L$ with fixed pop commit points is a $C\cup R\cup Com(pop)$-refinement of $AbsS$ if{f} there is a $C\cup R\cup Com(pop)$-forward simulation from $L$ to $AbsS$.
\vspace{-1.5mm}
\end{corollary}

Linearization points can also be seen as commit points and thus the following holds.

\vspace{-1.5mm}
\begin{corollary}
A stack implementation $L$ with fixed pop linearization points where transition labels $lin(pop,d,k)$ are substituted with $com(pop,d,k)$ is a $C\cup R\cup Com(pop)$-refinement of $AbsS_0$ if{f} there is a $C\cup R\cup Com(pop)$-forward simulation from $L$ to $AbsS$.
\vspace{-1.5mm}
\end{corollary}


\vspace{-6mm}
\subsection{A Correctness Proof For Time-Stamped Stack}
\vspace{-1mm}
We describe a forward simulation $\mathit{fs}_2$ from $\mathit{TSS}$ to $AbsS$. Except for the constraints on the components $be$ and $ov$ of a $AbsS$ state, it is similar to the simulation $\mathit{fs}_1$ from $\mathit{HWQ}$ to $AbsQ$. Thus, the $AbsS$ states $t=\tup{O,<,\ell,rv,cp,be,ov}$ associated by $\mathit{fs}_2$ to a $\mathit{TSS}$ state $s$ satisfy the following. The set $O$ consists of all the identifiers of pushes in $s$ which didn't added yet a node to {\tt pools} or for which the input is still present in {\tt pools} (i.e., the node created by the push has {\tt taken} set to {\tt false}). A push $k$ is labeled by $(d,{\tt PEND})$ where $d$ is the input value if it's pending and by $(d,{\tt COMP})$, otherwise. 

To describe the order relation $<$ we consider the following notations: ${\tt ts}_s(k)$, resp., ${\tt TID}_s(k)$, denotes the timestamp of the node created by the push $k$ in state $s$ (the {\tt ts} field of this node), resp., the id of the thread executing $k$. By an abuse of terminology, we call ${\tt ts}_s(k)$ the timestamp of $k$ in state $s$.
Also, $k \leadsto_s k'$ when intuitively, a traversal of {\tt pools}  would encounter the node created by $k$ before the one created by $k'$. More precisely, $k \leadsto_s k'$ when ${\tt TID}_s(k) < {\tt TID}_s(k')$, or ${\tt TID}_s(k) = {\tt TID}_s(k')$ and the node created by $k'$ is reachable from the one created by $k$ in the list pointed to by ${\tt pools}[{\tt TID}_s(k)]$.
The order relation $<$ satisfies the following: (1) pending pushes are maximal, (2) $<$ is consistent with the order between node timestamps, i.e., ${\tt ts}_s(k) \leq {\tt ts}_s(k')$ implies $k'\not< k$, and (3) $<$ includes the order between pushes executed in the same thread, i.e., ${\tt TID}_s(k) = {\tt TID}_s(k')$ and ${\tt ts}_s(k) < {\tt ts}_s(k')$ implies $k < k'$.

The components $be$ and $ov$ satisfy the following constraints (their domain is the set of identifiers of pending pops):
\vspace{-2mm}
\begin{itemize}
	\item every pop overlaps with all pending pushes, i.e., $k_1\in O$ is pending implies $k_1\in ov(k)$ for each $k, k_1$
	\item the greatest completed pushes in $<$ are either overlapping with a pop $k$ or they were the greatest completed pushes when $k$ started, i.e., every such push belongs either to $be(k)$ or $ov(k)$ for each $k$
	\item for every push that overlaps with a pop $k$ or was maximal in $<$ when $k$ started, its successors are overlapping with $k$, i.e., $k_1\in be(k)\cup ov(k)$ and $k_1 < k_2$ implies $k_2 \in ov(k)$ for each $k, k_1, k_2$
	\item $be(k)$ and $ov(k)$ don't contain predecessors of pushes from $be(k)$, i.e., $k_1 < k_2$ and $k_2 \in be(k)$ implies $k_1\not\in be(k)\cup ov(k)$ for each $k, k_1, k_2$
	\item immediate predecessors of pushes overlapping with a given pop $k$ are either overlapping with $k$ or they were maximal in $<$ when $k$ started, i.e., $k_2\in pred_{<}(k_1)$ and $k_1\in ov(k)$ implies $k_2\in be(k)\cup ov(k)$ for each $k,k_1,k_2$
	\item a pop $k$ that reached a node with timestamp $\tau$ (its variable {\tt n} points to this node) overlaps with any push that has a timestamp bigger than $\tau$ and which created a node that occurs in {\tt pools} before the node reached by $k$, i.e., ${\tt n}_s(k)={\tt n}_s(k_1)$, $k_2\leadsto_s k_1$, and ${\tt ts}_s(k_2) \geq {\tt ts}_s(k_1)$ implies $k_2\in ov(k)$, for each $k, k_1, k_2$
	\item if the variable {\tt youngest} of a pop $k$ points to a node which is not taken, then this node was created by a push in $be(k)\cup ov(k)$ or the node currently reached by $k$ is followed in {\tt pools} by another node which was created by a push in $be(k)\cup ov(k)$, i.e., ${\tt youngest}_s(k)={\tt n}_s(k_1)$, ${\tt n}_s(k_1)\text{{\tt ->taken}}={\tt false}$, and ${\tt n}_s(k)={\tt n}_s(k_2)$ implies $k_1\in be(k)\cup ov(k)$ or that there exists $k_3\in O$ such that ${\tt ts}_s(k_3) > {\tt ts}_s(k_1)$, $k_3\in be(k)\cup ov(k)$, and either $k_2\leadsto_s k_3$ or ${\tt n}_s(k_2)={\tt n}_s(k_3)$ and TODO $k$ is traversing the last list in the array {\tt pools}, for each $k, k_1,k_2$.
\vspace{-2mm}
\end{itemize}
Finally, for every pop operation $k$ such that ${\tt success}(k)={\tt true}$, we have that $rv(k)={\tt youngest}(k)\text{\tt->data}$. 

The proof that $\mathit{fs}_2$ is indeed a forward simulation from $\mathit{TSS}$ to $AbsS$ follows the same lines as the one given for the Herlihy\&Wing Queue. It can be found in Appendix~\ref{}.









%\section{Existence of Forward Simulations for Set Implementations That Have Fixed Remove Linearization Points }
For all the set libraries we fix $\mathcal{M} = \{ add, rmv, cnt\}$ where $rmv$ is short for remove and $cnt$ is short for contains methods and $\mathcal{D} = \{1, \texttt{TRUE}, \texttt{FALSE} \}$. We assume that only single element can be inserted into our list. Our results for this domain extends to other domains such as when $\mathcal{D} = \mathbb{N}$. We extend the definition of $A\Sigma$ introduced in Definition 2 for the set in our focus as $AS\Sigma = A\Sigma \cup \{lin(rmv,d,k)| d \in \{1\}, k \in \mathbb{N}\}$. We define s-refinement, s-linearizability and change the definitions of forward and backward simulations as in the beginning of Section 2.

We define $L_A$ as follows:
\begin{itemize}
\item $Q_A = 2^{\{1\}} \times (\mathbb{N} \rightarrow Lbl_A)$ where $Lbl_A = \{N, A_0, A_{1T}, A_{1F} A_2, R_0, R_{1T}, R_{1F}, R_2, C_0, C_{1T}, C_{1F}, C_2 \}$. For each $q \in Q_A$, $s_q$ represents the set component (first component), and $f_q$ represents the program counter (second component).
\item Transition labels consists of invocations, returns and linearizations of methods in $\mathcal{M}$: $\Sigma_A: AS\Sigma \cup \{lin(m,d,k)| m \in \{add,cnt\}, d\in \{1\}, k \in \mathbb{N} \}$. All methods return \texttt{TRUE} or \texttt{FALSE} and all take an element in $\{1\}$ as the input value.
\item ${q_0}_A = (\emptyset, f_{{q_0}_A}$ where $f_{{q_0}_A}(k) = N$ for all $k \in \mathbb{N}$.
\item State transitions:
\begin{itemize}
\item $(q, inv(add, d,k), q') \in \delta_A$ iff $d \in \{1\} \wedge f_q(k) = N \wedge f_{q'}(k) = A_0$
\item $(q, lin(add,d,k), q') \in \delta_A$ iff $d \in \{1\} \wedge f_q(k) = A_0 \wedge (d \in s_q \wedge s_q = s_{q'} \wedge f_{q'}(k) = A_{1F} \vee d \notin s_q \wedge s_{q'} = s_q \cup \{d\} \wedge f_{q'}(k) = A_{1T})$
\item $(q, ret(add,d,k) q') \in delta_A)$ iff $(f_q(k) = A_{1T} \wedge d = \texttt{TRUE} \vee f_q(k) = A_{1F} \wedge d = \texttt{FALSE}) \wedge f_{q'}(k) = A_2$
\item $(q, inv(rmv,d,k) q') \in \delta_A$ iff $d \in \{1\} \wedge f_q(k) = N \wedge f_{q'}(k)= R_0 $
\item $(q, lin(rmv,d,k), q') \in \delta_A$ iff $d \in \{1\} \wedge f_q(k) = R_0 \wedge (d \in s_q \wedge s_q = s_{q'} \cup \{d\} \wedge f_{q'}(k) = R_{1T} \vee d \notin s_q \wedge s_q = s_{q'} \wedge f_{q'}(k) = R_{1F} )$
\item $(q, ret(rmv,d,k) q') \in delta_A)$ iff $(f_q(k) = R_{1T} \wedge d = \texttt{TRUE} \vee f_q(k) = R_{1F} \wedge d = \texttt{FALSE}) \wedge f_{q'}(k) = R_2$
\item $(q, inv(cnt,d,k) q') \in \delta_A$ iff $d \in \{1\} \wedge f_q(k) = N \wedge f_{q'}(k)= C_0 $
\item $(q, lin(cnt,d,k), q') \in \delta_A$ iff $d \in \{1\} \wedge f_q(k) = C_0 \wedge s_q = s_{q'} \wedge (d \in s_q \wedge f_{q'}(k) = C_{1T} \vee d \notin s_q \wedge f_{q'}(k) = C_{1F} )$
\item $(q, ret(cnt,d,k) q') \in delta_A)$ iff $(f_q(k) = C_{1T} \wedge d = \texttt{TRUE} \vee f_q(k) = C_{1F} \wedge d = \texttt{FALSE}) \wedge f_{q'}(k) = C_2$
\end{itemize}
\end{itemize}

We define $L_I$ as follows:
\begin{itemize}
\item A state $q \in Q_I$ is a tuple of the form $(sac_q, src_q, UBSA_q, LBSA_q, CC_q, ubi,lbi, f_q)$ where $sac_q \in \mathbb{N}$ keeps the number of adds that return true so far, $src_q \in \mathbb{N}$ keeps the number of successful linearizations of remove (ones that move program counter from $R_0$ to $R_{1T}$), $f:\mathbb{N} \rightarrow Lbl_I$ is program counter that maps every operation to a label in $Lbl_I = \{N, A_0, A_1, R_0, R_{1T}, R_{1F}, R_2, C_0, C_1\}$. Let $PA_q = \{k \in \mathbb{N}| f_q(k) = A_0\}$ be the set of pending adds at state $q$. Then, $UBSA_q, LBSA_q: \mathbb{N} \rightarrow 2^{PE_q}$ are functions such that $UBSA_q(i)$ is a set of pending adds at most $i$ of which may return true and $LBSA_q(i)$ is a set of pending adds at least $i$ of which may return true. $ubi, lbi \in \mathbb{N}$ keeps the upper bound (lower bound) set indices that a new pending enqueue will be inserted to. Let $IC_q = \{k \in \mathbb{N}| f_q(k) = C_0 \vee f_q(k) = C_1$  be the set of contains operation identifiers. Then, $CC_q: IC_q \rightarrow 2^{PA}$ is a map that keeps a set of pending adds of which at least one should return true due to a true return of a contains operation.
\item Transition labels are exactly the abstract transition labels we have defined earlier: $\Sigma_I = AS\Sigma$
\item ${q_0}_I =(sac_{{q_0}_I}, src_{{q_0}_I}, UBSA_{{q_0}_I}, LBSA_{{q_0}_I}, CC_{{q_0}_I}, ubi_{{q_0}_I}, lbi_{{q_0}_I}, f_{{q_0}_I})$ where $sac_{{q_0}_I} = src_{{q_0}_I} =  0$, $lbi_{{q_0}_I} = ubi_{{q_0}_I} = 1$ and $f_{{q_0}_I}(k) = N$ for all $k \in \mathbb{N}$. Hence $PA_{{q_0}_I} = IC_{{q_0}_I} = \emptyset$ and $UBSA_{{q_0}_I}$, $LBSA_{{q_0}_I}$ and $CC_{{q_0}_I}$ are empty mappings.
\item Instead of giving state transition relation formally, I would like to explain how $UBSA$, $LBSA$ and $CC$ works by considering the possible state transformations (consider $q$ as pre-state and $q'$ as the post state as convention for the following):
\begin{itemize}
\item[$UBSA$] Initially, $ubi_{{q_0}_I} = 1$. Hence, a newly invoked add operation $a$ will be inserted into $UBSA_{{q_0}_I}(1)$. All the newly invoked operations are inserted to $UBSA_q(1)$ until a linearization of remove comes or one of the adds return successful. If the first case happens, we increment the index of every set by 1 i.e. $UBSA_{q'}(i) = UBSA_q(i-1)$ for all $i>0$ and $UBSA_{q'}(0) = \emptyset$. We also set $ubi_{q'} = 1$, if it was set to $0$ somehow before (Consider the trace $inv(add,a,1), ret(add,true,1), inv(add,a,2), lin(rmv,a,3)$. $ubi =0$ after second event and it needs to be incremented to $1$ after the fourth event). Next, consider the case of successful add. Let the operation identifier of the add that will return successful be $k$ and $i$ be the index such that $k \in UBSA_q(i)$. Then, for all $j > i$, $UBSA_{q'} (j-1) = UBSA_q(j)$, $UBSA_{q'}(i-1) = UBSA_q(i-1) \cup UBSA_q(i) \backslash \{ k\}$ and for all $j < i-1$, $UBSA_{q'}(j) = UBSA_q(j)$. We do not allow elements in $UBSA_q(0)$ to return true. So, $i=0$ cannot be true. If $i = 1$, we change $ubi_{q'} = 0$ to denote that no new coming add can return true from now on (consider the case: $inv(add,a,1), lin(rmv,a,2):true, inv(add,a,3), inv(add,a,4), lin(rmv,a,5):true, ret(add,true,3), ret(add,true,4), inv(add,a,6)$. This last invocation should be added to set with index $0$. So, we need to change $ubi$ to $0$ after last return. Lastly, a failing return of add simply removes this add from its set, without changing its index. Let $i$ be the index of the failing add operation $k$. Then, $UBSA_{q'}(i) = UBSA_q(i) \backslash \{k\}$. Observing a failing contains operation may change the upper bound. Consider the following trace: $inv(add,a,1), inv(add,a,2), inv(cnt,a,3), ret(cnt,false,3)$. Although $UBSA_{q'}(1) = \{1,2\}$ after the last event, none of $1$ or $2$ can return true due to the restrictions we impose on contains operations that we will explain later on.
\item[$LBSA$] Initially $lbi_{{q_0}_I} = 1$. When a new event of type invoke add comes, we modify $LBSA_{q'}(lbi_{q'}) = PE_{q'}$ where $lbi_{q'} = lbi_q$. Although it looks like that we build $LBSA_{q'}$ from scratch with every new add invocation, in practice, this is merely adding new pending add to the old set of the same index unless the $lbi$ field are the same. If a new successful linearization of remove comes, first it updates $LBSA_{q'}(lbi_q) = PE_{q'}$. Then, it increments the $lbi$ value by one if it was not $0$ before. If it was $0$ before, new $lbi$ value becomes $src_{q'} - sac_{q'}+1$. Note that a successful remove does not modify $LBSA$ field if a new add is invoked after the previous linearization of a successful remove. The first new add invocation that comes after a successful remove, copies the set $LBSA_{q}(lbi_q)$ and adds the new add's operation identifier as $LBSA_{q'}(lbi_{q'})$. However, if no new add comes between two successful remove linearizations, then $LBSA$ is modified by a sucsessful remove linearization. Next, consider the return events of add operations. First, consider the successful return. If an add operation with identifier $k$ returns true, we remove this identifier from the $LBSA$ sets of which $k$ is element of and decrement the index values of these sets by $1$. We can observe that if $k$ is the element of the set with index $i$ then it is element of every set with index $j>i$ \textcolor{red}{(We need to prove this)}. Hence we decrement the index of every set bigger than some $i$ value. For this case, we may end up with a situation that two sets fall into same index (at index $i-1$). In this case, one of the sets must be strictly subset of the other one \textcolor{red}{(We need to prove this)}. In this case, we keep the subset as the set of this index and delete the larger set. If $k$ is deleted from just no sets, we change $lbi_{q'} = 0$. To rationalize this behavior, consider the trace $inv(add,a,1), lin(rmv,a,2):true, inv(add,a,3), inv(add,a,4), lin(rmv,a,5):true, ret(add,true,3), ret(add,true,4), inv(add,a,6)$. The last add invocation should be put to the set with index $0$ and after the add with identifier $4$ returns true, our procedure makes $lbi_{q'} = 0$. If the removed element is in $LBSA_q(lbi_q) \backslash LBSA_q(lbi_q -1)$, then we also set $lbi_{q'} = 0$. The rationale behind this is the trace: $inv(add,a,1), lin(rmv,a,2):true, inv(add,a,3), ret(add,true,3), inv(add,a,4)$. The last operation with ID $4$ should be able to return false. If an add operation with identifier $k$ returns false, then we remove $k$ from all $LBSA$ sets that $k$ is element of. The constraint we check on $LBSA$ sets is that a return false event of an add operation keeps $|LBSA_{q'}(i)| \geq i$ for the indices $i$ it modified.
\item [$CC$] We need to keep a cc flag to show that now this operation may return true. It should be a map from identifiers to Boolean.
\end{itemize}
\end{itemize}


%!TEX root = draft.tex
\vspace{-1mm}
\section{Related Work}
%\vspace{-1.5mm}
Many techniques for linearizability verification, e.g.,~\cite{conf/ppopp/VafeiadisHHS06,conf/cav/AmitRRSY07,conf/vmcai/Vafeiadis09,conf/tacas/AbdullaHHJR13}, are based on forward simulation arguments, and typically only work for libraries where the linearization point of every invocation of a method $m$ is fixed to a particular statement in the code of $m$. The works in~\cite{conf/cav/Vafeiadis10,Derrick2011,conf/cav/DragoiGH13,DBLP:conf/cav/ZhuPJ15} deal with \emph{external} linearization points where the action of an operation $k$ can be the linearization point of a concurrently executing operation $k'$. We say that the linearization point of $k'$ is external. This situation arises in read-only methods like the {\tt contains} method of an optimistic set~\cite{conf/podc/OHearnRVYY10}, libraries based on the elimination back-off scheme, e.g.,~\cite{conf/spaa/HendlerSY04}, or flat combining~\cite{DBLP:conf/spaa/HendlerIST10,DBLP:conf/podc/GorelikH13}. 
In these implementations, an operation can do an update on the shared state that becomes the linearization point of a concurrent read-only method (e.g., a {\tt contains} returning {\tt true} may be linearized when an {\tt add} method adds a new value to the shared state) or an operation may update the data structure on behalf of other concurrently executing operations (whose updates are published in the shared state). In all these cases, every linearization point can still be associated syntactically to a statement in the code of a method and doesn't depend on operations executed in the future (unlike $\mathit{HWQ}$ and $\mathit{TSS}$). However, identifying the set of operations for which such a statement is a linearization point can only be done by looking at the whole program state (the local states of all the active operations). This poses a problem in the context of compositional reasoning (where auxiliary variables are required), but still admits a forward simulation argument. For manual proofs, such implementations with external linearization points can still be defined as LTSs that produce $Lin$-complete traces and thus still fall in the class of implementations for which forward simulations are enough for proving refinement. These proof methods are not complete and they are not able to deal with implementations like $\mathit{HWQ}$ or $\mathit{TSS}$.

There also exist linearizability proof techniques based on backward simulations or alternatively, prophecy variables, e.g.,~\cite{phd/Vafeiadis08,DBLP:conf/cav/SchellhornWD12,DBLP:conf/pldi/LiangF13}. These works can deal with implementations where the linearization points are not fixed, but the proofs are conceptually more complex and less amenable to automation.

The works in~\cite{conf/concur/HenzingerSV13,DBLP:conf/icalp/BouajjaniEEH15} propose reductions of linearizability to assertion checking where the idea is to define finite-state automata that recognize violations of concurrent queues and stacks. These automata are simple enough in the case of queues and there is a proof of $\mathit{HWQ}$ based on this reduction~\cite{conf/concur/HenzingerSV13}. However, in the case of stacks, the automata become much more complicated and we are not aware of a proof for an implementation such as $\mathit{TSS}$ which is based on this reduction.


\bibliographystyle{abbrvnat}
\bibliography{violin-short}

\newpage
\appendix

%!TEX root = draft.tex
\section{Libraries}\label{app:prelim}

Programs interact with libraries by calling named library \emph{methods}, which
receive \emph{parameter values} and yield \emph{return values} upon completion.
We fix arbitrary sets $\<Methods>$ and $\<Vals>$ of method names and
parameter/return values. 

%\begin{example}
%  \label{ex:methods}
%
%  TODO
%
%  The method and value sets for the stack implementation in
%  Figure~\ref{fig:treiber} are $\<Methods> = \set{ \<push>, \<pop> }$ and
%  $\<Vals> = \<Nats> \u \set{ {\tt EMPTY} }$.
%
%\end{example}

\noindent
We fix an arbitrary set $\<Ops>$ of operation identifiers, and for given sets
$\<Methods>$ and $\<Vals>$ of methods and values, we fix the sets
\begin{align*}
  & C = \set{ inv(m,d,k) : m \in \<Methods>, d \in \<Vals>, k \in \<Ops> }
  \text{, and } \\
  & R = \set{ ret(m,d,k) : m \in \<Methods>, d \in \<Vals>, k \in \<Ops> }  
\end{align*}
of \emph{call actions} and \emph{return actions}; each call action $inv(m,d,k)$
combines a method $m \in \<Methods>$ and value $d \in \<Vals>$ with an
\emph{operation identifier} $k \in \<Ops>$. Operation identifiers are used to
pair call and return actions. 
We assume every set of 
words is closed under isomorphic renaming of operation identifiers. 
We denote the operation identifier of a
call/return action $a$ by $\<op>(a)$. Call and return actions $c \in C$ and $r
\in R$ are \emph{matching}, written $c \match r$, when $\<op>(c) = \<op>(r)$. 
We may omit the second field from a call/return action $a$ for methods that have no inputs (e.g., the pop method of a stack) or return values (e.g., the push method of a stack).
A word $\tau \in @S^*$ over alphabet $@S$, such that $(C \u R) \subseteq @S$, is
\emph{well formed} when:
\begin{itemize}

  \item Each return is preceded by a matching call: \\
  $\tau_j \in R$ implies $\tau_i \match \tau_j$ for some $i < j$.

  \item Each operation identifier is used in at most one call/return: \\
  $\<op>(\tau_i) = \<op>(\tau_j)$ and $i < j$ implies $\tau_i \match \tau_j$.

\end{itemize}
We say that the well-formed word $\tau \in @S^*$ is \emph{sequential} when
\begin{itemize}

  \item Operations do not overlap: \\
  $\tau_i, \tau_k \in C$ and $i < k$ implies $\tau_i \match \tau_j$ for some $i < j < k$.

\end{itemize}
Well-formed words represent traces of a library. We assume every set of well-formed
words is closed under isomorphic renaming of operation identifiers. For
notational convenience, we take $\<Ops>=\<Nats>$ for the rest of the paper.
When the value of a certain field in a call/return action is not important we use 
the placeholder $\_$, e.g., $inv(m,\_,k)$ instead of $inv(m,d,k)$ when the input  
$d$ can take any value.

An operation $k$ is called \emph{completed} in a well-formed trace $\tau$ when
$ret(m,d,k)$ occurs in $\tau$, for some $m$ and $d$. Otherwise, it is called \emph{pending}.
%An operation $o$ of an execution $e$ is \emph{completed}
%when both call and return actions $m(u)_o$ and $\<ret>(v)_o$ of $o$ occur in
%$e$, and is otherwise \emph{pending}.

%\begin{example}
%  \label{ex:executions}
%
%  TODO
%
%  The well-formed words
%  \scriptsize
%  \begin{align*}
%     \<push>(0)_1\ \<pop>_2\ \<pop>_3\ \<ret>_1\ \<ret>(0)_3\ \<ret>(0)_2
%    \text{\normalsize, and } 
%    \<push>(0)_1\ \<pop>_2\ \<pop>_3\ \<ret>_1\ \<ret>(0)_2
%  \end{align*}
%  \normalsize
%  represent executions in which one call to the $\<push>(0)$ method overlaps
%  with two calls to $\<pop>$. In the first execution both calls to $\<pop>$
%  have matching return actions $\<ret>(0)$, i.e., the operations $2$ and $3$ are completed,
%  while operation $3$ is pending in the second, it has no matching return.
%
%\end{example}

Libraries dictate the execution of methods between their call and return
points. Accordingly, a library cannot prevent a method from being called,
though it can decide not to return. Furthermore, any library action performed
in the interval between call and return points can also be performed should the
call have been made earlier, and/or the return made later. 
A library thus allows any sequence of
invocations to its methods made by \emph{some} program.

\begin{definition}\label{def:libraries}
A \emph{library} $L$ is an LTS over alphabet $\Sigma$ such that $C \u R\subseteq \Sigma$
and each trace $\tau \in Tr(L)$ is well formed, and
  \begin{itemize}

    \item Call actions $c \in C$ cannot be disabled: \\
    $\tau \cdot \tau' \in Tr(L)$ implies $\tau \cdot c \cdot \tau' \in Tr(L)$
    if $\tau \cdot c \cdot \tau'$ is well formed.
  
    \item Call actions $c \in C$ cannot disable other actions: \\
    $\tau \cdot a \cdot c \cdot \tau' \in Tr(L)$ implies $\tau \cdot c \cdot a \cdot \tau' \in Tr(L)$.
  
    \item Return actions $r \in R$ cannot enable other actions: \\
    $\tau \cdot r \cdot a \cdot \tau' \in Tr(L)$ implies $\tau \cdot a \cdot r \cdot \tau' \in Tr(L)$.
  
  \end{itemize}

\end{definition}

Note that even a library that implements \emph{atomic methods}, e.g.,~by
guarding method bodies with a global-lock acquisition, admits executions in
which method calls and returns overlap. 
For simplicity, Definition~\ref{def:libraries} assumes that every thread performs a single operation. The extension to multiple operations per thread is straightforward, e.g. the closure rules must assume that the actions $a$ and $c$ belong to different threads
%A library which accesses the client's
%thread identifiers can be modeled by taking thread identifiers as method
%parameters.

%\textcolor{red}{For the below paragraph, I cannot see the equivalence  of informal explanation and formal definition of weakening. I think the formal one should be if a ret comes before a call in $h_1'$ then this order is preserved in $h_2$. But the definition says iff. Ignore this if I am wrong.}


%\begin{example}
%  \label{ex:libraries}
%
%  TODO (replace executions with histories)
%
%  Any library which admits the execution
%  \scriptsize
%  \begin{align*}
%    \<push>(0)_1\ \<ret>_1\ \<pop>_2\ \<ret>(0)_2
%  \end{align*}
%  \normalsize
%  with sequential calls to $\<push>$ and $\<pop>$ must also admit
%  \scriptsize
%  \begin{align*}
%    \<push>(0)_1\ \<pop>_2\ \<ret>_1\ \<ret>(0)_2
%    \text{ \normalsize and }
%    \<push>(0)_1\ \<pop>_2\ \<pop>_3\ \<ret>_1\ \<ret>(0)_2
%    \text{\normalsize,}
%  \end{align*}
%  \normalsize
%  among others, yet need not admit an execution
%  \scriptsize
%  \begin{align*}
%    \<push>(0)_1\ \<pop>_2\ \<pop>_3\ \<ret>_1\ \<ret>(0)_3\ \<ret>(0)_2
%  \end{align*}
%  \normalsize
%  with two completed $\<pop>$ operations returning $0$.
%  
%\end{example}




%Systems we consider are labeled transition systems (LTS):
%\begin{dfn}
%An LTS is defined over four-tuples $A=(Q,\Sigma, q_0, \delta)$ where $Q$ is the set of states, $\Sigma$ is the set of transition labels, $q_0 \in Q$ is the initial state and $\delta \subseteq Q \times \Sigma \times Q$ is the transition relation.
%\end{dfn}
%Executions generated by this system are alternating sequence of states and transition labels $\rho = s_0, e_0, s_1,... s_k, e_k,...$ where each $s_i \in Q$, each $e_i \in \Sigma$, $s_0 = q_0$ and each $(s_i, e_i s_{i+1}) \in \delta$. The projection of the sequence $\rho$ over the set $\Pi$ is denoted by $\rho | \Pi$, and it is the maximum subsequence of $\rho$ consisting of elements of $\Pi$. Traces of the LTS are obtained from executions by projecting them over $\Sigma$. For the rest of the paper and in all of the proofs, we consider only finite executions (denoted as $E(A)$) and/or traces (denoted as $Tr(A)$ of the LTSs in focus.

%Libraries are LTSs that provide methods. Let $\mathcal{M}$ be the set of method names and $\mathcal{D}$ be the domain of values as input/output parameters for the methods. Then, this library contains transition labels of the form $inv(m,d,i)$ representing the invocation of method $m \in \mathcal{M}$ with input value $d \in \mathcal{D}$. The third field is the operation identifier for differentiating the different calls of the same method from the set $\mathcal{O}$. For simplicity, we take $\mathcal{O} = \mathbb{N}$ for the rest of the paper. We also assume that methods could have at most one input parameter. If they do not have any input arguments (like pop method of a stack), we can omit the second field from the action. They also provide actions of the form $ret(m,d,i)$ representing the return of method $m \in M$ with value $d \in D$ which has been invoked previously with action $inv(m,d',i)$. Again, we assume that the methods can return at most one parameter and we may omit the second field from the action if they have none (like enqueue method of a queue). Before starting to reason about any set of libraries, we first fix the sets $\mathcal{M}$ and $\mathcal{D}$ and libraries in our focus agree on this sets. For any transition label $e = inv(m,d,i)$ or $e=ret(m,d,i)$, we have the function $oid(e) = i$.
%
%Since libraries are LTSs, they produce traces. A trace $e = e_1, e_2, ..., e_n$ of library $L$ is \emph{well-formed} iff (i) every return matches an earlier invocation: $e_j = ret(m,d,k)$ implies that there exists $i<j$ such that $e_i = inv(m,d',k)$ and (ii) every operation identifier is used at most one invocation/return pair: $oid(e_i) = oid(e_j) = k$ and $i<j$ implies $e_i = inv(m,d,k)$ and $e_j = ret(m,d',k)$. From now on, we assume that libraries produce well-formed traces. Let $f: \mathbb{N} \rightarrow \mathbb(N)$ be a bijection. Then, traces $e$ and $e'$ are equivalent if $e'$ is obtained from $e$ by replacing every action $inv(m,d,k)$ with $inv(m,d,f(k))$ and every action $ret(m,d,k)$ with $ret(m,d,f(k))$. 


%!TEX root = draft.tex
\section{Normal Forward/Backward Simulations}\label{app:backSim}

We define a class of forward/backward simulations, called \emph{normal simulations}, that are used in the proofs in Appendix~\ref{app:absImplQueue} and Appendix~\ref{app:absImplStack}. 

\begin{definition}\label{def:for_app}
Let $L_1=(Q_1,\Sigma, s_0^1, \delta_1)$ and $L_2=(Q_2,\Sigma, s_0^2, \delta_2)$ be two libraries over alphabets $\Sigma_1$ and $\Sigma_2$, respectively, such that $C\cup R \subseteq \Sigma_1\cap\Sigma_2$, and $\Gamma$ a set of actions such that $C\cup R\subseteq \Gamma\subseteq \Sigma_1\cap\Sigma_2$. A relation $\mathit{fs} \subseteq Q_{1} \times Q_{2}$ is called a \emph{normal $\Gamma$-forward simulation} from $L_1$ to $L_2$ iff the following holds:
\begin{itemize}
\item[(i)] $\mathit{fs}[s_0^1] = \{s_0^2 \}$ 
\item[(ii-a)] If $(s,c,s') \in \delta_1$, for some $c\in C$, and $u \in \mathit{fs}[s]$, then there exists $u' \in \mathit{fs}[s']$ such that $u \xrightarrow{@s} u'$, $@s_0=c$, and $@s_i\in \Sigma_2\setminus\Gamma$, for each $0<i<|@s|$.
%$@s = a_1, a_2, ..., a_n$ such that $a_1 = inv(m,d,k)$ and for all $i \in [2,n]$, $a_i \in \Sigma_{L_2} \backslash A\Sigma $. The expression $u \xrightarrow{a} u'$ means that there exists a sequence of states $u_1, u_2,...,u_{n+1}$ such that $u_1 = u$, $u_{n+1} = u'$ and for all $i \in [1,n]$, $(u_i, a_i, u_{i+1}) \in \delta_{L_2}$.
\item[(ii-b)] If $(s,r,s') \in \delta_{1}$, for some $r\in R$, and $u \in \mathit{fs}[s]$, then there exists $u' \in \mathit{fs}[s']$ such that $u \xrightarrow{@s} u'$, $@s_{|@s| -1}=r$, and $@s_i\in \Sigma_2\setminus\Gamma$, for each $0\leq i<|@s| -1$.
% where $a = a_1, a_2, ..., a_n$ such that $a_n = ret(m,d,k)$ and for all $i \in [1,n-1]$, $a_i \in \Sigma_{L_2} \backslash A\Sigma $. 
\item[(ii-c)] If $(s, \gamma , s') \in \delta_1$, for some $\gamma\in \Gamma\setminus (C\cup R)$, and $u \in fs[s]$, then there exists $u' \in fs[s']$ such that $\delta_2(u,\gamma, u')$. 
\item[(ii-d)] If $(s,e,s') \in \delta_1$, for some $e \in \Sigma_1\setminus \Gamma$ and $u \in \mathit{fs}[s]$, then there exists $u' \in \mathit{fs}[s']$ such that $u \xrightarrow{\sigma} u'$ and $\sigma\in (\Sigma_2\setminus\Gamma)^*$.  
%where $a = a_1, a_2, ..., a_n$ such that for all $i \in [1,n]$, $a_i \in \Sigma_{L_2} \backslash A\Sigma $. Moreover, $a$ could be the empty sequence.
\end{itemize}
\end{definition}
%If $\mathit{fs}[s]$ is a unique state for all $s \in Q_1$ then $\mathit{fs}$ is called a refinement mapping/function. 
With normal $\Gamma$-forward simulations, a step of $L_1$ labeled by a call, resp., return, action is simulated by a sequence of steps of $L_2$ that start, resp., end, with the same action, and a step of $L_1$ labeled by another observable action should be matched by a step of $L_2$ labeled by the same action. The rest of the transitions in $L_1$ are matched to a possibly empty sequence of transitions of $L_2$ with arbitrary labels.

A dual notion of forward simulation is the backward simulation:
\begin{definition}\label{def:back_app}
Let $L_1=(Q_1,\Sigma, s_0^1, \delta_1)$ and $L_2=(Q_2,\Sigma, s_0^2, \delta_2)$ be two libraries over a common alphabet $\Sigma$, and $\Gamma\subseteq \Sigma$ a set of actions such that $(C\cup R)\subseteq \Gamma$. A relation $bs \subseteq Q_1 \times Q_2$ is called a \emph{normal $\Gamma$-backward simulation} from $L_1$ to $L_2$ iff the following holds:
\begin{itemize}
\item[(i)] $bs[s_0^1] = \{s_0^2 \}$
\item[(ii-a)] If $(s,c,s') \in \delta_1$, for some $c\in C$, and $u' \in bs[s']$, then there exists $u \in bs[s]$ such that $u \xrightarrow{@s} u'$, $@s_0=c$, and $@s_i\in \Sigma\setminus\Gamma$, for each $0<i<|@s|$.
% where $a = a_1, a_2, ..., a_n$ such that $a_1 = inv(m,d,k)$ and for all $i \in [2,n]$, $a_i \in \Sigma_{L_2} \backslash A\Sigma $. 
\item[(ii-b)] If $(s,r,s') \in \delta_1$, for some $r\in R$, and $u' \in bs[s']$, then there exists $u \in bs[s]$ such that $u \xrightarrow{@s} u'$, $@s_{|@s| -1}=r$, and $@s_i\in \Sigma\setminus\Gamma$, for each $0\leq i<|@s| -1$.
%where $a = a_1, a_2, ..., a_n$ such that $a_n = ret(m,d,k)$ and for all $i \in [1,n-1]$, $a_i \in \Sigma_{L_2} \backslash A\Sigma $. 
\item[(ii-c)] If $(s,\gamma, s') \in \delta_1$, for some $\gamma\in \Gamma\setminus (C\cup R)$, and $u' \in bs[s']$, then there exists $u \in bs[s]$ such that $\delta_2(u,\gamma,u')$
\item[(ii-d)] If $(s,e,s') \in \delta_1$ for some $e \in \Sigma\setminus \Gamma$ and $u' \in bs[s']$, then there exists $u \in bs[s]$ such that $u \xrightarrow{@s} u'$ and $\sigma\in (\Sigma_2\setminus\Gamma)^*$.
%where $a = a_1, a_2, ..., a_n$ such that for all $i \in [1,n]$, $a_i \in \Sigma_{L_2} \backslash A\Sigma $. Moreover, $a$ could be the empty sequence.
\end{itemize}
\end{definition}

%\begin{lem}
%Let $L_1$ and $L_2$ be two libraries over a common alphabet $\Sigma$, and $\Gamma\subseteq \Sigma$ a set of actions such that $(C\cup R)\subseteq \Gamma$. $L_1$ $\Gamma$-refines $L_2$ if{f} there exists a $\Gamma$-backward simulation from $L_1$ to $L_2$.
%\end{lem}
%\begin{proof}
%Looks trivial and follows Lynch paper. Can be completed later.
%\end{proof}

%!TEX root = draft.tex
\section{Proof of Theorem~\ref{th:absImplQueue}}\label{app:absImplQueue}

\begin{figure} [t]
{\scriptsize
  \centering
  \begin{mathpar}
    \inferrule[call-enq]{
      k\not\in dom(cp^0) \\ 
      d\neq {\tt EMPTY}
    }{
      \sigma,in^0,rv^0,cp^0
      \xrightarrow{inv(enq,d,k)}
      %O\cup\{k\},<\cup \{(k',k): \ell_2(k')={\tt COMP}\},\ell[k\mapsto (d,{\tt PEND})],rv,cp[k\mapsto 1]
      \sigma,in^0[k\mapsto d], rv^0,cp^0[k\mapsto A_1]
    }\hspace{5mm}
    \inferrule[lin-enq]{
      cp^0(k)=A_1
    }{
      \sigma,in^0,rv^0,cp^0
      \xrightarrow{lin(enq,d,k)}
      %O\cup\{k\},<\cup \{(k',k): \ell_2(k')={\tt COMP}\},\ell[k\mapsto (d,{\tt PEND})],rv^0,cp[k\mapsto 1]
      d\cdot\sigma,in^0,rv^0,cp^0[k\mapsto A]
    }\hspace{5mm}
    
        \inferrule[ret-enq]{
      cp^0(k)=A
    }{
      \sigma,in^0,rv^0,cp^0
      \xrightarrow{ret(enq,k)}
      %O\cup\{k\},<\cup \{(k',k): \ell_2(k')={\tt COMP}\},\ell[k\mapsto (d,{\tt PEND})],rv^0,cp[k\mapsto 1]
      \sigma,in^0,rv^0,cp^0[k\mapsto A_2]
    }\hspace{5mm}


    \inferrule[call-deq]{
      k\not\in dom(cp^0) \\ 
    }{
      \sigma,in^0,rv^0,cp^0
      \xrightarrow{inv(deq,k)}
      %O\cup\{k\},<\cup \{(k',k): \ell_2(k')={\tt COMP}\},\ell[k\mapsto (d,{\tt PEND})],rv^0,cp[k\mapsto 1]
      \sigma,in^0,rv^0,cp^0[k\mapsto R_1]
    }\hspace{5mm}
        \inferrule[lin-deq1]{
      cp^0(k)=R_1 \\
      \sigma = \sigma'\cdot d 
    }{
      \sigma,in^0,rv^0,cp^0
      \xrightarrow{lin(deq,d,k)}
      %O\cup\{k\},<\cup \{(k',k): \ell_2(k')={\tt COMP}\},\ell[k\mapsto (d,{\tt PEND})],rv^0,cp[k\mapsto 1]
      \sigma',in^0,rv^0[k\mapsto d],cp^0[k\mapsto R_2]
    }\hspace{5mm}

        \inferrule[lin-deq2]{
      cp^0(k)=R_1 \\
      \sigma = \epsilon
    }{
      \sigma,in^0,rv^0,cp^0
      \xrightarrow{lin(deq,{\tt EMPTY},k)}
      %O\cup\{k\},<\cup \{(k',k): \ell_2(k')={\tt COMP}\},\ell[k\mapsto (d,{\tt PEND})],rv^0,cp[k\mapsto 1]
      \sigma,in^0,rv^0[k\mapsto {\tt EMPTY}],cp^0[k\mapsto R_2]
    }\hspace{5mm}
    \inferrule[ret-deq]{
      cp^0(k)=R_2 \\
      rv^0(k) = d
    }{
      \sigma,in^0,rv^0,cp^0
      \xrightarrow{ret(deq,d,k)}
      %O\cup\{k\},<\cup \{(k',k): \ell_2(k')={\tt COMP}\},\ell[k\mapsto (d,{\tt PEND})],rv^0,cp[k\mapsto 1]
      \sigma,in^0,rv^0,cp^0[k\mapsto R_3]
    }\hspace{5mm}
          \end{mathpar}
  }
 \vspace{-4mm}
  \caption{The transition relation of $AbsQ_0$. 
  %\textcolor{red}{Call-Enq must have $d!= \texttt{EMPTY}$ as a premise. Also lin deq returning empty must be changed as before.}
  }
  \label{fig:transitions:AbsQ_0}
\vspace{-2mm}
\end{figure}

We show that $AbsQ$ and $AbsQ_0$ refine each other. We start by giving a formal definition of the standard reference implementation $AbsQ_0$.
Thus, the states of $AbsQ_0$ are tuples $\tup{\sigma,in^0,rv^0,cp^0}$ where $\sigma\in\<Vals>^*$ is a sequence of values, $in^0:\<Ops> ~> \<Vals>$ records the input value of an enqueue, $rv^0:\<Ops> ~> \<Vals>$ records the return value of a dequeue fixed at its linearization point ($~>$ denotes a partial function), and $cp^0:\<Ops> ~> \{A_1,A,A_2,R_1,R_2,R_3\}$ records the control point of every enqueue ($A_1, A,A_2$) or dequeue operation ($R_1,R_2,R_3$).
All the components are $\emptyset$ in the initial state, and the transition relation $->$ is defined in Fig.~\ref{fig:transitions:AbsQ_0}. The alphabet of $AbsQ$ contains call/return actions and enqueue/dequeue linearization points.

To prove that $AbsQ$ is a refinement of $AbsQ_0$ we define a normal $C\cup R\cup Lin(deq)$-backward simulation (i.e, a backward simulation as in Definition~\ref{def:back_app}) from $AbsQ$ to $AbsQ_0$. The reverse is shown using a normal $C\cup R\cup Lin(deq)$-forward simulation (i.e, a forward simulation as in Definition~\ref{def:for_app}).


%In this section, we will show that for any concurrent queue implementation library $L_C$ for which we know the linearization points of the dequeue operation and that is linearizable with respect to the reference implementation library $L_A$, there exists a forward simulation $fs$ relating $L_C$ to $L_I$ where $L_I$ is an intermediate library equivalent to $L_A$ i.e. $L_I$ refines $L_A$ and vice versa.  
%
%For all the queue libraries, we fix $\mathcal{M} = \{ enq, deq \}$ and $\mathcal{D} = \mathbb{N} \cup \{\texttt{EMPTY} \}$. Since we know the linearization point of dequeues, we extend the definitions of the previous sections adding this information. First, we extend the set $A\Sigma$ introduced in Definition 2 for queues in our focus as $AQ\Sigma = A\Sigma \cup \{lin(deq,d,k)| d \in \mathcal{D}, k \in \mathbb{N}\}$. For any library $L$ we consider in this section, $AQ\Sigma \subseteq \Sigma_L$. Then, we define q-refinement by replacing $A\Sigma$ with $AQ\Sigma$ in Definition 2. We also define q-linearizability, by enforcing $lin(deq,d,k)$ to appear immediately after $inv(deq,k)$ and immediately before $ret(deq,d,k)$ in a sequential execution. We also change definition of forward and backward simulation relations by replacing $A\Sigma$ with $AQ\Sigma$ in the original definitions and adding a new condition $(ii-d)$ to each of them:
%\begin{itemize}
%\item[Forward Simulation: (ii-d)] If $(s, lin(deq,d,k), s') \in \delta_{L_1}$ and $u \in fs[s]$, then there exists $u' \in fs[s']$ such that $(u, lin(deq,d,k), u') \in \delta_{L_2}$
%\item[Backward Simulation: (ii-d)] If $(s, lin(deq,d,k), s') \in \delta_{L_1}$ and $u' \in bs[s']$, then there exists $u \in bs[s]$ such that $(u, lin(deq,d,k), u') \in \delta_{L_2}$
%\end{itemize}
%Lemma 1 of the previous section still holds if we replace refinement with q-refinement and use new definitions of backward and forward simulations.
%
%Next, we define the abstract library $L_A$ as follows:
%\begin{itemize}
%\item A queue state consists of a finite sequence of natural numbers representing the queue content and a program counter for each operation. More formally: $Q_A \subseteq \mathbb{N}^*  \times (\mathbb{N} \rightarrow Lbl_A)$ where $Lbl_A = \{N, E_0, E_1, E_2, D_0, D_1, D_2 \}$ is the set of transition labels of the operations. Operations that have not started yet are mapped to $N$. $E_i$ ($D_i$) denotes particular transitions in enqueue (dequeue) operations that will be clear when we define $\delta_A$. For a state $q$, we denote the contents of the queue (first field) with $s_q$ and the function that maps operations to labels (the second field) with $f_q$.
%\item Transition labels consists of invocations, returns and linearizations of enqueue and dequeue operations: $\Sigma_A = AQ\Sigma \cup \{lin(enq,d,k)| d \in \mathcal{D}, k \in \mathbb{N}\}$. Invocation of $deq$ operation does not have any input and return of $enq$ operation does not have any output value. We omit second fields from these labels.
%\item Initial state is the empty queue: ${q_0}_A = (\langle \rangle, f_{{q_0}_A})$ where $f_{{q_0}_A}(i) = N$ for all $i \in \mathbb{N}$.
%\item Each operation consists of invocation, linearization and return steps. These are reflected in the transition relation. For the below definitions, unchanged parts of the state are omitted from the formulae:
%\begin{itemize}
%\item $(q, inv(enq,d,k), q') \in \delta_A$ iff $ d \neq \texttt{EMPTY} \wedge f_q(k) = N \wedge f_{q'}(k) = E_0$,
%\item $(q,lin(enq,d,k),q') \in \delta_A$ iff $f_q(k) =E_0 \wedge s_{q'} = s_q \circ \langle d \rangle \wedge f_{q'}(k) = E_1$ where $\circ$ is the operation that concatenates two finite sequences,
%\item $(q,ret(enq,k),q') \in \delta_A$ iff $f_q(k) = E_1 \wedge f_{q'}(k) = E_2$,
%\item $(q, inv(deq,k), q') \in \delta_A$ iff $f_q(k) = N \wedge f_{q'}(k) = D_0$
%\item $(q,lin(deq,d,k),q') \in \delta_A$ iff $f_q(k) =D_0 \wedge (d \neq \texttt{EMPTY} \wedge s_q = \langle d \rangle \circ s_{q'} \vee d=\texttt{EMPTY} \wedge s_q = s_{q'} = \langle \rangle )\wedge f_{q'}(k) = D_1$, 
%\item $(q,ret(deq,d,k),q') \in \delta_A$ iff $f_q(k) = D_1 \wedge f_{q'}(k) = D_2$.
%\end{itemize}
%
%\end{itemize} 
%We restrict traces generated by this LTS by neglecting the invalid traces. If we project a trace to some operation identifier $k$ and obtain a sequence $\langle ..., inv(enq,d,k), lin(enq,d',k),...\rangle$ or $\langle ..., lin(deq,d,k), ret(deq,d',k),...\rangle$ where $d \neq d'$ we say that this trace is invalid.
%
%We want to introduce $L_I$ as the next step. States of $L_I$ consists of strict orders of enqueue operations. Nodes of the strict order come from the set $ND = \mathbb{N} \times \mathcal{D} \times \{ \texttt{PENDING}, \texttt{CLOSED}\}$. Basically each node is a tuple keeping the operation ID, the value to be enqueued and if this enqueue is pending or closed. Strict order also keeps a set of directed edges $ ed \subseteq ED = ND \times ND$. Then, a strict order is a tuple $(nd, ed)$ where the set $ed$ obeys the assumption of strict order i.e. irreflexivity, asymmetry and transitivity. Then, the LTS of $L_I$ is defined as follows:
%\begin{itemize}
%\item A state consists of a partial order and a function keeping the program counter. More formally $Q_I \subseteq ND \times ED \times (\mathbb{N} \rightarrow Lbl_I)$ where $Lbl_I = \{N, E_0, E_1, D_0, D_1, D_2\} $ is the set of transition labels. $nd_q$ $ed_q$ and $f_q$ denotes the nodes of the strict order in state $q$, edges of the strict order in state $q$ and the function that maps operation to labels in state $q$, respectively. 
%\item The transition label set consists of invocation and return action of both methods and linearization action of only the dequeue method: $\Sigma_I = AQ\Sigma$. Invocation of $deq$ operation does not have any input and return of $enq$ operation does not have any output value. We omit second fields from these labels.
%\item Initial state consists of an empty strict order and a function mapping every operation to $N$: ${q_0}_{I} = (so_{{q_0}_I}, f_{{q_0}_I}$ where $so_{{q_0}_I} =(\emptyset, \emptyset)$ and $f_{{q_0}_I}(i) = N$ for all $i \in \mathbb{N}$.
%\item While defining $\delta_I$, we again omit mentioning about the parts that has not changed:
%\begin{itemize}
%\item $(q, inv(enq,d,k), q') \in \delta_I$ iff $f_q(k) = N \wedge f_{q'}(k) = E_0 \wedge d \neq \texttt{EMPTY} \wedge (k,\_,\_) \notin nd_q \wedge nd_{q'} = nd_q \cup \{ (k,d,\texttt{PENDING}\} \wedge AddNode(ed_{q'}, ed_q,k)$ where $AddNode(ed_{q'}, ed_q,k)$ is true iff $ed_{q'}$ is obtained from $ed_q$ by adding edges from every closed node of $ed_q$ to the node with identifier $k$.
%\item $(q, ret(enq,k), q') \in \delta_I$ iff $f_q(k) =E_0 \wedge f_{q'}(k) = E_1 \wedge ( (k,\_,\texttt{PENDING}) \notin nd_q \wedge so_q = so_{q'} \vee UpdateNode(so_{q'}, so_q, k) )$ where $UpdateNode(so_{q'}, so_q, k)$ is true iff $(k,d,\texttt{PENDING}) \in nd_q$ for some $d \in \mathcal{D}$, it is replaced with node $(k,d,\texttt{CLOSED}$ in the state $q'$ and all the edges adjacent to $(k,d,\texttt{PENDING})$ in state $q$ are replaced with edges adjacent to $(k,d,\texttt{CLOSED})$ in the state $q'$.
%\item $(q, inv(deq,k), q') \in \delta_I$ iff $f_q(k) = N \wedge f_{q'}(k) = D_0$.
%\item $(q, lin(deq,d,k), q') \in \delta_I$ iff $f_q(k) = D_0 \wedge f_{q'}(k) = D_1 \wedge (d = \texttt{EMPTY} \wedge nd_q = ed_q = nd_{q'} = ed_{q'} \emptyset \vee d \neq \texttt{EMPTY} \wedge RemoveNode(so_q, so_{q'},d))$ where $RemoveNode(so_q, so_{q'},d)$ is true iff there is a minimal node $n \in so_q$ of which data value (second field) is $d$ and $so_{q'}$ is obtained from $so_q$ by removing $n$ and all the edges adjacent to it. Note that linearization of dequeue could remove a \texttt{PENDING} node.  
%\item $(q, ret(deq,d,k) q') \in \delta_I$ iff $f_q(k) = D_1 \wedge f_{q'}(k) = D_2$.
%\end{itemize}
%\end{itemize}
%Again, we omit the inconsistent traces from $L_I$ as in $L_A$. Note that the library $L_I$ is deterministic with respect to alphabe $AQ\Sigma$.
%
%Next, we show equivalence of $L_A$ and $L_I$ in terms of q-refinement.
\begin{lemma} 
$AbsQ$ is a refinement of $AbsQ_0$.
\end{lemma}
\begin{proof}
We define a normal $C\cup R\cup Lin(deq)$-backward simulation $bs$ from $AbsQ$ to $AbsQ_0$ as follows. Given an $AbsQ$ state $s=\tup{O,<,\ell,rv,cp}$ and an $AbsQ_0$ state $t=\tup{\sigma,in^0,rv^0,cp^0}$ we have that $(s,t)\in bs$ iff the following hold:
\begin{itemize}
	\item the sequence $\sigma$ is a linearization of a partial order $(D,\prec)$ where $D$ contains values labeling elements of $O$ and all the values corresponding to completed enqueues, i.e., $\ell_1({\tt COMP}(O))\subseteq D\subseteq \ell_1(O)$ ordered according to the happens-before order between the enqueues that added them, i.e., $d_1\prec d_2$ if{f} there exists $k_1,k_2$ such that $\ell_1(k_1)=d_1$, $\ell_1(k_2)=d_2$, and $k_1 < k_2$.
	\item the return values fixed at dequeue linearization points are the same, i.e., for every $k$, $rv(k)=rv^0(k)$,
	\item every dequeue is at the same control point in both $s$ and $t$, i.e., for every $k$ and $i\in \{1,2,3\}$, $cp(k)=R_i$ iff $cp^0(k)=R_i$,
	\item every pending enqueue has the same input value in both $s$ and $t$, i.e., for every $k$, $\ell_1(k)=in^0(k)$,
	\item a pending enqueue from $O$ has been linearized whenever its value is contained in $\sigma$, i.e., for every $k$, $cp^0(k)=A$ if $\ell_1(k)\in D$ and $\ell_2(k)={\tt PEND}$, 
	\item a pending enqueue from $O$ hasn't been linearized whenever its value is not in $\sigma$, i.e., for every $k$, $cp^0(k)=A_1$ iff $\ell_1(k)\not\in D$ and $\ell_2(k)={\tt PEND}$, 
	\item a pending enqueue which is not in $O$ has been linearized, i.e., for every $k$, $cp^0(k)=A$ if $k\not\in O$ and $cp(k)=A_1$, 
	\item an enqueue is completed in $s$ whenever it is completed in $t$, i.e., for every $k$, $cp(k)=A_2$ iff $cp^0(k)=A_2$,
\end{itemize}

For the conditions described above, if we fix the set $D$ and $\sigma_t$, then the state $t$ related to $s$ becomes unique. We use this fact in the proof. In some places, we only give $D$, $\sigma_t$ and $s$ without explicitly defining $t$ or show that there exists $t$ with the given $\sigma_t$ that is related to $s$ by just finding a $D$ such that $\sigma_t$ is a linearization of $(D, \prec)$ where $\prec$ is induced from $<_s$.

or $\sigma_t$ and not describing $t$ explicitly.

In the following, we show that indeed $bs$ is a normal $C\cup R\cup Lin(deq)$-backward simulation from $AbsQ$ to $AbsQ_0$. % mainly focusing on how to pick $\sigma$ fields of related states in $AbsQ_0$ and how the corresponding actions of $AbsQ_0$ modifies $\sigma$.

%TODO FILL IN THE STEPS
%We will provide a backward simulation relation btw $L_I$ and $L_A$. Given a state $q = (so_q, f_q) \in Q_I$,  $q' \in bs[q]$ iff (i) $s_{q'}$ is a linearization of $so_q$ projected to $d$ fields (second field). This linearization may omit the pending elements (the ones of which third field is \texttt{PENDING}), but it must contain all the \texttt{CLOSED} elements. Note that pending elements of $so_q$ are maximal and they can appear after the closed elements, towards the end of the sequence. (ii) If $f_q(k) \in \{N, D_0, D_1, D_2\}$, then $f_{q'}(k) = f_q(k)$ . If $f_q(k) = E_0$, $(k,d,\texttt{PENDING}) \in nd_q$ and this node participates in the linearization $s_{q'}$. Then $f_{q'}(k) = E_1$. If it does not participate in the linearization, then $f_{q'}(k) = E_0$. If $f_q(k) = E_0$ but there is no node with operation identifier $k$ in $nd_q$, then $f_{q'} = E_1$. Lastly, if $f_q(k) = E_1$ then $f_{q'}(k) = E_2$. Next, we will show that $bs$ is a backward simulation relation.
\begin{itemize}
\item[$\langle i \rangle$] $bs[s^{AbsQ}_0] = \{ s^{AbsQ_0}_0 \}$.

\item[\textsc{call-enq}] Let $s \xrightarrow{inv(enq,d,k)}_{AbsQ} s'$ and $t' \in bs[s']$. Either $k \in D_{t'}$ or not. 

First consider the former case. We know that $\ell_{s'}(k) = (d, \texttt{PEND})$ and $k$ is maximal in $s'$. Hence $\sigma_{t'} = \rho \circ \langle d \rangle \circ \pi$ where $\pi$ contains linearization of pending elements in $O_{s'}$. Then, pick $\sigma_t = \rho$. We can find such a $t \in bs[s]$ with $\sigma_t$. Let $(D, \prec)$ be the partial order that is used while constructing $\sigma_{t'}$ from $O_s$ and $<_s$. We can find $(D', \prec')$ for relating $s$ to $t$ such that $D'$ does not contain the values of pending elements that formed $\pi$ suffix of $s_{t'}$ and $d$ coming from linearization of $k \in O_{s'}$. 

One can also see that $t \xrightarrow{\alpha}_{AbsQ_0} t'$ where $\alpha = inv(enq,d,k), lin(enq,d,k), \\lin(enq,d_1,k_1), ..., lin(enq,d_j,k_j)$ such that $\pi = d_1,...,d_j$ and $k_1,...,k_j \in O_{s'}$ are the pending elements that are  linearized to form $\pi$. Note that $\alpha$ obeys the definition of normal backward simulation definition.

For the second case, pick $t$ such that $\sigma_t = \sigma_{t'}$. We can find a $t$ with $\sigma_t$ related to $s$ by $bs$ using the same $(D,\prec)$ partial order that is used while relating $s'$ to $t'$. $\ell_1(\texttt{COMP}(O_s)) \subseteq D$ holds because $\texttt{COMP}(O_s) = \texttt{COMP}(O_{s'})$.
%\item[$\langle ii-a-enq \rangle$] Let $(q, inv(enq,d,k), q') \in \delta_I$ and $t' \in bs[q']$. We know that $(k,d,\texttt{PENDING}$ is a maximal element in $so_{q'}$ and either $s_{t'}$ does not linearize it or $s_{t'} = \rho \circ \langle d \rangle \circ \pi$ where $\pi$ consists of only linearization of \texttt{PENDING} elements. For the first case $s_{t'} = s_t$ for some $t \in bs[q]$ and $(t,inv(enq,d,k),t') \in \delta_A$. For the latter case, $s_t = \rho$ for some $t \in bs[q]$ and $t \xrightarrow{a}_{L_A} t'$ where $a = inv(enq,d,k), lin(enq,d,k), lin(enq,d_1,k_1),...,lin(enq,d_j,k_j)$ where $\pi = d_1,...d_j$. The $f_t$ could be obtained easily by observing the sequence $a$ and it can be checked that such $t \in bs[q]$ exists.

\item[\textsc{call-deq}] Let $s \xrightarrow{inv(deq,d,k)}_{AbsQ} s'$ and $t' \in bs[s']$. Pick $t$ such that it is equal to $t'$ in every field except that $k \notin dom(cp^0_t)$. Then, $t \in bs[s]$ and $t \xrightarrow{inv(deq,d,k)_{AbsQ_0}} t'$.
%\item[$\langle ii-a-deq \rangle$] Let $(q, inv(deq,k), q') \in \delta_I$ and $t' \in bs[q']$. We know that $f_{t'}(k) = D_0$. We know that there is a $t \in bs[q]$ such that $s_t = s_{t'}$ (since $so_q = so_{q'}$), $f_t(k) = N$ and $(t,inv(deq,k),t') \in \delta_A$.

\item[\textsc{lin-deq1}]  Let $s \xrightarrow{lin(deq,d,k)}_{AbsQ} s'$, $t' \in bs[s']$ and $d \neq \texttt{EMPTY}$. We pick $t$ such that $\sigma_t = \langle d \rangle \circ \sigma_{t'}$. We first show that $t \in bs[s]$. Let $(D, \prec)$ be the partial order that is linearized to obtain $\sigma_{t'}$ and $k' \in O_s$ be the element such that ${\ell_s}_1(k') = d$. We know that $k'$ is minimal in $<_s$ due to the premise of the rule \textsc{lin-deq1}. Hence, we can obtain $(D', \prec')$ such that $D' = D \cup \{{\ell_s}_1(k')\}$ and $\sigma_t$ is a linearization of it.

In addition, $t \xrightarrow{lin(deq,d,k)}_{AbsQ_0} t'$. The action $lin(deq,d,k)$ is enabled in state $t$ since $d$ is the minimum element of $\sigma_t$. Note that the transition relating $t$ to $t'$ obeys the definition of normal forward simulation.

\item[\textsc{lin-deq2}]  Let $s \xrightarrow{lin(deq,\texttt{EMPTY},k)}_{AbsQ} s'$ and $t' \in bs[s']$. We pick $(D, \prec)$ for relating $s$ to $t$ such that $D = \emptyset$. Such a $D$ is a valid choice since all the elements $O_s$ are pending. Then, $\sigma_t = \langle \rangle$ is the only linearization of $(D, \prec)$. Hence, $lin(deq,\texttt{EMPTY},k)$ action is enabled in $AbsQ_0$ and $t \xrightarrow{lin(deq,\texttt{EMPTY},k)}_{AbsQ_0} t'$ holds.
%\item[$\langle ii-d \rangle$] Let $(q,lin(deq,d,k),q') \in \delta_I$ and $t' \in bs[q']$. First consider the case $d \neq \texttt{EMPTY}$. We have two cases: Either the node with operation id $k$ in the $so_q$ was \texttt{PENDING} or \texttt{CLOSED}. For both of the cases, we can find $t \in bs[q]$ such that $s_t = \langle d \rangle \circ s_{t'}$ and $f_t(k) = D_0$. Hence $(t,lin(deq,d,k),t') \in \delta_I$. Next, consider the case $d = \texttt{EMPTY}$. Then, we know that $s_{t'} = \langle \rangle$ and there exists $t \in bs[q]$ such that $s_t = \langle \rangle$ and $f_t = D_0$. Hence $(t,lin(deq,d,k),t') \in \delta_A$.

\item[\textsc{ret-enq1}]  Let $s \xrightarrow{ret(enq,k)}_{AbsQ} s'$, $\ell_s(k) = (d,\texttt{PEND})$ and $t' \in bs[s']$. Assume $(D, \prec)$ be the partial order of which linearization is $\sigma_{t'}$. Pick $D'=D$. Then, $\ell_1(\texttt{COMP}(O_s)) \subseteq D \subseteq \ell_1(O_s)$ holds since $\texttt{COMP}(O_s) = \texttt{COMP}(O_{s'}) \setminus \{k\}$ and $k \in \texttt{PEND}(O_s)$. Construct $t \in bs[s]$ such that $\sigma_t = \sigma_{t'}$ is obtained by linearizing the partial order $(D',\prec')$. Then, $t \xrightarrow{ret(enq,k)}_{AbsQ_0} t'$ holds and it is a valid action with respect to normal backward-simulation relation definition. 

\item[\textsc{ret-enq2}]  Let $s \xrightarrow{ret(enq,k)}_{AbsQ} s'$, $k \notin O_s$ and $t' \in bs[s']$. Since $O_s = O_{s'}$ and $\texttt{COMP}(O_s) = \texttt{COMP}(O_{s'})$, we can pick $D' = D$ where $(D, \prec)$ is the strict partial order such that $\sigma_{t'}$ is its linearization. Construct $t \in bs[s]$ such that $\sigma_t = \sigma_{t'}$ is obtained by linearizing the partial order $(D',\prec')$. Then, $t \xrightarrow{ret(enq,k)}_{AbsQ_0} t'$ holds and it is a valid action with respect to normal backward-simulation relation definition. 
%\item[$\langle ii-b-enq \rangle$] Let $(q,ret(enq,k),q') \in \delta_I$ and $t' \in bs[q']$. There are two possible cases: There exists a node $(k,d,\texttt{PENDING}) \in nd_q$ or for all nodes $n = (k,\_,\_)$, $n \notin nd_q$. For the former case, $s_{t'} = \rho \circ \langle d \rangle \circ \pi$ where $\rho$ is linearization of closed nodes in $nd_{q'}$ and $\pi$ is linearization of some open nodes in $node_{q'}$. We also know that $f_{t'}(k) = E_2$. Then, there exists a node $t \in bs[q]$ such that $s_t = s_{t'}$ (since nodes that generate $\rho$ are closed in $q$ and that generate $\pi$ are open in $q$) and $f_t(k) = E_1$. Hence, $(t,ret(deq,k),t') \in \delta_A$. For the latter case, we know that $so_q = so_{q'}$. Therefore, there exists $t \in bs[q]$ such that $s_t = s_{t'}$. Moreover $f_t(k) = E_1$. Hence, $(t,ret(enq,k),t') \in \delta_A$.

\item[\textsc{ret-deq}]  Let $s \xrightarrow{ret(deq,d,k)}_{AbsQ} s'$ and $t' \in bs[s']$. Assume $(D, prec)$ is the partial order of which linearization is $\sigma_{t'}$. Construct $t \in bs[s]$ such that $\sigma_t = \sigma_{t'}$ and $(D, \prec
)$ is the partial order $\sigma_t$ is obtained from. $\texttt{COMP}(O_s) \subseteq D \subseteq O_s$ since $\texttt{COMP}(O_s) = \texttt{COMP}(O_{s'})$ and $O_s = O_{s'}$. Then, $t \xrightarrow{ret(deq,d,k)}_{AbsQ_0} t'$ holds. We have $rv_s(k) = rv^0_t(k)$ since $t \in bs[s]$. Hence the $ret(deq,d,k)$ is enabled in $t$. Moreover, $ret(deq,d,k)$ is a valid transition with respect to the normal backward simulation relation definition.
%\item[$\langle ii-b-deq \rangle$] Let $(q,ret(deq,d,k), q') \in \delta_I$ and $t' \in bs[q']$. Then, we know that $f_{t'}(k) = D_2$ and there exists a $t \in bs[q]$ such that $s_t = s_{t'}$ (since $so_q = so_{q'})$ and $f_t(k) = D_1$. Then, $(t,ret(deq,d,k) \in \delta_A$.
\end{itemize}
\end{proof}

%\begin{lem}
%$L_A$ is a q-refinement of $L_I$.
%\end{lem}
%\begin{proof}
\begin{lemma} 
$AbsQ_0$ is a refinement of $AbsQ$.
\end{lemma}
\begin{proof}
We define a normal $C\cup R\cup Lin(deq)$-forward simulation $fs$ from $AbsQ_0$ to $AbsQ$ as follows. 
Given $AbsQ_0$ state $t=\tup{\sigma,in^0,rv^0,cp^0}$ and an $AbsQ$ state $s=\tup{O,<,\ell,rv,cp}$ we have that $(t,s)\in fs$ iff the following hold:
\begin{itemize}
	\item the sequence $\sigma$ is a linearization of a partial order $(D,\prec)$ where $D$ contains values labeling elements of $O$ and all the values corresponding to completed enqueues, i.e., $\ell_1({\tt COMP}(O))\subseteq D\subseteq \ell_1(O)$ ordered according to the happens-before order between the enqueues that added them, i.e., $d_1\prec d_2$ if{f} there exists $k_1,k_2$ such that $\ell_1(k_1)=d_1$, $\ell_1(k_2)=d_2$, and $k_1 < k_2$.
	\item every dequeue is at the same control point in both $s$ and $t$, i.e., for every $k$ and $i\in \{1,2,3\}$, $cp(k)=R_i$ iff $cp^0(k)=R_i$,
	\item every enqueue is pending in $s$ whenever it is pending in $t$, i.e., for every $k$, $cp(k)=A_1$ iff $cp^0(k)\in \{A_1,A\}$,
	\item every enqueue is completed in $s$ whenever it is completed in $t$, i.e., for every $k$, $cp(k)=A_2$ iff $cp^0(k) = A_2$,
	\item every pending enqueue which is not linearized or whose value is present in $\sigma$ is a member of $O$, i.e., for every $k$, 
	\begin{align*}
	&k\in O\land \ell(k)=(d,{\tt PEND})\mbox{ iff } \\
	&\hspace{2cm}(cp^0(k)=A_1\land in^0(k)=d)\vee (\exists i.\ \sigma_i =d\land cp^0(k)=A\land in^0(k)=d)
	\end{align*}
	\item every completed enqueue whose value is present in $\sigma$ is a member of $O$, i.e., for every $k$, 
	\begin{align*}
	&k\in O\land \ell(k)=(d,{\tt COMP})\mbox{ iff } \exists i.\ \sigma_i =d\land cp^0(k)=A_2\land in^0(k)=d
	\end{align*}
	\item pending enqueues are maximal in $<$, i.e., for every $k$ and $k'$, $k \not< k'$ if $\ell_2(k)={\tt PEND}$,
	\item the return values fixed at dequeue linearization points are the same, i.e., for every $k$, $rv(k)=rv^0(k)$.
\end{itemize}
In the following, we show that indeed $fs$ is a normal $C\cup R\cup Lin(deq)$-forward simulation from $AbsQ_0$ to $AbsQ$.

%TODO FILL IN THE STEPS
%Since $L_I$ is deterministic with respect to the alphabet $AQ\Sigma$, we should be able to find a forward simulation between $L_A$ and $L_I$ if our claim is correct. We propose a forward simulation by adding some auxiliary varibles to the state of $L_A$. In the new augmented state of $L_A$, the sequence does not only keep the elements of the queue but also the operation identifiers that enqueued them. Hence, the sequence consists of  pairs of the form $s_q(i) = (k,d)$ where $q \in Q_A$, $i \in \mathbb{N}$ is an index, $d \in \mathcal{D}\backslash \{ \texttt{EMPTY} \}$ is a value and $k \in \mathbb{N}$ is an operation identifier. We also add a set $InvEnq_q$ component to state that keeps the enqueue operations at the point $E_0$ i.e. enqueues that are invoked bu have not been linearized yet. Elements of $InvEnq_q$ are pairs of the form $(d,k)$ where  $d \in \mathcal{D}\backslash \{ \texttt{EMPTY} \}$ is a value and $k \in \mathbb{N}$ is an operation identifier. Then our forward simulation $fs \subseteq Q_A \times Q_I$ relates the state $q = (s_q, f_q) \in Q_A$ to a state $q' =(so_{q'}, f_{q'}) \in Q_I$ iff (i) $f_{q'}(k) = f_q(k)$ for all $f_q(k) \in \{N, D_0, D_1, D_2\}$, (ii) $f_q(k) = E_0$ or $f_q(k) = E_1$ implies $f_{q'}(k) = E_0$, (iii) $f_q(k) = E_2$ implies $f_{q'}(k) = E_1$, (iv) If $(d,k) \in InvEnq_q$ or there exists an index $i \in \mathbb{N}$ such that $s_q(i) = (k,d)$ and $f_q(k) = E_1$, then $(k,d,\texttt{PENDING})$ is a node in $so_{q'}$, (v) if there exists an index $i \in \mathbb{N}$ such that $s_q(i) = (k,d)$ and $f_q(k) = E_2$ then $(k,d,\texttt{CLOSED})$ is a node in $so_{q'}$, (vi) open nodes in $so_{q'}$ are maximal and (vii) there exists a linearization (total order) $lo_{q'}$ of $so_{q'}$ that may omit some open nodes of $so_{q'}$ such that if we project nodes of $lo_{q'}$ to the first two fields, this linear order is equal to the sequence $s_q$.
%Next, we will show that $fs$ is a forward simulation relation:

\begin{itemize}
\item[$\langle i \rangle$] $fs[s_0^{AbsQ_0}] = \{s_0^{AbsQ}\}$
\item[\textsc{call-enq}] Let $t \xrightarrow{inv(enq,d,k)}_{AbsQ_0} t'$ and $s \in fs[t]$. Then, $inv(enq,d,k)$ is an enabled action in $AbsQ$ since premise of \textsc{call-enq} holds in $t$ and $s \in fs[t]$. Obtain $s'$ such that $s \xrightarrow{inv(enq,d,k)}_{AbsQ} s'$. Note that $s'$ is unique since $AbsQ$ is deterministic with respect to $C \cup R \cup Lin(deq)$. 

Next, we show that $s' \in fs[t']$. Let $(D, \prec)$ be the partial order used while relating $t$ to $s$. Same partial order can be used while relating $\sigma_{s'}$ to $t'$ since $\texttt{COMP}(O_s) = \texttt{COMP}(O_{s'}$, $O_s \subseteq O_{s'}$ and $<_s \subseteq <_{s'}$. The only change we have in control point fields after the actions is that $cp^0_{s'}(k)= A_1$ and $cp_{t'}(k)=A_1$ which satisfies the conditions on $fs$. Moreover $k$ is a maximal pending node in $t'$ as required by the $fs$ conditions. Consequently, $s' \in fs[t']$.
%\item[$\langle ii-a-enq \rangle$] Let $(q,inv(enq,d,k),q') \in \delta_A$ and $t \in fs[q]$. We can obtain $t'$ such that $(t,inv(enq,d,k),t') \in \delta_I$. We show that $t' \in fs[q']$ by checking seven conditions of our $fs$ relation. We omit the node $(k,d)$ while linearizing $so_{t'}$ and obtain $s_{q'}$.

\item[\textsc{call-deq}] Let $t \xrightarrow{inv(deq,k)}_{AbsQ_0} t'$ and $s \in fs[t]$. Then, $inv(deq,k)$ is an enabled action in $AbsQ$ since premise of \textsc{call-deq} holds in $t$ and $s \in fs[t]$. Obtain $s'$ such that $s \xrightarrow{inv(deq,k)}_{AbsQ} s'$. Note that $s'$ is unique since $AbsQ$ is deterministic with respect to $C \cup R \cup Lin(deq)$. 

Next, we show that $s' \in fs[t']$. Since $\sigma_s =\sigma_{s'}$, $O_s = O_{s'}$ and $\texttt{COMP}(O_s) = \texttt{COMP}(O_{s'}$, we can pick same $(D,\prec)$  partial order in $s'$ and show that $\sigma_{t'}$ is a linearization of it. The only change in control points after the transitions is that $cp^0_{t'}(k) = cp_{s'}(k) = R_1$ which does not violate any condition in $fs$. Consequently, $s' \in fs[t']$. 
%\item[$\langle ii-a-deq \rangle$] Let $(q, inv(deq,k), q') \in \delta_A$ and $t \in fs[q]$. We obtain $t'$ such that $(t, inv(deq,k),t') \in \delta_I$. The only difference between $t'$ and $t$ is that $f_{t'}(k) = D_0$. We can again see that $t' \in fs[q']$. 

\item[\textsc{lin-enq}] Let $t \xrightarrow{lin(enq,d,k)}_{AbsQ_0} t'$ and $s \in fs[t]$. Then, pick $s' = s$ such that $s \xrightarrow{\epsilon}_{AbsQ} s'$. Note that $\epsilon$ is a valid transition with respect to the normal forward simulation relation definition. We show that $s \in fs[t']$. If $(D, \prec)$ is the partial order in $s$ of which one linearization is $\sigma_t$, we pick $D' = D \cup \{k\} \subseteq O_s$. $(D', \prec')$ can be linearized to $\sigma_{t'}$ since $k$ is a maximal pending node and can be linearized at the end. Moreover, the only change in control point $cp^0_{t'}(k) = A$ which does not violate the $fs$ conditions.
%\item[$\langle ii-c \rangle$] Let $(q,lin(enq,d,k),q') \in \delta_A$ and $t \in fs[q]$. Pick $t' = t$. We can see that $t \in fs[q']$. While obtaining linearization of $so_{t'}$, we do not neglect the node $(k,d, \texttt{PENDING})$ this time, although we neglect it in the linearization of $so_t$.

\item[\textsc{lin-deq1}] Let $t \xrightarrow{lin(deq,d,k)}_{AbsQ_0} t'$, $d \neq \texttt{EMPTY}$ and $s \in fs[t]$. Then, $lin(deq,d,k)$ is an enabled action in $AbsQ$. There must exist $d \in D \subseteq \ell_1(O_s)$ such that ${\ell_s}_1(k') = d$ and $k'$ is minimal in $D$ (since $d$ is linearized as the minimum element in $\sigma_t$ according to premise of \textsc{lin-deq1} of $AbsQ$). Obtain $s'$ such that $s \xrightarrow{lin(deq,d,k)}_{AbsQ} s'$. Note that $s'$ is unique since $AbsQ$ is deterministic with respect to $C \cup R \cup Lin(deq)$. 

Next, we show that $s' \in fs[t']$. Let $(D, \prec)$ be the partial order used while relating $t$ to $s$ such that $\sigma_t$ is a linearization of this partial order. Since we have shown that $k'$ is minimal in that partial order, $\sigma_{t'}$ is a linearization of $(D', \prec')$ where $D' = D \setminus \{\ell_1(k')\}$. Note that $\ell_1(\texttt{COMP}(O_{s'})) \subseteq D' \subseteq \ell_1(O_{s'})$ holds. The only change in control points is that $cp^0_{t'}(k) = cp_{s'}(k) = R_2$ which does not violate the conditions for relating $t'$ to $s'$. Note that the fifth condition of $fs$ still holds for $k'$ while relating $t'$ to $s'$. After transitions $rv^0_{t'}(k) = rv_{s'}(k) = d$ and the last condition on $fs$ is preserved.

\item[\textsc{lin-deq2}] Let $t \xrightarrow{lin(deq,\texttt{EMPTY},k)}_{AbsQ_0} t'$ and $s \in fs[t]$. Then, $lin(deq,d,k)$ is an enabled action in $AbsQ$. If $\texttt{COMP}(O_t) \neq \emptyset$, then $D$ use for linearization cannot be $\emptyset$ $\sigma_t = \langle\rangle$ cannot be a linearization of $(D, \prec)$. Obtain $s'$ such that $s \xrightarrow{lin(deq,\texttt{EMPTY},k)}_{AbsQ} s'$. Note that $s'$ is unique since $AbsQ$ is deterministic with respect to $C \cup R \cup Lin(deq)$. 

Next, we show that $s' \in fs[t']$. Let $(D, \prec)$ be the partial order used while relating $t$ to $s$ such that $\sigma_t = \langle \rangle $ is a linearization of this partial order. We can use the same $(D, \prec)$ for relating $t'$ to $s'$ because $\sigma$ field is the same for both $s$, $s'$; and $O$, $<$, $\ell$ fields are same for both $t$ and $t'$. The only change in control points is that $cp^0_{t'}(k) = cp_{s'}(k) = R_2$ which does not violate the conditions for relating $t'$ to $s'$.  After transitions $rv^0_{t'}(k) = rv_{s'}(k) = d$ and the last condition on $fs$ is preserved.
%\item[$\langle ii-d \rangle$] Let $(q, lin(deq,d,k), q') \in \delta_A$ and $t \in fs[q]$. Obtain $t'$ such that $(t, lin(deq,d,k), t') \in \delta_I$. In $s_q$, $(k,d)$ must be the minimum element. Hence, $(k,d,\_)$ is a minimal node in $so_t$ and $lin(deq,d,k)$ is an enabled action in $L_I$. This action removes the node $(k,d,\_)$ from $so_t$ and changes $f_t(k)=D_0$ to $f_{t'}(k)= D_1$. We can see that $t' \in fs[q']$ by checking the seven conditions.

\item[\textsc{ret-enq}] Let $t \xrightarrow{ret(enq,k)}_{AbsQ_0} t'$ and $s \in fs[t]$. Then, there are two cases assuming data independence: (i) $in^0_t(k)=d$ and $\exists i. \sigma_t(i)= d$ (ii) or not.   

First, consider the former case. Then, \textsc{ret-enq1} rule of $AbSQ$ is applicable. Its precondition holds since fifth condition of $fs$ holds while relating $t$ to $s$. Apply this rule ($ret(enq,k)$) to obtain $s'$. Note that $s'$ is unique since $AbsQ$ is deterministic with respect to $C \cup R \cup Lin(deq)$ and it is a valid action according to the normal forward-simulation relation definition.

Next, we show that $s' \in fs[t']$. Since $\exists i. \sigma_t(i) = d$, we know that $d \in D$ where $(D, \prec)$ is the partial order satisfying first condition of $fs$ while relating $t$ to $s$, and $k \in O_s$ takes part in the linearization i.e., ${\ell_t}_1(k) \in D$. We can use the same partial order $(D,\prec)$ for relating $t'$ to $s'$ such that it satisfies the first condition of $fs$. The only change in control points is that $cp^0_{t'}(k) = cp_{s'}(k) = A_2$ which does not violate the conditions for relating $t'$ to $s'$. Note that the sixth condition of $fs$ also continue to hold for $k$ for the post-states.

Second, consider the latter case:  $in^0_t(k)=d$, but $\forall i. \sigma_t(i)\neq d$. Since $(t,s) \in fs$, $k \notin O_s$ by the fifth and sixth conditions. Hence, the pre-condition of \textsc{ret-enq2} is satisfied by $t$. Apply this rule ($ret(enq,k)$) to obtain $s'$. Note that $s'$ is unique since $AbsQ$ is deterministic with respect to $C \cup R \cup Lin(deq)$ and it is a valid action according to the normal forward-simulation relation definition.

Next, we show that $s' \in fs[t']$. For satisfying the first condition, one can use the same $(D, \prec)$ partial order that is used for relating pre-states since $\sigma$ fields of $t$, $t'$ and $O$, $<$, $\ell$ fields of $s$ and $s'$ are the same.  The only change in control points is that $cp^0_{t'}(k) = cp_{s'}(k) = A_2$ which does not violate the conditions for relating $t'$ to $s'$.
%\item[$\langle ii-b-enq \rangle$] Let $(q, ret(enq,k), q') \in \delta_A$ and $t \in fs[q]$. There are two cases: Either there exists an index $i \in \mathbb{N}$ such that $s_q(i) = (k,d)$ or there is no such $i$. For the former case, there is no node of the form $(k,d,\_)$ in $so_t$. We obtain $t'$ such that $(t,ret(enq,k),t') \in \delta_I$. Then, there is no node of the form $(k,d,\_)$ in $so_{t'}$ neither and $so_{t'} = so_t$. since $s_q = s_{q'}$ also holds for this case, $t' \in fs[q']$ holds. For the latter case, we again pick $t'$ such that $(t, ret(enq,k), t') \in \delta_I$. Again, $so_t = so_{t'}$ and $s_q = s_{q'}$ holds and $t' \in fs[q']$ can be observed.

\item[\textsc{ret-deq}] Let $t \xrightarrow{ret(deq,d,k)}_{AbsQ_0} t'$ and $s \in fs[t]$. Then, $ret(deq,d,k)$ is an enabled action in $AbsQ$ due to premise \textsc{ret-deq} of $AbsQ_0$ and the last condition on $fs$ (since $(t,s) \in fs$). Obtain $s'$ such that $s \xrightarrow{ret(deq,d,k)}_{AbsQ} s'$. Note that $s'$ is unique since $AbsQ$ is deterministic with respect to $C \cup R \cup Lin(deq)$.

We see that $s' \in fs[t']$. Pre-states are equal to the post-states with the only exception in the control points such that $cp^0_{t'}(k) = cp_{s'}(k) = R_3$. All the conditions except the third one continues to hold in the post states since they hold in the pre-states. The third rule regarding the control points of dequeues also continue to the hold since changes in the control point of $k$ does not violate it.
%\item[$\langle ii-b-deq \rangle$] Let $(q, ret(deq,d,k), q') \in \delta_A$ and $t \in fs[q]$. We pick $t'$ such that $(t, ret(deq,d,k), t') \in \delta_I$. We know that $so_t = so_{t'}$ and only change is $f_{t'}(k) = D_2$. Since we know that $s_q = s_{q'}$ and $f_{q'}(k) = D_2$, $t' \in fs[q']$ holds.  
\end{itemize}
\end{proof}
\input{app-absImplStack}
%!TEX root = draft.tex
\section{Proving the correctness of $TSS$}\label{app:tss}

%First, we describe the TS-Stack algorithm:
%\begin{lstlisting}
%struct Node{
%  int data;
%  int ts;
%  Node* next;
%  boolean taken;
%};
%Node* pools[maxThreads];
%int TS = 0;   
%
%void push(int x) {
%  Node* n = new Node(x,MAX_INT,
%                        null,false);
%  n->next = pools[myTID];
%  pools[myTID] = n;
%  int i = TS++;
%  n->ts = i;
%}
%int pop() {
% boolean success = false;
% int maxTS = -1;
% Node* youngest = null;
% while ( !success ) {
%   maxTS = -1; youngest = null;
%   for(int i=0; i<maxThreads; i++){
%     Node* n = pools[i];
%     while (n->taken && n->next != n)
%       n = n->next;
%     if(maxTS < n->ts) {
%       maxTS = n->ts; youngest = n;
%     }
%   }
%   if (youngest != null)
%     success=CAS(youngest->taken,
%                       false,true);
% }
% return youngest->data;
%}
%\end{lstlisting}
\begin{figure}[t]
\centering
\includegraphics[width=7cm]{tssPushFlow.pdf}
\includegraphics[width=16cm]{tssPopFlow.pdf}
\vspace{-8mm}
\caption{The flow diagram for the pop and push methods of the Time-Stamped Stack algorithm. The blue points show the control points roughly and the arrows show the possible transitions.}
\label{fig:tssFlow}
\end{figure}



The LTS corresponding to the description of $TSS$ given in Fig.~\ref{fig:TimeStamped} is defined as usual. The control points and transition labels we use in the following proof are pictured in Fig.~\ref{fig:tssFlow}. To simplify the proof, we take the initializations of some local variables together as atomic.

States of the TS-Stack contains the global variables and local variables as fields. Global variables are just elements of their domains and local variables are maps from operation identifiers to their domains. We say $i_q(k)$ for referencing the value of local variable $i$ of operation $k$ in state $q$. There is only one special local variable called $myTID$. Its value is unique to each pending operation in a state i.e., concurrent operations cannot have the same $myTID$ value. TS-Stack states also contains sets $O_a, O_r \in \mathbb{O}$ which are operation identifier sets of push and pops respectively, and the control point function $cp$ which is a map from operation identifiers to the control points set that are presented in the flow diagram Figure ~\ref{fig:tssFlow}. Transition relation of the TS-STack is presented in Figure~\ref{fig:transitions:TSSPush} (push rules) and Figure~\ref{fig:transitions:TSSPop} (pop rules).
Next, we show that the linearizability of TS Stack.

\begin{figure} [t]
{\scriptsize
  \centering
  \begin{mathpar}
    \inferrule[call-push]{
      k\not\in dom(cp) \\ 
      d\neq {\tt null}
    }{
      ..., O_a, x, cp,...
      \xrightarrow{inv(push,d,k)} 
     ..., O_a\cup\{k\}, x[k \mapsto d], cp[k \mapsto A_1], ...
    }\hspace{5mm}

    \inferrule[push1]{
      cp(k) = A_1 \\ 
      *n' = (x(k),\texttt{MAX\_INT}, \texttt{null}, \texttt{false})
    }{
      ..., n, cp,...
      \xrightarrow{push1(k)} 
     ..., n[k \mapsto n'], cp[k \mapsto A_2], ...
    }\hspace{5mm}
    
  
  \inferrule[push2]{
      cp(k) = A_2 
    }{
      ..., pools, cp,...
      \xrightarrow{push2(k)} 
     ..., pools[myTid(k) \mapsto n(k)], cp[k \mapsto A_3], ...
    }\hspace{5mm}
    
      \inferrule[push3]{
      cp(k) = A_3 
    }{
      ..., i, TS, cp,...
      \xrightarrow{push3(k)} 
     ..., i[k \mapsto TS], TS+1, cp[k \mapsto A_4], ...
    }\hspace{5mm}
    
   \inferrule[push4]{
      cp(k) = A_4 \\
      n'(k) = n(k)[ts \mapsto i(k)] \\
      \forall k'. cp(k') = A_6 \implies i(k') < i(k)
    }{
      ..., n, cp,...
      \xrightarrow{push4(k)} 
     ..., n[k \mapsto n'(k)], cp[k \mapsto A_5], ...
    }\hspace{5mm}
    
   \inferrule[ret-push]{
      cp(k) = A_5 
    }{
      ...,cp,...
      \xrightarrow{ret(push,k)} 
     ...,cp[k \mapsto A_6], ...
    }\hspace{5mm}
    
      \end{mathpar}
  }
 \vspace{-5mm}
  \caption{The push derivation rules of $TSS$. We only mention the state components that are modified. Unmentioned state components have the names in the algorithm in the prestate. $*n = (a,b,c,d)$ is shorthand for $n->data = a$, $n->ts = b$, ... $n' = n[ts \mapsto expr]$ is short for $n'->ts = expr$ and all the other fields of $n$ and $n'$ are the same.
  }
  \label{fig:transitions:TSSPush}
\vspace{-2mm}
\end{figure}

\begin{figure} [t]
{\scriptsize
  \centering
  \begin{mathpar}
    \inferrule[call-pop]{
      k\not\in dom(cp) 
    }{
      ..., O_r, cp,...
      \xrightarrow{inv(pop,k)} 
     ..., O_r\cup\{k\}, cp[k \mapsto R_1], ...
    }\hspace{5mm}
    
    \inferrule[pop1]{
      cp(k) = R_1 \\
      maxThreads > 0
    }{
      ..., suc, ygst, mTS, i, cp
      \xrightarrow{pop1(k)} 
     ..., suc[k \mapsto \texttt{false}], ygst[k \mapsto \texttt{null}], mTS[k \mapsto -1], i[k \mapsto 0],  cp[k \mapsto R_2]
    }\hspace{5mm}
    
    \inferrule[pop2]{
      cp(k) = R_2 \\
      0 \leq i(k) < \texttt{maxThreads}
    }{
      ..., n, cp, ...
      \xrightarrow{pop2(k)} 
     ..., n[k \mapsto pools(i(k))], cp[k \mapsto R_3],...
    }\hspace{5mm}
    
    \inferrule[pop3]{
      cp(k) = R_3 \\
      n(k) \neq \texttt{null}\\
      n(k)->taken = \texttt{true}\\
      n(k)->next \neq n(k)
    }{
      ..., n, ...
      \xrightarrow{pop3(k)} 
     ..., n[k \mapsto n(k)->next],...
    }\hspace{5mm}
    
    \inferrule[pop4]{
      cp(k) = R_3 \\
      n(k) \neq \texttt{null}\\
      n(k)->taken = \texttt{false}\\
      n(k)->ts > maxTS(k)
    }{
      ..., maxTS, cp ...
      \xrightarrow{pop4(k)} 
     ..., maxTS[k \mapsto n(k)->ts], cp[k \mapsto R_4]...
    }\hspace{5mm}
    
    \inferrule[pop5]{
      cp(k) = R_4 
    }{
      ..., youngest, cp ...
      \xrightarrow{pop5(k)} 
     ..., youngest[k \mapsto n(k)], cp[k \mapsto R_5]...
    }\hspace{5mm}
    
    \inferrule[pop6]{
      cp(k) = R_3 \\
      n(k) \neq \texttt{null}\\
      n(k)->taken = \texttt{false}\\
      n(k)->ts \leq maxTS(k)
    }{
      ..., cp, ...
      \xrightarrow{pop6(k)} 
     ...,cp[k \mapsto R_5],...
    }\hspace{5mm}
    
    \inferrule[pop7]{
      cp(k) = R_5 \\
      i(k) < \texttt{maxThreads}-1
    }{
      ...,i, cp, ...
      \xrightarrow{pop7(k)} 
     ...,i[k \mapsto i(k)+1], cp[k \mapsto R_2],...
    }\hspace{5mm}
    
    \inferrule[pop8]{
      cp(k) = R_5 \\
      youngest(k) = \texttt{null} \vee (youngest(k) \neq \texttt{null} \wedge youngest->taken)
    }{
      ...,success, cp, ...
      \xrightarrow{pop7(k)} 
     ...,success[k \mapsto \texttt{false}], cp[k \mapsto R_6],...
    }\hspace{5mm}
    
    \inferrule[com-pop]{
      cp(k) = R_5 \\
      youngest(k) \neq \texttt{null} \\
      youngest(k) = m \\
      d = m->data\\
      m->taken = false \\
      m' = m[taken \mapsto true]
    }{
      ...,success, youngest, cp, ...
      \xrightarrow{com(pop,d,k)} 
     ...,success[k\mapsto true], youngest[k \mapsto m'], cp[k \mapsto R_6],...
    }\hspace{5mm}
    
    \inferrule[pop9]{
      cp(k) = R_6 \\
      success(k) = \texttt{false}
    }{
      ..., cp, ...
      \xrightarrow{pop9(k)} 
     ..., cp[k \mapsto R_1],...
    }\hspace{5mm}
    \inferrule[ret-pop]{
      cp(k) = R_6 \\
      suc(k) = \texttt{false} \\ 
      d = yst(k)->data
    }{
      ..., cp, ...
      \xrightarrow{ret(pop,d,k)} 
     ..., cp[k \mapsto R_7],...
    }\hspace{5mm}
    
    


    \end{mathpar}
  }
 \vspace{-5mm}
  \caption{The pop derivation rules of $TSS$. We only mention the state components that are modified. Unmentioned state components have the names in the algorithm in the pre-state. $n' = n[taken \mapsto expr]$ is short for $n'->taken = expr$ and all the other fields of $n$ and $n'$ are the same.
  }
  \label{fig:transitions:TSSPop}
\vspace{-6mm}
\end{figure}

\begin{lem}
$TSS$ is a $C\cup R\cup Com(pop)$-refinement of $AbsS$. 
\end{lem}
\begin{proof}
We show that the relation $\mathit{fs}_2$ defined in Section~\ref{sec:corr_tss} is a $C\cup R\cup Com(pop)$-forward simulation  from $TSS$ to $AbsS$. For readability, we recall the definition of $\mathit{fs}_2$.

Let us make some clarifications before defining the relation. In order not to confuse nodes in TS Stack and nodes in $AbsS$, we call nodes of $AbsS$ as vertices from now on. We also define ordering relation (called traverse order) among the operations in a state of $TS$. It basically reflects the traverse order of pop operations. For two push operations $m,n \in O_a$ is state $s$ we say that $m <^{tr}_s n$ iff either $myTid(m) < myTid(n)$ or $myTid(m) = myTid(n)$ and $n_s(n)$ is reachable from $n_s(m)$ using next pointers. $\geq^{tr}$ is obtained from $<^{tr}$ in the usual way.

The relation $\mathit{fs}_2 \subseteq Q_C \rightarrow Q_{AbsS}$ contains $(s,t)$ iff the following are satisfied:
\begin{itemize}
\item[\emph{Nodes}] $k \in O_t$ iff $k$ is a push operation in $s$ ($k \in O_a$) such that either it has not inserted its node to the pool yet ( $cp_s(k) = A_i$ and $i<3$) or its node is not taken by a pop ($cp_s(k) = A_i$, $i\geq 3$ and $n_s(k)->taken = false$). 
\item[\emph{Pend/Comp}] A vertex $k \in O_t$ is pending ($\ell_t(k) = (d, \texttt{PEND})$) iff $k$ satisfies the previous condition, $x_s(k) = d$ and it is not completed in $s$ ($cp_s(k) = A_i$ and $i<6$). Similarly, this vertex is completed ($\ell_t(k) = (d, \texttt{COMP})$) iff $k$ satisfies the previous condition, $x_s(k) = d$ and it is completed in $s$ ($cp_s(k) = A_6$). Pending vertices are maximal with respect to $<_t$ i.e., if $k \in O_t$ is a pending vertex, then for all $k' \in O_t$ $k \nless_t k'$.
\item[\emph{TSOrder}] If a node has a smaller timestamp than the other node in $s$, the operations that inserted them cannot be ordered reversely in $t$. More formally, let $k, k' \in O_t$ s.t. $n_s(k)-> ts \leq n_s(k')->ts$. Then, $k' \nless_t k$.
\item[\emph{TidOrder}] Order among the nodes inserted by the same threads in $s$ must be preserved among the operations that inserted them in $t$. Let $k, k' \in O_t$ s.t. $myTid_s(k) = myTid_s(k')$ and $n_s(k)->ts < n_s(k')->ts$. Then, $k <_t k'$.
\item[Frontiers] Every maximally closed or pending vertex can be removed by a pending pop. More formally, let $k \in O_t$ such that $\ell_t(k) = (\_,\texttt{PEND})$. Then, for all pops $p$, $k \in ov_t(p)$. In the other case, let $k \in O_t$ such that $\ell_t(k) = (\_,\texttt{COMP})$ and for all other $k' \in O_t$ such that $k<_t k'$, we know $\ell_t(k') = (\_,\texttt{PEND})$. Then, for all pop operations $p$, $k \in be_t(p)$ or $k \in ov_t(p)$. 
\item[\emph{MaximalOV}] If a push $k \in O_t$ is a candidate to be removed by a pop $p$, then every other push $k'$ invoked after $k$ is a candidate to be removed by $p$ since $k$ is concurrent with $p$. More formally, let $k, k' \in O_t$ such that $k <_t k'$ and there exists a pop $p$ such that $k \in be_t(p)$ or $k \in ov_t(p)$. Then, $k' \in ov_t(p)$.
\item[\emph{MinimalBE}] If a push $k \in O_t$ has finished before the pop $p$ is invoked and yet $k$ is a candidate to be removed by $p$, other pushes completed before $k$ can not be candidates to be removed by $p$ at that state. More formally, let $k, k' \in O_t$ such that $k <_t k'$ and there exists a pop $p$ such that $k' \in be_t(p)$. Then, neither $k \in be_t(p)$ nor $k \in ov_t(b)$.
\item[\emph{ReverseFrontiers}] If all immediate followers $k' \in O_t$ of a push $k \in O_t$ are concurrent with pop $p$, then $k$ is either concurrent or maximally closed with respect to $p$. More formally, let $k \in O_t$ and for all $k' \in O_t$ such that $k \in pred_{<_t}(k')$, $k' \in ov_t(p)$, where $p$ is a pop operation. Then, $k \in ov_t(p) \cup be_t(p)$. 
%If a push $k \in O_t$ is concurrent with the pop $p$ and there exists another push $k' \in O_t$ that is immediate predecessor of $k$, then $k'$ is either concurrent or maximally closed with respect to $p$. More formally, let $k, k' \in O_t$ such that $k' \in pred_{<_t}(k)$ and $k \in ov_t(p)$ for some pop $p$. Then, either $k' \in ov_t(p)$ or $k' \in be_t(p)$. 
\item[\emph{FixReturn}] If a pop $p$ is after its commit point action in $s$, then the $rv$ value of this operation in $t$ is fixed to $youngest_s(p)->data$. More formally, Let $p$ be the pop operation such that $cp_s(p) = R_6$ and $success_s(p) = \texttt{true}$. Then, $rv_t(p) = youngest_s(p)->data$. 
\item[\emph{TraverseBefore}] If a pop operation $p$ is currently visiting node $n$, it has non-null node $y$ as the $youngest$ and there is a non-null not taken node $m$ coming before $n$ in the traverse order with a greater timestamp than $y$, then the operation that inserts $m$ must be concurrent with $p$. More formally, assume $youngest_s(p) = y$ and $ y \neq \texttt{null}$. Let $k \in O_t$ such that $n_s(k) \neq \texttt{null}$, $n_s(k)->taken = \texttt{false}$, $n_s(k) <^{tr}_s n_s(p)$ and $n_s(k)->ts \geq y->ts$. Then, $k \in ov_t(p)$.
\item[\emph{TraverseBeforeNull}] If a pop operation $p$ is currently visiting node $n$, and its $youngest$ field is \texttt{null}, then every other node $m$ coming before $n$ in the traverse order must be concurrent with $p$. More formally , let $youngest_s(p) = \texttt{null}$ and assume there exists  an operation $k \in O_t$ such that $n_s(k) \neq \texttt{null}$, $n_s(k)->taken = \texttt{false}$ and  $n_s(k) <^{tr}_s n_s(p)$. Then, $k \in ov_t(p)$. 
\item[\emph{TraverseAfter}] If a pop operation $p$ is currently visiting node $n$ that is not null and its youngest element $m$ is not null and still not taken in state $s$, then either $m$ is a candidate to be removed by $p$ in $t$ or there exists a later node $m'$ than $n$ such that $m'$ is a candidate in $t$ and it has a bigger timestamp than n. More formally, assume that there exists $k, k' \in O_t$ such that $youngest_s(p)->taken \neq false$, $youngest_s(p) = n_s(k)$ and $n_s(k') = n_s(p)$. Then, either $k \in ov_t(p) \vee k \in be_t(p)$ or there exists $k'' \in O_t$ s.t. $n_s(k'')->ts > n_s(k) ->ts$ and $k'' \in ov_t(p) \vee k'' \in be_t(p)$ and either $k' <^{tr}_s k''$ or $n_s(p) = n_s(k'') \wedge cp_s(p) = R_j \wedge j<5$. 
\end{itemize}
Next, we will show that $\mathit{fs}_2$ is really a Com(pop)-forward simulation relation. Except the trivial base case, we case-split on the transition rules. We first assume $(s, \alpha s') \in \delta_C$ and $t \in \mathit{fs}_2[s]$. Then, we find  corresponding transition $\alpha' \in \Sigma_{AbsS}$ obeying the Com(pop)-forward simulation relation conditions and obtain $t'$ such that $(t, \alpha' t') \in \delta_{AbsS}$  and $t' \in \mathit{fs}_2[s']$.

We observe that if $\alpha \in C \cup R \cup Com(pop)$, then the corresponding rule in $AbsS$ is $\alpha' = \alpha$. Otherwise, $\alpha' = \epsilon$.

Let the following derivation rule of $TSS$ be the one for describing $\alpha$:
\begin{mathpar}
    \inferrule{
      \psi
    }{
      s
      \xrightarrow{\alpha} 
      s'
    }
\end{mathpar}
and the following one be the derivation rule of $AbsS$ describing $\alpha'$ if $\alpha' \neq \epsilon$ (equivalently $\alpha' = \alpha$):
\begin{mathpar}
    \inferrule{
      \psi'
    }{
      q
      \xrightarrow{\alpha'} 
      q'
    }
\end{mathpar}
For the cases $\alpha' = \alpha$, we first need to show $\alpha'$ is enabled in state $t$ i.e., $t$ satisfies $\psi'$. If this can not be directly obtained from $s$ satisfies $\psi$ and using one or two obvious conditions on $\mathit{fs}_2$ (since $t \in \mathit{fs}_2[s]$), we show the derivation in the proof. Then, $t'$ is obtained in a unique way since $AbsS$ is deterministic on its alphabet $ \Sigma_{AbsS} = C \cup R \cup Com(pop)$. The, only other thing to show is $t' \in \mathit{fs}_2[t']$. We show this by proving that $t'$ does not violate any of the conditions of the $\mathit{fs}_2$ described above. We only explain why the new instantiations due to the difference between $s'$ and $s$ or the difference between $t'$ and $t$ do not violate the conditions. We skip the instances that we assumed while relating $s$ to $t$.

For the cases in which $\alpha' = \epsilon$, we have $t'=t$ and the only thing to show is $t \in \mathit{fs}_2[s']$. Again, we only explain why the new instantions due to the difference between $s'$ and $s$ do not violate the conditions.
\begin{itemize}
\item[\textsc{init}] $\mathit{fs}_2[{q_0}_{TSS}] =\{{q_0}_{AbsS}\}$
\item[\textsc{call-push}] The same derivation rule of $TSS$ is applied to $t$  to obtain $t'$. The premise of the rule is satisfied by $t$ trivially in the sense explained before. The new vertex $k$ is added to the $O_t$ such that $k$ is maximal, pending and every completed vertex is ordered before $k$ in $t'$. Moreover, $k$ is overlapping with every pending pop. To see that $t' \in \mathit{fs}_2[s']$ we observe the following: \emph{Nodes} condition is preserved because $k \in O_{t'}$. Since the newly added vertex $k$ is maximal and pending in $t'$, \emph{Pend/Comp} condition is preserved. \emph{Frontiers} and \emph{MaximalOV} conditions are not violated since $k$ is added to $ov(p)$ set for every pending pop operation $p$. 
\item[\textsc{push1}] We have $t' = t$ and show $t \in \mathit{fs}_2[s']$.\emph{Nodes} and \emph{Pend/Comp} conditions are still satisfied since $k$ remains to be a pending vertex. \emph{TSOrder} is still preserved. Timestamp of $n_{s'}(k)$ is maximal and every other nodes of push operations with maximal timestamp in $s'$ are pending vertices in $t$. Hence there can be no ordering between those pushes and $k$ in $t$ that can violate \emph{TSOrder}. Moreover, $k$ is maximal in $t$ which means that it cannot be ordered before another push $k'$ of which node has a lower timestamp. \emph{TidOrder} is also satisfied. Since $k$ is ordered after every completed push in $t$ and every other push by the same thread is completed, ordering required by the \emph{TidOrder} is present.
\item[\textsc{push2}] We have $t' = t$ and show $t \in \mathit{fs}_2[s']$. \emph{Nodes} and \emph{Pend/Comp} conditions are still satisfied since $k$ remains to be a pending vertex. One can also see that the \emph{TraverseBefore} condition is preserved. Let the pop $p$ visiting node $m$ and $n_{s'}(k) <^{tr}_{s'} m$. Since $k$ and $p$ are both pending in $s$ and $t \in \mathit{fs}_2[s]$, $k \in ov_t(p)$ (by the \emph{Frontiers} condition). Hence, \emph{TraverseBefore} is preserved. 
\item[\textsc{push3}] We have $t' = t$ and show $t \in \mathit{fs}_2[s']$. We consider two cases: $n_s(k)->taken$ is \texttt{true} or it is \texttt{false}. For the former case, $k \notin O_t$. The only new instantiation we check is $k \notin O_t$ does not violate \emph{Nodes} condition while relating $s'$ to $t$. 

For the latter case, we have $k \in O_t$. \emph{Nodes} and \emph{Pend/Comp} conditions are still satisfied since $k$ remains to be a pending vertex after changing $s$ to $s'$.
\item[\textsc{push4}] We have $t' = t$ and show $t \in \mathit{fs}_2[s']$. 

We consider two cases: $n_s(k)->taken$ is \texttt{true} or it is \texttt{false}. For the former case, \emph{Nodes} condition is still satisfied since $k$ remains to be not a vertex. 

For the latter case \emph{Nodes} and \emph{Pend/Comp} conditions are still satisfied since $k$ remains to be a pending vertex. \emph{TSOrder} condition is still not violated since if $k'<_t k$, then $k'$ is a completed vertex in $s$ and $s'$. By the premise of the rule (which can be shown to hold for every operation at control point $A_4$) $i_s(k') < i_s(k)$ and consequently $n_{s'}(k')->ts < n_{s'}(k)->ts$. Since every other push by the thread of $k$ is completed, \emph{TidOrder} still continues to hold for the same reasons. \emph{TraverseAfter} condition is also preserved. Let $k'$ be the push and $p$ be the pop such that $n_s(k') = youngest_s(p)$, $n_s(k') \leq^{tr}_s n_s(k)$, $n_s(k')->ts < n_s(k)->ts$ and $k \in ov_t(p)$ or $k \in be_t(p)$. Assume $n_{s'}(k')->ts \geq N_{s'}(k)->ts$ after the action. Then, $k'$ must be a pending push both in $s$ and $s'$ by the premise of the derivation rule and $k' \in ov_t(p)$ must be true by \emph{Frontiers} condition and $t \in \mathit{fs}_2[s]$. Hence, the \emph{TraverseAfter} condition is preserved.
\item[\textsc{ret-push}] We consider two cases, $n_s(k)->taken$ is \texttt{false} or \texttt{true}. For the former case, we obtain $t'$ by applying \textsc{ret-push1} rule of $AbsS$. \emph{Nodes} and \emph{Pend/Comp} conditions are still satisfied since $k$ becomes a completed vertex in $t'$. \emph{Frontiers} condition still holds since although $k$ become a maximally closed vertex in $t'$, we have $k \in ov_{t'}(p)$ for all pending nodes $p$ (due to \emph{Frontiers} condition, $t \in \mathit{fs}_2[s]$ and $k$ was a pending operation in state $t$, $k \in ov_t(p)$). 

For the latter case, we obtain $t'$ by applying \textsc{ret-push2} rule of $AbsS$. \emph{Nodes} condition is still satisfied since $k \notin O_{t'}$. 
\item[\textsc{call-pop}] The same derivation rule of $TSS$ is applied to $t$  to obtain $t'$. \emph{Frontiers} condition holds for $p = k$ relating $s'$ to $t'$ since $k' \in ov_{t'}(k)$ for every pending vertex $k'$ and $k'' \in be_{t'}(p)$ for all completed vertex $k''$. $t'$ due to action $inv(pop,k)$ applied on $t$. \emph{MaximalOV} condition holds for $p = k$ since pending vertices are maximal in $t'$ and for any maximally closed vertex $k'$ in $t'$, if $k'$ is ordered before other vertex $k''$, then $k''$ is a pending operation by definition of being maximally closed and $k'' \in ov_t(k)$ due to the changes by \textsc{inv-pop} action on $t$. \emph{MinimalBE} condition holds while relating $s'$ to $t'$ for the pop $ p = k$ because only maximally closed vertices are in $be(k)$ and if a push $k'$ is ordered before a maximally closed push $k''$ in $t$, neither $k'' \in be_{t'}(k)$ (since $k''$ is not maximally closed) nor $k'' \in ov_{t'}$ (since $k''$ cannot be pending). \emph{ReverseFrontiers} condition holds while relating $s'$ to $t'$ for the pop $p=k$ because, if $k'' \in ov_{t'}(k)$ for all immediate successors of $k'$ in $t$, then $k'' $ are pending vertices (due to \emph{call-pop} action of $AbsS$), $k'$ is a maximally closed vertex and $k' \in be_{t'}(k)$ (due to \emph{call-pop} action of $AbsS$).
\item[\textsc{pop1}]We have $t' = t$ and $t \in \mathit{fs}_2[s']$.
\item[\textsc{pop2}]We have $t' = t$ and show $t \in \mathit{fs}_2[s']$. \emph{TraverseBefore} condition while relating $s'$ to $t$ still holds for $p=k$. Assume $youngest_{s'}(k) = y$ is a non-null node. Then, for all nodes $m$ in $s'$ such that $n_s(k) \leq^{tr}_{s'} m <^{tr}_{s'} n_{s'}(k)$ we have $m->ts < y->ts$ in $s'$ because  $n_s(k)->ts > m->ts$ (since $n_s(k)$ is added to the pool after $m$ by the same thread) and $y->ts \geq n_s(k)->ts$ in $s'$ (since either $youngest_{s'}(k) = n_s(k)$ or $youngest_{s'}(k)->ts > n_s(k)->ts$). \emph{TraverseAfter} does not have any new instatiations since the guard mentions the nodes after $n_s(k)$ while relating $s$ to $t$ whereas it mentions nodes after or including $n_{s'}(k)$ which contains the all nodes in the former case.
\item[\textsc{pop3}]We have $t' = t$ and $t \in \mathit{fs}_2[s']$.
\item[\textsc{pop4}]We have $t' = t$ and $t \in \mathit{fs}_2[s']$.
\item[\textsc{pop5}]We have $t' = t$ and show $t \in \mathit{fs}_2[s']$. 
\emph{TraverseBefore} condition while relating $s'$ to $t$ still holds for $p=k$ since $youngest_s(k)->ts < youngest_{s'}(k)->ts$ and \emph{TraverseBefore} holds while relating $s$ to $t$. 

\emph{TraverseAfter} condition also continues to hold for $p=k$. There are two possible cases: $youngest_s(k) = \texttt{null}$ or not. 

First, consider the former case. Since \emph{TraverseBeforeNull} is satisfied while relating $s$ to $t$, for every operation $k', k'' \in O_t$ such that  $k'' <^{tr}_s k'$ and $n_s(k') = youngest_{s'}(k)$ we have $k'' \in ov_t(k)$. Consider all such $k''$ such that $n_s(k'')->ts > n_s(k')->ts$. If there exists such a $k''$ such that $k' \in pred_{<_t}(k'')$, then $k' \in ov_t(k) \cup be_t(k)$ since \emph{ReverseFrontiers} condition holds relating $s$ to $t$. Otherwise, either $k'$ is maximal in $t$ or all the vertices $v$ ordered after $k'$ in $t$ we have $v >^{tr}_s k'$. Then, either $k'$ or one of these $v$ vertices must be maximal in $t$ and must be in $be_t(k) \cup ov_t(k)$ since \emph{Frontiers} condition holds (one of them is maximal in $t$) while relating $s$ to $t$. 

Second, assume there exists push operations $j, k'$ such that $n_s(j) = youngest_s(k) \neq \texttt{null}$ and $n_s(k') = n_s(k) = youngest_{s'}(k)$ . Since \emph{TraverseBefore} is satisfied while relating $s$ to $t$, if there exists a push $k'' <^{tr}_s k'$ such that $n_s(k'')$ is not taken and $n_s(k'')->ts \geq n_s(j)->ts$, then $k'' \in ov_t(k)$. Then, for all $k'' <^{tr}_s k'$ such that $n_s(k'')$ is not taken and $n_s(k'')->ts \geq n_s(k')->ts$, then $k'' \in ov_t(k)$ since $n_s(k')->ts \geq n_s(j)->ts$. If there exists such a $k''$ such that $k' \in pred_{<_t}(k'')$, then $k' \in ov_t(k) \cup be_t(k)$ since \emph{ReverseFrontiers} condition holds relating $s$ to $t$. Otherwise, either $k'$ is maximal in $t$ or all the vertices $v$ ordered after $k'$ in $t$ we have $v >^{tr}_s k'$. Then, either $k'$ or one of these $v$ vertices must be maximal in $t$ and must be in $be_t(k) \cup ov_t(k)$ since \emph{Frontiers} condition holds (one of them is maximal in $t$) while relating $s$ to $t$. 

\item[\textsc{pop6}] We have $t' = t$ and show $t \in \mathit{fs}_2[s']$. \emph{TraverseAfter} continues to hold while relating $s'$ to $t$ for $p=k$. Let $k', k'' \in O_t$ such that $youngest_s(k) = n_s(k')$, $n_s(k) = n_s(k'')$ and $k' \notin  ov_t(k) \cup be_t(k)$. Note that $k' <^{tr}_s k''$. Then, $n_s(k'')->ts < n_s(k')->ts$ since $n_s(k'')->ts < maxTS(k)$ and $maxTS(k) = n_s(k')->TS$ ($n_s(k')->ts$ cannot be \texttt{MAX\_INT} since $k'$ would be pending and $k' \in ov_t(k)$ otherwise). Hence, there exists another push $j$ such that $j >^{tr}_s$ and $j \in ov_t(k) \cup be_t(k)$. 

\item[\textsc{pop7}] We have $t' = t$ and $t \in \mathit{fs}_2[s']$.
\item[\textsc{pop8}] We have $t' = t$ and $t \in \mathit{fs}_2[s']$.
\item[\textsc{com-pop}] $t'$ is obtained by applying \textsc{com-pop1} rule of $AbsS$.
We first show that precondition of \textsc{com-pop1} rule of $AbsS$ si satisfied by $t$. If $com(pop,d,k)$ removes a  node $n$ such that there exists a push $k'$ such that $n_s(k') =n$ in $s$, then $k' \in O_t$ since it is non-null and not taken. Moreover, $k' \in ov_t(k) \cup be_t(k)$ since \emph{TraverseAfter} is preserved while relating $s$ to $t$ and all the nodes that come after $n_s(k)$ in terms of traverse order in $s$ have lower timestamp values than $n_s(k)->ts$ and $n_s(k)->ts \leq youngest_s(k)->ts$.

Next, we show that $t' \in \mathit{fs}_2[s']$. We case split on the conditions of $\mathit{fs}_2$ considering new instantiations.

\emph{Nodes} condition is still preserved after $k$ removes the node pushed by operation $k'$ in $s$ since $k' \notin O_{t'}$ anymore by due to $com(pop,d,k)$ action. 

\emph{Frontiers} condition is still preserved if $k$ removes the vertex $k'$ and makes another $k''$ maximally closed in $t$. Since all the other nodes $j$ ordered after $k''$ (except possibly $k'$) in $t$ are pending, $j \in ov_t(p)$ (due to \emph{Frontiers} condition while relating $s$ to $t$) for some pending pop $p \neq k$. Then, $k'' \in be_{t'}(p)$ by $com(pop,d,k)$ action. 

For the \emph{MinimalBE} condition, we do not have a new instance. If $k' \in be_{t'}(p)$ becomes true although $k' \notin be_t(p)$, we cannot have $k'' \in O_{t'}$ such that $k' \in pred_{<_{t'}}(k'')$ and $k'' \in be_{t'}(p)$ since $com(pop,d,k)$ does not add $k''$ to $ov(p)$ if its successor is not pending with respect to $p$.

\emph{ReverseFrontiers} condition is still preserved. If $k$ removes the vertex $k'$ and there exists an immediate predecessor $k''$ of $k'$ such that all of immediate successors of $k''$ are in $ov_{t'}(p)$, then $k'' \in ov_{t'}(p)$ due to the action $com(pop,d,k)$.

\emph{TraverseAfter} condition is still preserved after $k$ removes the node of push $k'$. Let $p \neq k$ be another pop operation such that $n_s(j) = youngest_s(p)$ for some push $j$ and $n_s(k')$ be the only node such that $n_s(k')->ts > youngest_s(p)->ts$ and $n_s(k')$ comes after $n_s(p)$ in the traverse order of $s$ and $k' \in ov_t(p) \cup be_t(p)$. Hence, there is no $k''$ such that $n_s(k'')$ comes after $n_s(p)$ in the traverse order and $j <_t k''$ except $k'$ (i). In other direction, if for all $k'' \in O_t$ such that $n_s(k'')$ comes before $n_s(p)$ in the traverse order and $n_s(k'')->ts > youngest_s(p)->ts$ , then $k'' \in ov_t(p)$ since \emph{TraverseBefore} condition holds while relating $s$ to $t$. Then, for all $k'' \in O_t$ such that $n_s(k'')$ comes before $n_s(p)$ in the traverse order of $s$ and $k'' >_t j$ implies $k'' \in ov_t(p)$ since $n_s(k'')->ts > n_s(j)->ts$ if $k'' >_t j$ (ii). Then, for all $k'' \in O_t$ such that if $k'' >_t j$, then $k'' \in ov_t(p)$ except $k'$ due to (i) and (ii). If $j \nless_t k'$, then $j \in ov_t(p) \cup be_t(p)$ since \emph{ReverseFrontiers} hold while relating $s$ to $t$ and $j \in ov_{t'} \cup be_{t'}$ after applying the action $com(pop,d,k)$. Otherwise, if $j <_t k'$, then $k \in be_{t'}$ after applying $com(pop,d,k)$.

\emph{FixReturn} condition continues to hold. If $com(pop,d,k)$ removes the node pushed by $k'$ in $s$, then $com(pop,d,k)$ removes the vertex $k'$ (assuming data independece) and $youngest_s(k')->data = \ell_t(k')_1$. Then, $youngest_{s'}(p)->data = rv_{t'}(p)$ after applying commit actions at both sides.

\item[\textsc{pop9}]We have $t' = t$ and $t \in \mathit{fs}_2[s']$.
\item[\textsc{ret-pop}] $t'$ is obtained by applying \textsc{ret-pop} rule of $AbsS$ and $t' \in \mathit{fs}_2[s']$.
\end{itemize}
\end{proof}
%\textcolor{red}{TODO: Check ReverseFrontiers for the cases before compop.}
%\textcolor{red}{TODO: Check TraverseBefore for the cases before pop2.}
%\textcolor{red}{TODO: CheckTraverseBeforeNull for the cases befoere pop 5.}
%\textcolor{red}{TODO: Apply Dr. Enea's comments.}
%\textcolor{red}{TODO: Try case split on fs conditions instantiations check.}
%\textcolor{red}{TODO: Add Flow Diagrams to the paper.} 
%\textcolor{red}{TODO: }

\end{document}
