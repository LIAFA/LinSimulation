%!TEX root = draft.tex
\vspace{-3.5mm}
\section{Refinement Proofs}
\vspace{-1.5mm}
Library refinement is the instance of a more general notion of refinement between LTSs
which for some alphabet $\Gamma$ of \emph{observable actions} is defined as the inclusion of sets of 
traces projected on $\Gamma$. Library refinement corresponds to the case $\Gamma=C\cup R$. 
Typically, $\Gamma$-refinement between two LTSs $A_1$ and $A_2$ is proved using \emph{simulation relations} which roughly, require that 
$A_2$ can mimic every step of $A_1$ using a (possibly empty) sequence of steps. Mainly, there are two kinds of simulation
relations, forward or backward, depending on whether the preservation of steps is proved starting from a similar state
forward or backward. It has been shown
that $\Gamma$-refinement is equivalent to the existence of \emph{backward simulations}, modulo the addition of history variables
that record events in the implementation, and to the existence of \emph{forward simulations} provided that the right-hand side
LTS $A_2$ is $\Gamma$-deterministic~\cite{DBLP:journals/tcs/AbadiL91,DBLP:journals/iandc/LynchV95}. 
We focus on proofs based on forward simulations because they are easier to automatize.

%TODO SAY THAT THE EXTENSION TO $\Gamma$-determinism is new (the previous results were about determinism).

In general, forward simulations are \emph{not} a complete proof method for library refinement because libraries are not 
$C\cup R$-deterministic (the same sequence of call/return actions can lead to different states depending on the interleaving of the internal actions).
However, there are classes of atomic libraries, e.g., libraries with ``fixed linearization points'' (defined later in this section), 
for which it is possible to identify a larger alphabet $\Gamma$ of observable actions (including call/return actions), 
and implementations that are $\Gamma$-deterministic. For queues and stacks, 
Section~\ref{sec:queues} and Section~\ref{sec:stacks} define other such classes of implementations that cover
all the implementations that we are aware of.

Let $A_1=(Q_1,\Sigma, s_0^1, \delta_1)$ and $A_2=(Q_2,\Sigma, s_0^2, \delta_2)$ be two LTSs over $\Sigma_1$ and $\Sigma_2$, respectively, such that $\Gamma\subseteq \Sigma_1\cap\Sigma_2$. 

%Refinement between two libraries $L_1$ and $L_2$  can be proved using  In general, forward simulations are easier to establish, but they are incomplete, i.e., refinement may hold while
%there exists no forward simulation. On the other hand, proofs based on backward simulations are more difficult, but modulo the
%addition of history variables (auxiliary variables recording events in the implementation) backward simulations are complete for 
%proving refinement~\footnote{Adding history variables has no impact on the incompleteness of forward simulations.}. 

%This section introduces notions used to enlarge the scope of proofs based on forward simulations, which is the main goal of this paper.
%We first introduce a stronger notion of refinement that compares the set of traces of two libraries projected
%on a larger alphabet of actions than the set of call/return actions. These actions are called \emph{observable actions},
%and they may represent for instance, linearization points which informally, are points in time where an operation 
%is conceptually effectuated. We then define a notion of forward simulation which is complete for proving
%this notion of refinement provided that the alphabet of the ``abstract'' library $L_2$ equals the set of 
%observable actions and $L_2$ is deterministic. We discuss an application of this result, namely,
%proving the correctness of libraries with ``fixed linearization points'', where the so-called ``linearization points''
%are fixed to particular implementation actions.
\vspace{-1.5mm}
\begin{definition}\label{def:gref}
%Let $L_1$ and $L_2$ be two libraries over alphabets $\Sigma_1$ and $\Sigma_2$, respectively, such that $C\cup R \subseteq \Sigma_1\cap\Sigma_2$. Also, let $\Gamma$ be a set of actions such that $C\cup R\subseteq \Gamma\subseteq \Sigma_1\cap\Sigma_2$. 
The LTS $A_1$ \emph{$\Gamma$-refines} $A_2$ if{f} $Tr(A_1) | \Gamma \subseteq Tr(A_2) | \Gamma$.
%We say that $L_1$ \emph{refines} $L_2$ when $H(L_1) \subseteq H(L_2)$.
%for every trace $e \in Tr(L_1)$, there exists a trace $e' \in Tr(L_2)$ such that $e|A\Sigma$ is equivalent to $e'|A\Sigma$.
\vspace{-1.5mm}
\end{definition}

The notion of $\Gamma$-refinement instantiated to libraries (i.e., to LTSs defining libraries) implies the notion of refinement in Definition~\ref{def:lib_ref} for every $\Gamma$ such that $C\cup R \subseteq \Gamma$.
%, resp., such that $C\cup R \subseteq \Sigma_1\cap\Sigma_2$. Also, let $\Gamma$ be a set of actions s.t. $C\cup R\subseteq $
%
%When $A_1$ and $A_2$ are libraries, $\Gamma$-refinement implies the notion of library refinement  for any $\Gamma$ as in Definition~\ref{def:gref}.

We define a notion of \emph{forward} simulation that can be used to prove $\Gamma$-refinement %We define a notion of forward simulation that is a sound approximation of refinement and a notion of backward simulation that precisely characterizes refinement. 
For a relation $R\subseteq A\times B$, $R[X]$ is the set of elements related by $R$ to elements of $X$, i.e., $R[X]=\set{y:\exists x\in X.\ R(x,y)}$.

%We can extend the relations between libraries by introducing simulation relations. We will later show that simulation relations imply refinement. 
\vspace{-1.5mm}
\begin{definition}
%Let $L_1=(Q_1,\Sigma, s_0^1, \delta_1)$ and $L_2=(Q_2,\Sigma, s_0^2, \delta_2)$ be two libraries over alphabets $\Sigma_1$ and $\Sigma_2$, respectively, such that $C\cup R \subseteq \Sigma_1\cap\Sigma_2$, and $\Gamma$ a set of actions such that $C\cup R\subseteq \Gamma\subseteq \Sigma_1\cap\Sigma_2$. 
A relation $F \subseteq Q_{1} \times Q_{2}$ is called a \emph{$\Gamma$-forward simulation} from $A_1$ to $A_2$ if{f} $F(s_0^1,s_0^2)$ and:
\vspace{-1.5mm}
\begin{itemize}
\item For all $s,s'\in Q_1$, $\gamma\in \Gamma$, and $u\in Q_2$, such that $(s,\gamma,s') \in \delta_1$ and $F(s,u)$, we have that there exists $u'\in Q_2$ such that $F(s',u')$ and $u \xrightarrow{@s} u'$ where $@s_i=\gamma$, for some $i$, and $@s_j\in \Sigma_2\setminus\Gamma$, for all $j\neq i$.
\item For all $s,s'\in Q_1$, $e \in \Sigma_1\setminus \Gamma$, and $u\in Q_2$, such that $(s,e,s') \in \delta_1$ and $F(s,u)$, we have that there exists $u'\in Q_2$ such that $F(s',u')$ and $u \xrightarrow{\sigma} u'$ where $\sigma\in (\Sigma_2\setminus\Gamma)^*$.  
%where $a = a_1, a_2, ..., a_n$ such that for all $i \in [1,n]$, $a_i \in \Sigma_{L_2} \backslash A\Sigma $. Moreover, $a$ could be the empty sequence.
\end{itemize}
\vspace{-3.5mm}
\end{definition}
%If $\mathit{fs}[s]$ is a unique state for all $s \in Q_1$ then $\mathit{fs}$ is called a refinement mapping/function. 
A $\Gamma$-forward simulation requires that every step of $A_1$ corresponds to a sequence of steps of $A_2$. To imply $\Gamma$-refinement, every step of $A_1$ labeled by an observable action $\gamma\in \Gamma$ should be simulated by a sequence of steps of $A_2$ where exactly one transition is labeled by $\gamma$ and all the other transitions are labeled by non-observable actions.
The dual notion of \emph{backward} simulation where steps are simulated backwards can be defined in a similar way.




The following shows the soundness and the completeness of $\Gamma$-forward simulations (when $A_2$ is $\Gamma$-deterministic). It is an instantiation of previous results~~\cite{DBLP:journals/tcs/AbadiL91,DBLP:journals/iandc/LynchV95}.

\vspace{-1.5mm}
\begin{theorem}\label{th:forSim}
%Let $L_1$ and $L_2$ be two libraries over alphabets $\Sigma_1$ and $\Sigma_2$, resp., such that $C\cup R \subseteq \Sigma_1\cap\Sigma_2$. Also, let $\Gamma$ be a set of actions such that $C\cup R\subseteq \Gamma\subseteq \Sigma_1\cap\Sigma_2$. 
$A_1$ $\Gamma$-refines $A_2$ when there is a $\Gamma$-forward simulation from $A_1$ to $A_2$. Moreover, if $A_1$ $\Gamma$-refines $A_2$ and $A_2$ is $\Gamma$-deterministic, then there is a $\Gamma$-forward simulation from $A_1$ to $A_2$.
\vspace{-1.5mm}
\end{theorem}

The linearization of a concurrent history can be also defined in terms of \emph{linearization points}. Informally, a linearization point of 
an operation in an execution is a point in time where the operation is conceptually effectuated; given the linearization points of 
each operation, the linearization of a concurrent history is the sequential history which takes operations in order of their linearization points.
For some libraries, the linearization points correspond to a fixed set of actions. For instance, in the case of atomic libraries  
where method bodies are guarded with a global-lock acquisition, the linearization point of every method invocation corresponds to the execution 
of the body. When the linearization points are fixed, we assume that the library is an LTS over an alphabet that includes actions 
$lin(m,d,k)$ with $m\in\<Methods>$, $d\in\<Vals>$ and $k\in \<Ops>$. The action $lin(m,d,k)$ represents the linearization point of the operation $k$ 
%which is an invocation of method $m$ 
returning value $d$.
Let $Lin$ denote the set of such actions. 
%We extend the notion of
%well-formedness to sequences $@t$ over alphabet $\Sigma$,
%such that $C\cup R\cup Lin\subseteq \Sigma$, by adding the following constraints:
%\begin{itemize}
%	\item every linearization point corresponds to an operation that has been called, i.e., TODO
%	\item every linearization point occurs between the call and the return of the corresponding operation, i.e., TODO
%	\item the return value of an operation is fixed by the linearization point, i.e., TODO
%\end{itemize}
%We assume that libraries produce only well-formed traces. 
The projection of a library trace over $C\cup R\cup Lin$ is called an 
\emph{extended history}. A trace or extended history is called \emph{$Lin$-complete} when every completed operation has a linearization 
point, i.e., each return action $ret(m,d,k)$ is preceded by an action $lin(m,d,k)$. 
A library $L$ over alphabet $\Sigma$ is called \emph{with fixed linearization points} if{f} $C\cup R\cup Lin\subseteq \Sigma$ 
and every trace $@t\in Tr(L)$ is $Lin$-complete. 

Proving the correctness of an implementation $L_1$ of a concurrent object such as a queue or a stack with fixed linearization points
reduces to proving that $L_1$ is a $(C\cup R\cup Lin)$-refinement of an abstract implementation $L_2$ of the same object where method
bodies are guarded with a global-lock acquisition. As a direct consequence of Theorem~\ref{th:forSim}, since the abstract implementation is 
 $(C\cup R\cup Lin)$-deterministic, proving $(C\cup R\cup Lin)$-refinement is equivalent to finding a $(C\cup R\cup Lin)$-forward simulation 
from $L_1$ to $L_2$.

Section~\ref{sec:queues} and Section~\ref{sec:stacks} extend this result to queue and stack implementations where the linearization point of the methods 
\emph{adding} values to the collection is \emph{not} fixed.
