%!TEX root = draft.tex
\section{Libraries}\label{app:prelim}

Programs interact with libraries by calling named library \emph{methods}, which
receive \emph{parameter values} and yield \emph{return values} upon completion.
We fix arbitrary sets $\<Methods>$ and $\<Vals>$ of method names and
parameter/return values. 

%\begin{example}
%  \label{ex:methods}
%
%  TODO
%
%  The method and value sets for the stack implementation in
%  Figure~\ref{fig:treiber} are $\<Methods> = \set{ \<push>, \<pop> }$ and
%  $\<Vals> = \<Nats> \u \set{ {\tt EMPTY} }$.
%
%\end{example}

\noindent
We fix an arbitrary set $\<Ops>$ of operation identifiers, and for given sets
$\<Methods>$ and $\<Vals>$ of methods and values, we fix the sets
\begin{align*}
  & C = \set{ inv(m,d,k) : m \in \<Methods>, d \in \<Vals>, k \in \<Ops> }
  \text{, and } \\
  & R = \set{ ret(m,d,k) : m \in \<Methods>, d \in \<Vals>, k \in \<Ops> }  
\end{align*}
of \emph{call actions} and \emph{return actions}; each call action $inv(m,d,k)$
combines a method $m \in \<Methods>$ and value $d \in \<Vals>$ with an
\emph{operation identifier} $k \in \<Ops>$. Operation identifiers are used to
pair call and return actions. 
We assume every set of 
words is closed under isomorphic renaming of operation identifiers. 
We denote the operation identifier of a
call/return action $a$ by $\<op>(a)$. Call and return actions $c \in C$ and $r
\in R$ are \emph{matching}, written $c \match r$, when $\<op>(c) = \<op>(r)$. 
We may omit the second field from a call/return action $a$ for methods that have no inputs (e.g., the pop method of a stack) or return values (e.g., the push method of a stack).
A word $\tau \in @S^*$ over alphabet $@S$, such that $(C \u R) \subseteq @S$, is
\emph{well formed} when:
\begin{itemize}

  \item Each return is preceded by a matching call: \\
  $\tau_j \in R$ implies $\tau_i \match \tau_j$ for some $i < j$.

  \item Each operation identifier is used in at most one call/return: \\
  $\<op>(\tau_i) = \<op>(\tau_j)$ and $i < j$ implies $\tau_i \match \tau_j$.

\end{itemize}
We say that the well-formed word $\tau \in @S^*$ is \emph{sequential} when
\begin{itemize}

  \item Operations do not overlap: \\
  $\tau_i, \tau_k \in C$ and $i < k$ implies $\tau_i \match \tau_j$ for some $i < j < k$.

\end{itemize}
Well-formed words represent traces of a library. We assume every set of well-formed
words is closed under isomorphic renaming of operation identifiers. For
notational convenience, we take $\<Ops>=\<Nats>$ for the rest of the paper.
When the value of a certain field in a call/return action is not important we use 
the placeholder $\_$, e.g., $inv(m,\_,k)$ instead of $inv(m,d,k)$ when the input  
$d$ can take any value.

An operation $k$ is called \emph{completed} in a well-formed trace $\tau$ when
$ret(m,d,k)$ occurs in $\tau$, for some $m$ and $d$. Otherwise, it is called \emph{pending}.
%An operation $o$ of an execution $e$ is \emph{completed}
%when both call and return actions $m(u)_o$ and $\<ret>(v)_o$ of $o$ occur in
%$e$, and is otherwise \emph{pending}.

%\begin{example}
%  \label{ex:executions}
%
%  TODO
%
%  The well-formed words
%  \scriptsize
%  \begin{align*}
%     \<push>(0)_1\ \<pop>_2\ \<pop>_3\ \<ret>_1\ \<ret>(0)_3\ \<ret>(0)_2
%    \text{\normalsize, and } 
%    \<push>(0)_1\ \<pop>_2\ \<pop>_3\ \<ret>_1\ \<ret>(0)_2
%  \end{align*}
%  \normalsize
%  represent executions in which one call to the $\<push>(0)$ method overlaps
%  with two calls to $\<pop>$. In the first execution both calls to $\<pop>$
%  have matching return actions $\<ret>(0)$, i.e., the operations $2$ and $3$ are completed,
%  while operation $3$ is pending in the second, it has no matching return.
%
%\end{example}

Libraries dictate the execution of methods between their call and return
points. Accordingly, a library cannot prevent a method from being called,
though it can decide not to return. Furthermore, any library action performed
in the interval between call and return points can also be performed should the
call have been made earlier, and/or the return made later. 
A library thus allows any sequence of
invocations to its methods made by \emph{some} program.

\begin{definition}\label{def:libraries}
A \emph{library} $L$ is an LTS over alphabet $\Sigma$ such that $C \u R\subseteq \Sigma$
and each trace $\tau \in Tr(L)$ is well formed, and
  \begin{itemize}

    \item Call actions $c \in C$ cannot be disabled: \\
    $\tau \cdot \tau' \in Tr(L)$ implies $\tau \cdot c \cdot \tau' \in Tr(L)$
    if $\tau \cdot c \cdot \tau'$ is well formed.
  
    \item Call actions $c \in C$ cannot disable other actions: \\
    $\tau \cdot a \cdot c \cdot \tau' \in Tr(L)$ implies $\tau \cdot c \cdot a \cdot \tau' \in Tr(L)$.
  
    \item Return actions $r \in R$ cannot enable other actions: \\
    $\tau \cdot r \cdot a \cdot \tau' \in Tr(L)$ implies $\tau \cdot a \cdot r \cdot \tau' \in Tr(L)$.
  
  \end{itemize}

\end{definition}

Note that even a library that implements \emph{atomic methods}, e.g.,~by
guarding method bodies with a global-lock acquisition, admits executions in
which method calls and returns overlap. 
For simplicity, Definition~\ref{def:libraries} assumes that every thread performs a single operation. The extension to multiple operations per thread is straightforward, e.g. the closure rules must assume that the actions $a$ and $c$ belong to different threads
%A library which accesses the client's
%thread identifiers can be modeled by taking thread identifiers as method
%parameters.

%\textcolor{red}{For the below paragraph, I cannot see the equivalence  of informal explanation and formal definition of weakening. I think the formal one should be if a ret comes before a call in $h_1'$ then this order is preserved in $h_2$. But the definition says iff. Ignore this if I am wrong.}


%\begin{example}
%  \label{ex:libraries}
%
%  TODO (replace executions with histories)
%
%  Any library which admits the execution
%  \scriptsize
%  \begin{align*}
%    \<push>(0)_1\ \<ret>_1\ \<pop>_2\ \<ret>(0)_2
%  \end{align*}
%  \normalsize
%  with sequential calls to $\<push>$ and $\<pop>$ must also admit
%  \scriptsize
%  \begin{align*}
%    \<push>(0)_1\ \<pop>_2\ \<ret>_1\ \<ret>(0)_2
%    \text{ \normalsize and }
%    \<push>(0)_1\ \<pop>_2\ \<pop>_3\ \<ret>_1\ \<ret>(0)_2
%    \text{\normalsize,}
%  \end{align*}
%  \normalsize
%  among others, yet need not admit an execution
%  \scriptsize
%  \begin{align*}
%    \<push>(0)_1\ \<pop>_2\ \<pop>_3\ \<ret>_1\ \<ret>(0)_3\ \<ret>(0)_2
%  \end{align*}
%  \normalsize
%  with two completed $\<pop>$ operations returning $0$.
%  
%\end{example}




%Systems we consider are labeled transition systems (LTS):
%\begin{dfn}
%An LTS is defined over four-tuples $A=(Q,\Sigma, q_0, \delta)$ where $Q$ is the set of states, $\Sigma$ is the set of transition labels, $q_0 \in Q$ is the initial state and $\delta \subseteq Q \times \Sigma \times Q$ is the transition relation.
%\end{dfn}
%Executions generated by this system are alternating sequence of states and transition labels $\rho = s_0, e_0, s_1,... s_k, e_k,...$ where each $s_i \in Q$, each $e_i \in \Sigma$, $s_0 = q_0$ and each $(s_i, e_i s_{i+1}) \in \delta$. The projection of the sequence $\rho$ over the set $\Pi$ is denoted by $\rho | \Pi$, and it is the maximum subsequence of $\rho$ consisting of elements of $\Pi$. Traces of the LTS are obtained from executions by projecting them over $\Sigma$. For the rest of the paper and in all of the proofs, we consider only finite executions (denoted as $E(A)$) and/or traces (denoted as $Tr(A)$ of the LTSs in focus.

%Libraries are LTSs that provide methods. Let $\mathcal{M}$ be the set of method names and $\mathcal{D}$ be the domain of values as input/output parameters for the methods. Then, this library contains transition labels of the form $inv(m,d,i)$ representing the invocation of method $m \in \mathcal{M}$ with input value $d \in \mathcal{D}$. The third field is the operation identifier for differentiating the different calls of the same method from the set $\mathcal{O}$. For simplicity, we take $\mathcal{O} = \mathbb{N}$ for the rest of the paper. We also assume that methods could have at most one input parameter. If they do not have any input arguments (like pop method of a stack), we can omit the second field from the action. They also provide actions of the form $ret(m,d,i)$ representing the return of method $m \in M$ with value $d \in D$ which has been invoked previously with action $inv(m,d',i)$. Again, we assume that the methods can return at most one parameter and we may omit the second field from the action if they have none (like enqueue method of a queue). Before starting to reason about any set of libraries, we first fix the sets $\mathcal{M}$ and $\mathcal{D}$ and libraries in our focus agree on this sets. For any transition label $e = inv(m,d,i)$ or $e=ret(m,d,i)$, we have the function $oid(e) = i$.
%
%Since libraries are LTSs, they produce traces. A trace $e = e_1, e_2, ..., e_n$ of library $L$ is \emph{well-formed} iff (i) every return matches an earlier invocation: $e_j = ret(m,d,k)$ implies that there exists $i<j$ such that $e_i = inv(m,d',k)$ and (ii) every operation identifier is used at most one invocation/return pair: $oid(e_i) = oid(e_j) = k$ and $i<j$ implies $e_i = inv(m,d,k)$ and $e_j = ret(m,d',k)$. From now on, we assume that libraries produce well-formed traces. Let $f: \mathbb{N} \rightarrow \mathbb(N)$ be a bijection. Then, traces $e$ and $e'$ are equivalent if $e'$ is obtained from $e$ by replacing every action $inv(m,d,k)$ with $inv(m,d,f(k))$ and every action $ret(m,d,k)$ with $ret(m,d,f(k))$. 

