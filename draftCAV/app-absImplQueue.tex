%!TEX root = draft.tex
\section{Proof of Theorem~\ref{th:absImplQueue}}\label{app:absImplQueue}

\begin{figure} [t]
{\scriptsize
  \centering
  \begin{mathpar}
    \inferrule[call-enq]{
      k\not\in dom(cp^0) \\ 
      d\neq {\tt EMPTY}
    }{
      \sigma,in^0,rv^0,cp^0
      \xrightarrow{inv(enq,d,k)}
      %O\cup\{k\},<\cup \{(k',k): \ell_2(k')={\tt COMP}\},\ell[k\mapsto (d,{\tt PEND})],rv,cp[k\mapsto 1]
      \sigma,in^0[k\mapsto d], rv^0,cp^0[k\mapsto A_1]
    }\hspace{5mm}
    \inferrule[lin-enq]{
      cp^0(k)=A_1
    }{
      \sigma,in^0,rv^0,cp^0
      \xrightarrow{lin(enq,d,k)}
      %O\cup\{k\},<\cup \{(k',k): \ell_2(k')={\tt COMP}\},\ell[k\mapsto (d,{\tt PEND})],rv^0,cp[k\mapsto 1]
      d\cdot\sigma,in^0,rv^0,cp^0[k\mapsto A]
    }\hspace{5mm}
    
        \inferrule[ret-enq]{
      cp^0(k)=A
    }{
      \sigma,in^0,rv^0,cp^0
      \xrightarrow{ret(enq,k)}
      %O\cup\{k\},<\cup \{(k',k): \ell_2(k')={\tt COMP}\},\ell[k\mapsto (d,{\tt PEND})],rv^0,cp[k\mapsto 1]
      \sigma,in^0,rv^0,cp^0[k\mapsto A_2]
    }\hspace{5mm}


    \inferrule[call-deq]{
      k\not\in dom(cp^0) \\ 
    }{
      \sigma,in^0,rv^0,cp^0
      \xrightarrow{inv(deq,k)}
      %O\cup\{k\},<\cup \{(k',k): \ell_2(k')={\tt COMP}\},\ell[k\mapsto (d,{\tt PEND})],rv^0,cp[k\mapsto 1]
      \sigma,in^0,rv^0,cp^0[k\mapsto R_1]
    }\hspace{5mm}
        \inferrule[lin-deq1]{
      cp^0(k)=R_1 \\
      \sigma = \sigma'\cdot d 
    }{
      \sigma,in^0,rv^0,cp^0
      \xrightarrow{lin(deq,d,k)}
      %O\cup\{k\},<\cup \{(k',k): \ell_2(k')={\tt COMP}\},\ell[k\mapsto (d,{\tt PEND})],rv^0,cp[k\mapsto 1]
      \sigma',in^0,rv^0[k\mapsto d],cp^0[k\mapsto R_2]
    }\hspace{5mm}

        \inferrule[lin-deq2]{
      cp^0(k)=R_1 \\
      \sigma = \epsilon
    }{
      \sigma,in^0,rv^0,cp^0
      \xrightarrow{lin(deq,{\tt EMPTY},k)}
      %O\cup\{k\},<\cup \{(k',k): \ell_2(k')={\tt COMP}\},\ell[k\mapsto (d,{\tt PEND})],rv^0,cp[k\mapsto 1]
      \sigma,in^0,rv^0[k\mapsto {\tt EMPTY}],cp^0[k\mapsto R_2]
    }\hspace{5mm}
    \inferrule[ret-deq]{
      cp^0(k)=R_2 \\
      rv^0(k) = d
    }{
      \sigma,in^0,rv^0,cp^0
      \xrightarrow{ret(deq,d,k)}
      %O\cup\{k\},<\cup \{(k',k): \ell_2(k')={\tt COMP}\},\ell[k\mapsto (d,{\tt PEND})],rv^0,cp[k\mapsto 1]
      \sigma,in^0,rv^0,cp^0[k\mapsto R_3]
    }\hspace{5mm}
          \end{mathpar}
  }
 \vspace{-4mm}
  \caption{The transition relation of $AbsQ_0$. 
  %\textcolor{red}{Call-Enq must have $d!= \texttt{EMPTY}$ as a premise. Also lin deq returning empty must be changed as before.}
  }
  \label{fig:transitions:AbsQ_0}
\vspace{-2mm}
\end{figure}

We show that $AbsQ$ and $AbsQ_0$ refine each other. We start by giving a formal definition of the standard reference implementation $AbsQ_0$.
Thus, the states of $AbsQ_0$ are tuples $\tup{\sigma,in^0,rv^0,cp^0}$ where $\sigma\in\<Vals>^*$ is a sequence of values, $in^0:\<Ops> ~> \<Vals>$ records the input value of an enqueue, $rv^0:\<Ops> ~> \<Vals>$ records the return value of a dequeue fixed at its linearization point ($~>$ denotes a partial function), and $cp^0:\<Ops> ~> \{A_1,A,A_2,R_1,R_2,R_3\}$ records the control point of every enqueue ($A_1, A,A_2$) or dequeue operation ($R_1,R_2,R_3$).
All the components are $\emptyset$ in the initial state, and the transition relation $->$ is defined in Fig.~\ref{fig:transitions:AbsQ_0}. The alphabet of $AbsQ$ contains call/return actions and enqueue/dequeue linearization points.

To prove that $AbsQ$ is a refinement of $AbsQ_0$ we define a normal $C\cup R\cup Lin(deq)$-backward simulation (i.e, a backward simulation as in Definition~\ref{def:back_app}) from $AbsQ$ to $AbsQ_0$. The reverse is shown using a normal $C\cup R\cup Lin(deq)$-forward simulation (i.e, a forward simulation as in Definition~\ref{def:for_app}).


%In this section, we will show that for any concurrent queue implementation library $L_C$ for which we know the linearization points of the dequeue operation and that is linearizable with respect to the reference implementation library $L_A$, there exists a forward simulation $fs$ relating $L_C$ to $L_I$ where $L_I$ is an intermediate library equivalent to $L_A$ i.e. $L_I$ refines $L_A$ and vice versa.  
%
%For all the queue libraries, we fix $\mathcal{M} = \{ enq, deq \}$ and $\mathcal{D} = \mathbb{N} \cup \{\texttt{EMPTY} \}$. Since we know the linearization point of dequeues, we extend the definitions of the previous sections adding this information. First, we extend the set $A\Sigma$ introduced in Definition 2 for queues in our focus as $AQ\Sigma = A\Sigma \cup \{lin(deq,d,k)| d \in \mathcal{D}, k \in \mathbb{N}\}$. For any library $L$ we consider in this section, $AQ\Sigma \subseteq \Sigma_L$. Then, we define q-refinement by replacing $A\Sigma$ with $AQ\Sigma$ in Definition 2. We also define q-linearizability, by enforcing $lin(deq,d,k)$ to appear immediately after $inv(deq,k)$ and immediately before $ret(deq,d,k)$ in a sequential execution. We also change definition of forward and backward simulation relations by replacing $A\Sigma$ with $AQ\Sigma$ in the original definitions and adding a new condition $(ii-d)$ to each of them:
%\begin{itemize}
%\item[Forward Simulation: (ii-d)] If $(s, lin(deq,d,k), s') \in \delta_{L_1}$ and $u \in fs[s]$, then there exists $u' \in fs[s']$ such that $(u, lin(deq,d,k), u') \in \delta_{L_2}$
%\item[Backward Simulation: (ii-d)] If $(s, lin(deq,d,k), s') \in \delta_{L_1}$ and $u' \in bs[s']$, then there exists $u \in bs[s]$ such that $(u, lin(deq,d,k), u') \in \delta_{L_2}$
%\end{itemize}
%Lemma 1 of the previous section still holds if we replace refinement with q-refinement and use new definitions of backward and forward simulations.
%
%Next, we define the abstract library $L_A$ as follows:
%\begin{itemize}
%\item A queue state consists of a finite sequence of natural numbers representing the queue content and a program counter for each operation. More formally: $Q_A \subseteq \mathbb{N}^*  \times (\mathbb{N} \rightarrow Lbl_A)$ where $Lbl_A = \{N, E_0, E_1, E_2, D_0, D_1, D_2 \}$ is the set of transition labels of the operations. Operations that have not started yet are mapped to $N$. $E_i$ ($D_i$) denotes particular transitions in enqueue (dequeue) operations that will be clear when we define $\delta_A$. For a state $q$, we denote the contents of the queue (first field) with $s_q$ and the function that maps operations to labels (the second field) with $f_q$.
%\item Transition labels consists of invocations, returns and linearizations of enqueue and dequeue operations: $\Sigma_A = AQ\Sigma \cup \{lin(enq,d,k)| d \in \mathcal{D}, k \in \mathbb{N}\}$. Invocation of $deq$ operation does not have any input and return of $enq$ operation does not have any output value. We omit second fields from these labels.
%\item Initial state is the empty queue: ${q_0}_A = (\langle \rangle, f_{{q_0}_A})$ where $f_{{q_0}_A}(i) = N$ for all $i \in \mathbb{N}$.
%\item Each operation consists of invocation, linearization and return steps. These are reflected in the transition relation. For the below definitions, unchanged parts of the state are omitted from the formulae:
%\begin{itemize}
%\item $(q, inv(enq,d,k), q') \in \delta_A$ iff $ d \neq \texttt{EMPTY} \wedge f_q(k) = N \wedge f_{q'}(k) = E_0$,
%\item $(q,lin(enq,d,k),q') \in \delta_A$ iff $f_q(k) =E_0 \wedge s_{q'} = s_q \circ \langle d \rangle \wedge f_{q'}(k) = E_1$ where $\circ$ is the operation that concatenates two finite sequences,
%\item $(q,ret(enq,k),q') \in \delta_A$ iff $f_q(k) = E_1 \wedge f_{q'}(k) = E_2$,
%\item $(q, inv(deq,k), q') \in \delta_A$ iff $f_q(k) = N \wedge f_{q'}(k) = D_0$
%\item $(q,lin(deq,d,k),q') \in \delta_A$ iff $f_q(k) =D_0 \wedge (d \neq \texttt{EMPTY} \wedge s_q = \langle d \rangle \circ s_{q'} \vee d=\texttt{EMPTY} \wedge s_q = s_{q'} = \langle \rangle )\wedge f_{q'}(k) = D_1$, 
%\item $(q,ret(deq,d,k),q') \in \delta_A$ iff $f_q(k) = D_1 \wedge f_{q'}(k) = D_2$.
%\end{itemize}
%
%\end{itemize} 
%We restrict traces generated by this LTS by neglecting the invalid traces. If we project a trace to some operation identifier $k$ and obtain a sequence $\langle ..., inv(enq,d,k), lin(enq,d',k),...\rangle$ or $\langle ..., lin(deq,d,k), ret(deq,d',k),...\rangle$ where $d \neq d'$ we say that this trace is invalid.
%
%We want to introduce $L_I$ as the next step. States of $L_I$ consists of strict orders of enqueue operations. Nodes of the strict order come from the set $ND = \mathbb{N} \times \mathcal{D} \times \{ \texttt{PENDING}, \texttt{CLOSED}\}$. Basically each node is a tuple keeping the operation ID, the value to be enqueued and if this enqueue is pending or closed. Strict order also keeps a set of directed edges $ ed \subseteq ED = ND \times ND$. Then, a strict order is a tuple $(nd, ed)$ where the set $ed$ obeys the assumption of strict order i.e. irreflexivity, asymmetry and transitivity. Then, the LTS of $L_I$ is defined as follows:
%\begin{itemize}
%\item A state consists of a partial order and a function keeping the program counter. More formally $Q_I \subseteq ND \times ED \times (\mathbb{N} \rightarrow Lbl_I)$ where $Lbl_I = \{N, E_0, E_1, D_0, D_1, D_2\} $ is the set of transition labels. $nd_q$ $ed_q$ and $f_q$ denotes the nodes of the strict order in state $q$, edges of the strict order in state $q$ and the function that maps operation to labels in state $q$, respectively. 
%\item The transition label set consists of invocation and return action of both methods and linearization action of only the dequeue method: $\Sigma_I = AQ\Sigma$. Invocation of $deq$ operation does not have any input and return of $enq$ operation does not have any output value. We omit second fields from these labels.
%\item Initial state consists of an empty strict order and a function mapping every operation to $N$: ${q_0}_{I} = (so_{{q_0}_I}, f_{{q_0}_I}$ where $so_{{q_0}_I} =(\emptyset, \emptyset)$ and $f_{{q_0}_I}(i) = N$ for all $i \in \mathbb{N}$.
%\item While defining $\delta_I$, we again omit mentioning about the parts that has not changed:
%\begin{itemize}
%\item $(q, inv(enq,d,k), q') \in \delta_I$ iff $f_q(k) = N \wedge f_{q'}(k) = E_0 \wedge d \neq \texttt{EMPTY} \wedge (k,\_,\_) \notin nd_q \wedge nd_{q'} = nd_q \cup \{ (k,d,\texttt{PENDING}\} \wedge AddNode(ed_{q'}, ed_q,k)$ where $AddNode(ed_{q'}, ed_q,k)$ is true iff $ed_{q'}$ is obtained from $ed_q$ by adding edges from every closed node of $ed_q$ to the node with identifier $k$.
%\item $(q, ret(enq,k), q') \in \delta_I$ iff $f_q(k) =E_0 \wedge f_{q'}(k) = E_1 \wedge ( (k,\_,\texttt{PENDING}) \notin nd_q \wedge so_q = so_{q'} \vee UpdateNode(so_{q'}, so_q, k) )$ where $UpdateNode(so_{q'}, so_q, k)$ is true iff $(k,d,\texttt{PENDING}) \in nd_q$ for some $d \in \mathcal{D}$, it is replaced with node $(k,d,\texttt{CLOSED}$ in the state $q'$ and all the edges adjacent to $(k,d,\texttt{PENDING})$ in state $q$ are replaced with edges adjacent to $(k,d,\texttt{CLOSED})$ in the state $q'$.
%\item $(q, inv(deq,k), q') \in \delta_I$ iff $f_q(k) = N \wedge f_{q'}(k) = D_0$.
%\item $(q, lin(deq,d,k), q') \in \delta_I$ iff $f_q(k) = D_0 \wedge f_{q'}(k) = D_1 \wedge (d = \texttt{EMPTY} \wedge nd_q = ed_q = nd_{q'} = ed_{q'} \emptyset \vee d \neq \texttt{EMPTY} \wedge RemoveNode(so_q, so_{q'},d))$ where $RemoveNode(so_q, so_{q'},d)$ is true iff there is a minimal node $n \in so_q$ of which data value (second field) is $d$ and $so_{q'}$ is obtained from $so_q$ by removing $n$ and all the edges adjacent to it. Note that linearization of dequeue could remove a \texttt{PENDING} node.  
%\item $(q, ret(deq,d,k) q') \in \delta_I$ iff $f_q(k) = D_1 \wedge f_{q'}(k) = D_2$.
%\end{itemize}
%\end{itemize}
%Again, we omit the inconsistent traces from $L_I$ as in $L_A$. Note that the library $L_I$ is deterministic with respect to alphabe $AQ\Sigma$.
%
%Next, we show equivalence of $L_A$ and $L_I$ in terms of q-refinement.
\begin{lemma} 
$AbsQ$ is a refinement of $AbsQ_0$.
\end{lemma}
\begin{proof}
We define a normal $C\cup R\cup Lin(deq)$-backward simulation $bs$ from $AbsQ$ to $AbsQ_0$ as follows. Given an $AbsQ$ state $s=\tup{O,<,\ell,rv,cp}$ and an $AbsQ_0$ state $t=\tup{\sigma,in^0,rv^0,cp^0}$ we have that $(s,t)\in bs$ iff the following hold:
\begin{itemize}
	\item the sequence $\sigma$ is a linearization of a partial order $(D,\prec)$ where $D$ contains values labeling elements of $O$ and all the values corresponding to completed enqueues, i.e., $\ell_1({\tt COMP}(O))\subseteq D\subseteq \ell_1(O)$ ordered according to the happens-before order between the enqueues that added them, i.e., $d_1\prec d_2$ if{f} there exists $k_1,k_2$ such that $\ell_1(k_1)=d_1$, $\ell_1(k_2)=d_2$, and $k_1 < k_2$.
	\item the return values fixed at dequeue linearization points are the same, i.e., for every $k$, $rv(k)=rv^0(k)$,
	\item every dequeue is at the same control point in both $s$ and $t$, i.e., for every $k$ and $i\in \{1,2,3\}$, $cp(k)=R_i$ iff $cp^0(k)=R_i$,
	\item every pending enqueue has the same input value in both $s$ and $t$, i.e., for every $k$, $\ell_1(k)=in^0(k)$,
	\item a pending enqueue from $O$ has been linearized whenever its value is contained in $\sigma$, i.e., for every $k$, $cp^0(k)=A$ if $\ell_1(k)\in D$ and $\ell_2(k)={\tt PEND}$, 
	\item a pending enqueue from $O$ hasn't been linearized whenever its value is not in $\sigma$, i.e., for every $k$, $cp^0(k)=A_1$ iff $\ell_1(k)\not\in D$ and $\ell_2(k)={\tt PEND}$, 
	\item a pending enqueue which is not in $O$ has been linearized, i.e., for every $k$, $cp^0(k)=A$ if $k\not\in O$ and $cp(k)=A_1$, 
	\item an enqueue is completed in $s$ whenever it is completed in $t$, i.e., for every $k$, $cp(k)=A_2$ iff $cp^0(k)=A_2$,
\end{itemize}

For the conditions described above, if we fix the set $D$ and $\sigma_t$, then the state $t$ related to $s$ becomes unique. We use this fact in the proof. In some places, we only give $D$, $\sigma_t$ and $s$ without explicitly defining $t$ or show that there exists $t$ with the given $\sigma_t$ that is related to $s$ by just finding a $D$ such that $\sigma_t$ is a linearization of $(D, \prec)$ where $\prec$ is induced from $<_s$.

or $\sigma_t$ and not describing $t$ explicitly.

In the following, we show that indeed $bs$ is a normal $C\cup R\cup Lin(deq)$-backward simulation from $AbsQ$ to $AbsQ_0$. % mainly focusing on how to pick $\sigma$ fields of related states in $AbsQ_0$ and how the corresponding actions of $AbsQ_0$ modifies $\sigma$.

%TODO FILL IN THE STEPS
%We will provide a backward simulation relation btw $L_I$ and $L_A$. Given a state $q = (so_q, f_q) \in Q_I$,  $q' \in bs[q]$ iff (i) $s_{q'}$ is a linearization of $so_q$ projected to $d$ fields (second field). This linearization may omit the pending elements (the ones of which third field is \texttt{PENDING}), but it must contain all the \texttt{CLOSED} elements. Note that pending elements of $so_q$ are maximal and they can appear after the closed elements, towards the end of the sequence. (ii) If $f_q(k) \in \{N, D_0, D_1, D_2\}$, then $f_{q'}(k) = f_q(k)$ . If $f_q(k) = E_0$, $(k,d,\texttt{PENDING}) \in nd_q$ and this node participates in the linearization $s_{q'}$. Then $f_{q'}(k) = E_1$. If it does not participate in the linearization, then $f_{q'}(k) = E_0$. If $f_q(k) = E_0$ but there is no node with operation identifier $k$ in $nd_q$, then $f_{q'} = E_1$. Lastly, if $f_q(k) = E_1$ then $f_{q'}(k) = E_2$. Next, we will show that $bs$ is a backward simulation relation.
\begin{itemize}
\item[$\langle i \rangle$] $bs[s^{AbsQ}_0] = \{ s^{AbsQ_0}_0 \}$.

\item[\textsc{call-enq}] Let $s \xrightarrow{inv(enq,d,k)}_{AbsQ} s'$ and $t' \in bs[s']$. Either $k \in D_{t'}$ or not. 

First consider the former case. We know that $\ell_{s'}(k) = (d, \texttt{PEND})$ and $k$ is maximal in $s'$. Hence $\sigma_{t'} = \rho \circ \langle d \rangle \circ \pi$ where $\pi$ contains linearization of pending elements in $O_{s'}$. Then, pick $\sigma_t = \rho$. We can find such a $t \in bs[s]$ with $\sigma_t$. Let $(D, \prec)$ be the partial order that is used while constructing $\sigma_{t'}$ from $O_s$ and $<_s$. We can find $(D', \prec')$ for relating $s$ to $t$ such that $D'$ does not contain the values of pending elements that formed $\pi$ suffix of $s_{t'}$ and $d$ coming from linearization of $k \in O_{s'}$. 

One can also see that $t \xrightarrow{\alpha}_{AbsQ_0} t'$ where $\alpha = inv(enq,d,k), lin(enq,d,k), \\lin(enq,d_1,k_1), ..., lin(enq,d_j,k_j)$ such that $\pi = d_1,...,d_j$ and $k_1,...,k_j \in O_{s'}$ are the pending elements that are  linearized to form $\pi$. Note that $\alpha$ obeys the definition of normal backward simulation definition.

For the second case, pick $t$ such that $\sigma_t = \sigma_{t'}$. We can find a $t$ with $\sigma_t$ related to $s$ by $bs$ using the same $(D,\prec)$ partial order that is used while relating $s'$ to $t'$. $\ell_1(\texttt{COMP}(O_s)) \subseteq D$ holds because $\texttt{COMP}(O_s) = \texttt{COMP}(O_{s'})$.
%\item[$\langle ii-a-enq \rangle$] Let $(q, inv(enq,d,k), q') \in \delta_I$ and $t' \in bs[q']$. We know that $(k,d,\texttt{PENDING}$ is a maximal element in $so_{q'}$ and either $s_{t'}$ does not linearize it or $s_{t'} = \rho \circ \langle d \rangle \circ \pi$ where $\pi$ consists of only linearization of \texttt{PENDING} elements. For the first case $s_{t'} = s_t$ for some $t \in bs[q]$ and $(t,inv(enq,d,k),t') \in \delta_A$. For the latter case, $s_t = \rho$ for some $t \in bs[q]$ and $t \xrightarrow{a}_{L_A} t'$ where $a = inv(enq,d,k), lin(enq,d,k), lin(enq,d_1,k_1),...,lin(enq,d_j,k_j)$ where $\pi = d_1,...d_j$. The $f_t$ could be obtained easily by observing the sequence $a$ and it can be checked that such $t \in bs[q]$ exists.

\item[\textsc{call-deq}] Let $s \xrightarrow{inv(deq,d,k)}_{AbsQ} s'$ and $t' \in bs[s']$. Pick $t$ such that it is equal to $t'$ in every field except that $k \notin dom(cp^0_t)$. Then, $t \in bs[s]$ and $t \xrightarrow{inv(deq,d,k)_{AbsQ_0}} t'$.
%\item[$\langle ii-a-deq \rangle$] Let $(q, inv(deq,k), q') \in \delta_I$ and $t' \in bs[q']$. We know that $f_{t'}(k) = D_0$. We know that there is a $t \in bs[q]$ such that $s_t = s_{t'}$ (since $so_q = so_{q'}$), $f_t(k) = N$ and $(t,inv(deq,k),t') \in \delta_A$.

\item[\textsc{lin-deq1}]  Let $s \xrightarrow{lin(deq,d,k)}_{AbsQ} s'$, $t' \in bs[s']$ and $d \neq \texttt{EMPTY}$. We pick $t$ such that $\sigma_t = \langle d \rangle \circ \sigma_{t'}$. We first show that $t \in bs[s]$. Let $(D, \prec)$ be the partial order that is linearized to obtain $\sigma_{t'}$ and $k' \in O_s$ be the element such that ${\ell_s}_1(k') = d$. We know that $k'$ is minimal in $<_s$ due to the premise of the rule \textsc{lin-deq1}. Hence, we can obtain $(D', \prec')$ such that $D' = D \cup \{{\ell_s}_1(k')\}$ and $\sigma_t$ is a linearization of it.

In addition, $t \xrightarrow{lin(deq,d,k)}_{AbsQ_0} t'$. The action $lin(deq,d,k)$ is enabled in state $t$ since $d$ is the minimum element of $\sigma_t$. Note that the transition relating $t$ to $t'$ obeys the definition of normal forward simulation.

\item[\textsc{lin-deq2}]  Let $s \xrightarrow{lin(deq,\texttt{EMPTY},k)}_{AbsQ} s'$ and $t' \in bs[s']$. We pick $(D, \prec)$ for relating $s$ to $t$ such that $D = \emptyset$. Such a $D$ is a valid choice since all the elements $O_s$ are pending. Then, $\sigma_t = \langle \rangle$ is the only linearization of $(D, \prec)$. Hence, $lin(deq,\texttt{EMPTY},k)$ action is enabled in $AbsQ_0$ and $t \xrightarrow{lin(deq,\texttt{EMPTY},k)}_{AbsQ_0} t'$ holds.
%\item[$\langle ii-d \rangle$] Let $(q,lin(deq,d,k),q') \in \delta_I$ and $t' \in bs[q']$. First consider the case $d \neq \texttt{EMPTY}$. We have two cases: Either the node with operation id $k$ in the $so_q$ was \texttt{PENDING} or \texttt{CLOSED}. For both of the cases, we can find $t \in bs[q]$ such that $s_t = \langle d \rangle \circ s_{t'}$ and $f_t(k) = D_0$. Hence $(t,lin(deq,d,k),t') \in \delta_I$. Next, consider the case $d = \texttt{EMPTY}$. Then, we know that $s_{t'} = \langle \rangle$ and there exists $t \in bs[q]$ such that $s_t = \langle \rangle$ and $f_t = D_0$. Hence $(t,lin(deq,d,k),t') \in \delta_A$.

\item[\textsc{ret-enq1}]  Let $s \xrightarrow{ret(enq,k)}_{AbsQ} s'$, $\ell_s(k) = (d,\texttt{PEND})$ and $t' \in bs[s']$. Assume $(D, \prec)$ be the partial order of which linearization is $\sigma_{t'}$. Pick $D'=D$. Then, $\ell_1(\texttt{COMP}(O_s)) \subseteq D \subseteq \ell_1(O_s)$ holds since $\texttt{COMP}(O_s) = \texttt{COMP}(O_{s'}) \setminus \{k\}$ and $k \in \texttt{PEND}(O_s)$. Construct $t \in bs[s]$ such that $\sigma_t = \sigma_{t'}$ is obtained by linearizing the partial order $(D',\prec')$. Then, $t \xrightarrow{ret(enq,k)}_{AbsQ_0} t'$ holds and it is a valid action with respect to normal backward-simulation relation definition. 

\item[\textsc{ret-enq2}]  Let $s \xrightarrow{ret(enq,k)}_{AbsQ} s'$, $k \notin O_s$ and $t' \in bs[s']$. Since $O_s = O_{s'}$ and $\texttt{COMP}(O_s) = \texttt{COMP}(O_{s'})$, we can pick $D' = D$ where $(D, \prec)$ is the strict partial order such that $\sigma_{t'}$ is its linearization. Construct $t \in bs[s]$ such that $\sigma_t = \sigma_{t'}$ is obtained by linearizing the partial order $(D',\prec')$. Then, $t \xrightarrow{ret(enq,k)}_{AbsQ_0} t'$ holds and it is a valid action with respect to normal backward-simulation relation definition. 
%\item[$\langle ii-b-enq \rangle$] Let $(q,ret(enq,k),q') \in \delta_I$ and $t' \in bs[q']$. There are two possible cases: There exists a node $(k,d,\texttt{PENDING}) \in nd_q$ or for all nodes $n = (k,\_,\_)$, $n \notin nd_q$. For the former case, $s_{t'} = \rho \circ \langle d \rangle \circ \pi$ where $\rho$ is linearization of closed nodes in $nd_{q'}$ and $\pi$ is linearization of some open nodes in $node_{q'}$. We also know that $f_{t'}(k) = E_2$. Then, there exists a node $t \in bs[q]$ such that $s_t = s_{t'}$ (since nodes that generate $\rho$ are closed in $q$ and that generate $\pi$ are open in $q$) and $f_t(k) = E_1$. Hence, $(t,ret(deq,k),t') \in \delta_A$. For the latter case, we know that $so_q = so_{q'}$. Therefore, there exists $t \in bs[q]$ such that $s_t = s_{t'}$. Moreover $f_t(k) = E_1$. Hence, $(t,ret(enq,k),t') \in \delta_A$.

\item[\textsc{ret-deq}]  Let $s \xrightarrow{ret(deq,d,k)}_{AbsQ} s'$ and $t' \in bs[s']$. Assume $(D, prec)$ is the partial order of which linearization is $\sigma_{t'}$. Construct $t \in bs[s]$ such that $\sigma_t = \sigma_{t'}$ and $(D, \prec
)$ is the partial order $\sigma_t$ is obtained from. $\texttt{COMP}(O_s) \subseteq D \subseteq O_s$ since $\texttt{COMP}(O_s) = \texttt{COMP}(O_{s'})$ and $O_s = O_{s'}$. Then, $t \xrightarrow{ret(deq,d,k)}_{AbsQ_0} t'$ holds. We have $rv_s(k) = rv^0_t(k)$ since $t \in bs[s]$. Hence the $ret(deq,d,k)$ is enabled in $t$. Moreover, $ret(deq,d,k)$ is a valid transition with respect to the normal backward simulation relation definition.
%\item[$\langle ii-b-deq \rangle$] Let $(q,ret(deq,d,k), q') \in \delta_I$ and $t' \in bs[q']$. Then, we know that $f_{t'}(k) = D_2$ and there exists a $t \in bs[q]$ such that $s_t = s_{t'}$ (since $so_q = so_{q'})$ and $f_t(k) = D_1$. Then, $(t,ret(deq,d,k) \in \delta_A$.
\end{itemize}
\end{proof}

%\begin{lem}
%$L_A$ is a q-refinement of $L_I$.
%\end{lem}
%\begin{proof}
\begin{lemma} 
$AbsQ_0$ is a refinement of $AbsQ$.
\end{lemma}
\begin{proof}
We define a normal $C\cup R\cup Lin(deq)$-forward simulation $fs$ from $AbsQ_0$ to $AbsQ$ as follows. 
Given $AbsQ_0$ state $t=\tup{\sigma,in^0,rv^0,cp^0}$ and an $AbsQ$ state $s=\tup{O,<,\ell,rv,cp}$ we have that $(t,s)\in fs$ iff the following hold:
\begin{itemize}
	\item the sequence $\sigma$ is a linearization of a partial order $(D,\prec)$ where $D$ contains values labeling elements of $O$ and all the values corresponding to completed enqueues, i.e., $\ell_1({\tt COMP}(O))\subseteq D\subseteq \ell_1(O)$ ordered according to the happens-before order between the enqueues that added them, i.e., $d_1\prec d_2$ if{f} there exists $k_1,k_2$ such that $\ell_1(k_1)=d_1$, $\ell_1(k_2)=d_2$, and $k_1 < k_2$.
	\item every dequeue is at the same control point in both $s$ and $t$, i.e., for every $k$ and $i\in \{1,2,3\}$, $cp(k)=R_i$ iff $cp^0(k)=R_i$,
	\item every enqueue is pending in $s$ whenever it is pending in $t$, i.e., for every $k$, $cp(k)=A_1$ iff $cp^0(k)\in \{A_1,A\}$,
	\item every enqueue is completed in $s$ whenever it is completed in $t$, i.e., for every $k$, $cp(k)=A_2$ iff $cp^0(k) = A_2$,
	\item every pending enqueue which is not linearized or whose value is present in $\sigma$ is a member of $O$, i.e., for every $k$, 
	\begin{align*}
	&k\in O\land \ell(k)=(d,{\tt PEND})\mbox{ iff } \\
	&\hspace{2cm}(cp^0(k)=A_1\land in^0(k)=d)\vee (\exists i.\ \sigma_i =d\land cp^0(k)=A\land in^0(k)=d)
	\end{align*}
	\item every completed enqueue whose value is present in $\sigma$ is a member of $O$, i.e., for every $k$, 
	\begin{align*}
	&k\in O\land \ell(k)=(d,{\tt COMP})\mbox{ iff } \exists i.\ \sigma_i =d\land cp^0(k)=A_2\land in^0(k)=d
	\end{align*}
	\item pending enqueues are maximal in $<$, i.e., for every $k$ and $k'$, $k \not< k'$ if $\ell_2(k)={\tt PEND}$,
	\item the return values fixed at dequeue linearization points are the same, i.e., for every $k$, $rv(k)=rv^0(k)$.
\end{itemize}
In the following, we show that indeed $fs$ is a normal $C\cup R\cup Lin(deq)$-forward simulation from $AbsQ_0$ to $AbsQ$.

%TODO FILL IN THE STEPS
%Since $L_I$ is deterministic with respect to the alphabet $AQ\Sigma$, we should be able to find a forward simulation between $L_A$ and $L_I$ if our claim is correct. We propose a forward simulation by adding some auxiliary varibles to the state of $L_A$. In the new augmented state of $L_A$, the sequence does not only keep the elements of the queue but also the operation identifiers that enqueued them. Hence, the sequence consists of  pairs of the form $s_q(i) = (k,d)$ where $q \in Q_A$, $i \in \mathbb{N}$ is an index, $d \in \mathcal{D}\backslash \{ \texttt{EMPTY} \}$ is a value and $k \in \mathbb{N}$ is an operation identifier. We also add a set $InvEnq_q$ component to state that keeps the enqueue operations at the point $E_0$ i.e. enqueues that are invoked bu have not been linearized yet. Elements of $InvEnq_q$ are pairs of the form $(d,k)$ where  $d \in \mathcal{D}\backslash \{ \texttt{EMPTY} \}$ is a value and $k \in \mathbb{N}$ is an operation identifier. Then our forward simulation $fs \subseteq Q_A \times Q_I$ relates the state $q = (s_q, f_q) \in Q_A$ to a state $q' =(so_{q'}, f_{q'}) \in Q_I$ iff (i) $f_{q'}(k) = f_q(k)$ for all $f_q(k) \in \{N, D_0, D_1, D_2\}$, (ii) $f_q(k) = E_0$ or $f_q(k) = E_1$ implies $f_{q'}(k) = E_0$, (iii) $f_q(k) = E_2$ implies $f_{q'}(k) = E_1$, (iv) If $(d,k) \in InvEnq_q$ or there exists an index $i \in \mathbb{N}$ such that $s_q(i) = (k,d)$ and $f_q(k) = E_1$, then $(k,d,\texttt{PENDING})$ is a node in $so_{q'}$, (v) if there exists an index $i \in \mathbb{N}$ such that $s_q(i) = (k,d)$ and $f_q(k) = E_2$ then $(k,d,\texttt{CLOSED})$ is a node in $so_{q'}$, (vi) open nodes in $so_{q'}$ are maximal and (vii) there exists a linearization (total order) $lo_{q'}$ of $so_{q'}$ that may omit some open nodes of $so_{q'}$ such that if we project nodes of $lo_{q'}$ to the first two fields, this linear order is equal to the sequence $s_q$.
%Next, we will show that $fs$ is a forward simulation relation:

\begin{itemize}
\item[$\langle i \rangle$] $fs[s_0^{AbsQ_0}] = \{s_0^{AbsQ}\}$
\item[\textsc{call-enq}] Let $t \xrightarrow{inv(enq,d,k)}_{AbsQ_0} t'$ and $s \in fs[t]$. Then, $inv(enq,d,k)$ is an enabled action in $AbsQ$ since premise of \textsc{call-enq} holds in $t$ and $s \in fs[t]$. Obtain $s'$ such that $s \xrightarrow{inv(enq,d,k)}_{AbsQ} s'$. Note that $s'$ is unique since $AbsQ$ is deterministic with respect to $C \cup R \cup Lin(deq)$. 

Next, we show that $s' \in fs[t']$. Let $(D, \prec)$ be the partial order used while relating $t$ to $s$. Same partial order can be used while relating $\sigma_{s'}$ to $t'$ since $\texttt{COMP}(O_s) = \texttt{COMP}(O_{s'}$, $O_s \subseteq O_{s'}$ and $<_s \subseteq <_{s'}$. The only change we have in control point fields after the actions is that $cp^0_{s'}(k)= A_1$ and $cp_{t'}(k)=A_1$ which satisfies the conditions on $fs$. Moreover $k$ is a maximal pending node in $t'$ as required by the $fs$ conditions. Consequently, $s' \in fs[t']$.
%\item[$\langle ii-a-enq \rangle$] Let $(q,inv(enq,d,k),q') \in \delta_A$ and $t \in fs[q]$. We can obtain $t'$ such that $(t,inv(enq,d,k),t') \in \delta_I$. We show that $t' \in fs[q']$ by checking seven conditions of our $fs$ relation. We omit the node $(k,d)$ while linearizing $so_{t'}$ and obtain $s_{q'}$.

\item[\textsc{call-deq}] Let $t \xrightarrow{inv(deq,k)}_{AbsQ_0} t'$ and $s \in fs[t]$. Then, $inv(deq,k)$ is an enabled action in $AbsQ$ since premise of \textsc{call-deq} holds in $t$ and $s \in fs[t]$. Obtain $s'$ such that $s \xrightarrow{inv(deq,k)}_{AbsQ} s'$. Note that $s'$ is unique since $AbsQ$ is deterministic with respect to $C \cup R \cup Lin(deq)$. 

Next, we show that $s' \in fs[t']$. Since $\sigma_s =\sigma_{s'}$, $O_s = O_{s'}$ and $\texttt{COMP}(O_s) = \texttt{COMP}(O_{s'}$, we can pick same $(D,\prec)$  partial order in $s'$ and show that $\sigma_{t'}$ is a linearization of it. The only change in control points after the transitions is that $cp^0_{t'}(k) = cp_{s'}(k) = R_1$ which does not violate any condition in $fs$. Consequently, $s' \in fs[t']$. 
%\item[$\langle ii-a-deq \rangle$] Let $(q, inv(deq,k), q') \in \delta_A$ and $t \in fs[q]$. We obtain $t'$ such that $(t, inv(deq,k),t') \in \delta_I$. The only difference between $t'$ and $t$ is that $f_{t'}(k) = D_0$. We can again see that $t' \in fs[q']$. 

\item[\textsc{lin-enq}] Let $t \xrightarrow{lin(enq,d,k)}_{AbsQ_0} t'$ and $s \in fs[t]$. Then, pick $s' = s$ such that $s \xrightarrow{\epsilon}_{AbsQ} s'$. Note that $\epsilon$ is a valid transition with respect to the normal forward simulation relation definition. We show that $s \in fs[t']$. If $(D, \prec)$ is the partial order in $s$ of which one linearization is $\sigma_t$, we pick $D' = D \cup \{k\} \subseteq O_s$. $(D', \prec')$ can be linearized to $\sigma_{t'}$ since $k$ is a maximal pending node and can be linearized at the end. Moreover, the only change in control point $cp^0_{t'}(k) = A$ which does not violate the $fs$ conditions.
%\item[$\langle ii-c \rangle$] Let $(q,lin(enq,d,k),q') \in \delta_A$ and $t \in fs[q]$. Pick $t' = t$. We can see that $t \in fs[q']$. While obtaining linearization of $so_{t'}$, we do not neglect the node $(k,d, \texttt{PENDING})$ this time, although we neglect it in the linearization of $so_t$.

\item[\textsc{lin-deq1}] Let $t \xrightarrow{lin(deq,d,k)}_{AbsQ_0} t'$, $d \neq \texttt{EMPTY}$ and $s \in fs[t]$. Then, $lin(deq,d,k)$ is an enabled action in $AbsQ$. There must exist $d \in D \subseteq \ell_1(O_s)$ such that ${\ell_s}_1(k') = d$ and $k'$ is minimal in $D$ (since $d$ is linearized as the minimum element in $\sigma_t$ according to premise of \textsc{lin-deq1} of $AbsQ$). Obtain $s'$ such that $s \xrightarrow{lin(deq,d,k)}_{AbsQ} s'$. Note that $s'$ is unique since $AbsQ$ is deterministic with respect to $C \cup R \cup Lin(deq)$. 

Next, we show that $s' \in fs[t']$. Let $(D, \prec)$ be the partial order used while relating $t$ to $s$ such that $\sigma_t$ is a linearization of this partial order. Since we have shown that $k'$ is minimal in that partial order, $\sigma_{t'}$ is a linearization of $(D', \prec')$ where $D' = D \setminus \{\ell_1(k')\}$. Note that $\ell_1(\texttt{COMP}(O_{s'})) \subseteq D' \subseteq \ell_1(O_{s'})$ holds. The only change in control points is that $cp^0_{t'}(k) = cp_{s'}(k) = R_2$ which does not violate the conditions for relating $t'$ to $s'$. Note that the fifth condition of $fs$ still holds for $k'$ while relating $t'$ to $s'$. After transitions $rv^0_{t'}(k) = rv_{s'}(k) = d$ and the last condition on $fs$ is preserved.

\item[\textsc{lin-deq2}] Let $t \xrightarrow{lin(deq,\texttt{EMPTY},k)}_{AbsQ_0} t'$ and $s \in fs[t]$. Then, $lin(deq,d,k)$ is an enabled action in $AbsQ$. If $\texttt{COMP}(O_t) \neq \emptyset$, then $D$ use for linearization cannot be $\emptyset$ $\sigma_t = \langle\rangle$ cannot be a linearization of $(D, \prec)$. Obtain $s'$ such that $s \xrightarrow{lin(deq,\texttt{EMPTY},k)}_{AbsQ} s'$. Note that $s'$ is unique since $AbsQ$ is deterministic with respect to $C \cup R \cup Lin(deq)$. 

Next, we show that $s' \in fs[t']$. Let $(D, \prec)$ be the partial order used while relating $t$ to $s$ such that $\sigma_t = \langle \rangle $ is a linearization of this partial order. We can use the same $(D, \prec)$ for relating $t'$ to $s'$ because $\sigma$ field is the same for both $s$, $s'$; and $O$, $<$, $\ell$ fields are same for both $t$ and $t'$. The only change in control points is that $cp^0_{t'}(k) = cp_{s'}(k) = R_2$ which does not violate the conditions for relating $t'$ to $s'$.  After transitions $rv^0_{t'}(k) = rv_{s'}(k) = d$ and the last condition on $fs$ is preserved.
%\item[$\langle ii-d \rangle$] Let $(q, lin(deq,d,k), q') \in \delta_A$ and $t \in fs[q]$. Obtain $t'$ such that $(t, lin(deq,d,k), t') \in \delta_I$. In $s_q$, $(k,d)$ must be the minimum element. Hence, $(k,d,\_)$ is a minimal node in $so_t$ and $lin(deq,d,k)$ is an enabled action in $L_I$. This action removes the node $(k,d,\_)$ from $so_t$ and changes $f_t(k)=D_0$ to $f_{t'}(k)= D_1$. We can see that $t' \in fs[q']$ by checking the seven conditions.

\item[\textsc{ret-enq}] Let $t \xrightarrow{ret(enq,k)}_{AbsQ_0} t'$ and $s \in fs[t]$. Then, there are two cases assuming data independence: (i) $in^0_t(k)=d$ and $\exists i. \sigma_t(i)= d$ (ii) or not.   

First, consider the former case. Then, \textsc{ret-enq1} rule of $AbSQ$ is applicable. Its precondition holds since fifth condition of $fs$ holds while relating $t$ to $s$. Apply this rule ($ret(enq,k)$) to obtain $s'$. Note that $s'$ is unique since $AbsQ$ is deterministic with respect to $C \cup R \cup Lin(deq)$ and it is a valid action according to the normal forward-simulation relation definition.

Next, we show that $s' \in fs[t']$. Since $\exists i. \sigma_t(i) = d$, we know that $d \in D$ where $(D, \prec)$ is the partial order satisfying first condition of $fs$ while relating $t$ to $s$, and $k \in O_s$ takes part in the linearization i.e., ${\ell_t}_1(k) \in D$. We can use the same partial order $(D,\prec)$ for relating $t'$ to $s'$ such that it satisfies the first condition of $fs$. The only change in control points is that $cp^0_{t'}(k) = cp_{s'}(k) = A_2$ which does not violate the conditions for relating $t'$ to $s'$. Note that the sixth condition of $fs$ also continue to hold for $k$ for the post-states.

Second, consider the latter case:  $in^0_t(k)=d$, but $\forall i. \sigma_t(i)\neq d$. Since $(t,s) \in fs$, $k \notin O_s$ by the fifth and sixth conditions. Hence, the pre-condition of \textsc{ret-enq2} is satisfied by $t$. Apply this rule ($ret(enq,k)$) to obtain $s'$. Note that $s'$ is unique since $AbsQ$ is deterministic with respect to $C \cup R \cup Lin(deq)$ and it is a valid action according to the normal forward-simulation relation definition.

Next, we show that $s' \in fs[t']$. For satisfying the first condition, one can use the same $(D, \prec)$ partial order that is used for relating pre-states since $\sigma$ fields of $t$, $t'$ and $O$, $<$, $\ell$ fields of $s$ and $s'$ are the same.  The only change in control points is that $cp^0_{t'}(k) = cp_{s'}(k) = A_2$ which does not violate the conditions for relating $t'$ to $s'$.
%\item[$\langle ii-b-enq \rangle$] Let $(q, ret(enq,k), q') \in \delta_A$ and $t \in fs[q]$. There are two cases: Either there exists an index $i \in \mathbb{N}$ such that $s_q(i) = (k,d)$ or there is no such $i$. For the former case, there is no node of the form $(k,d,\_)$ in $so_t$. We obtain $t'$ such that $(t,ret(enq,k),t') \in \delta_I$. Then, there is no node of the form $(k,d,\_)$ in $so_{t'}$ neither and $so_{t'} = so_t$. since $s_q = s_{q'}$ also holds for this case, $t' \in fs[q']$ holds. For the latter case, we again pick $t'$ such that $(t, ret(enq,k), t') \in \delta_I$. Again, $so_t = so_{t'}$ and $s_q = s_{q'}$ holds and $t' \in fs[q']$ can be observed.

\item[\textsc{ret-deq}] Let $t \xrightarrow{ret(deq,d,k)}_{AbsQ_0} t'$ and $s \in fs[t]$. Then, $ret(deq,d,k)$ is an enabled action in $AbsQ$ due to premise \textsc{ret-deq} of $AbsQ_0$ and the last condition on $fs$ (since $(t,s) \in fs$). Obtain $s'$ such that $s \xrightarrow{ret(deq,d,k)}_{AbsQ} s'$. Note that $s'$ is unique since $AbsQ$ is deterministic with respect to $C \cup R \cup Lin(deq)$.

We see that $s' \in fs[t']$. Pre-states are equal to the post-states with the only exception in the control points such that $cp^0_{t'}(k) = cp_{s'}(k) = R_3$. All the conditions except the third one continues to hold in the post states since they hold in the pre-states. The third rule regarding the control points of dequeues also continue to the hold since changes in the control point of $k$ does not violate it.
%\item[$\langle ii-b-deq \rangle$] Let $(q, ret(deq,d,k), q') \in \delta_A$ and $t \in fs[q]$. We pick $t'$ such that $(t, ret(deq,d,k), t') \in \delta_I$. We know that $so_t = so_{t'}$ and only change is $f_{t'}(k) = D_2$. Since we know that $s_q = s_{q'}$ and $f_{q'}(k) = D_2$, $t' \in fs[q']$ holds.  
\end{itemize}
\end{proof}